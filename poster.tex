%%%%%%%%%%%%%%%%%%%%%%%%%%%%%%%%%%%%%%%%%
% Jacobs Landscape Poster
% LaTeX Template
% Version 1.1 (14/06/14)
%
% Created by:
% Computational Physics and Biophysics Group, Jacobs University
% https://teamwork.jacobs-university.de:8443/confluence/display/CoPandBiG/LaTeX+Poster
% 
% Further modified by:
% Nathaniel Johnston (nathaniel@njohnston.ca)
%
% This template has been downloaded from:
% http://www.LaTeXTemplates.com
%
% License:
% CC BY-NC-SA 3.0 (http://creativecommons.org/licenses/by-nc-sa/3.0/)
%
%%%%%%%%%%%%%%%%%%%%%%%%%%%%%%%%%%%%%%%%%

%------------------------------------------------------------------
%	PACKAGES AND OTHER DOCUMENT CONFIGURATIONS
%------------------------------------------------------------------

\documentclass[final]{beamer}

\usepackage{pwintz_configuration}
\usepackage{pwintz_definitions}
% Insert definitions used in your document here. 

\graphicspath{{images/}{../images/}}

\newcommand{\mysubsection}[1]{\vspace{20pt plus 5pt}{\color{structure} \textbf{#1}}\par}

\usepackage[scale=0.95]{beamerposter} % Use the beamerposter package for laying out the poster

\usepackage{paracol}

\usetheme{confposter} % Use the confposter theme supplied with this template

%-----------------------------------------------------------
% Define the column widths and overall poster size
% To set effective colspaceinner, onecolwid and twocolwid values, first choose how many columns you want and how much separation you want between columns
% In this template, the separation width chosen is 0.024 of the paper width and a 4-column layout
% onecolwid should therefore be (1-(# of columns+1)*colspaceinner)/# of columns e.g. (1-(4+1)*0.024)/4 = 0.22
% Set twocolwid to be (2*onecolwid)+colspaceinner = 0.464
% Set threecolwid to be (3*onecolwid)+2*colspaceinner = 0.708

\usepackage{calc}% http://ctan.org/pkg/calc
\newlength{\colspace}
\newlength{\colspaceinner}
\newlength{\colspaceouter}
\newlength{\onecolwid}
\newlength{\twocolwid}
\newlength{\threecolwid}
\setlength{\paperwidth}{48in} % A0 width: 46.8in
\setlength{\paperheight}{36in} % A0 height: 33.1in
% The normative space between columns/the edge of the paper.
\setlength{\colspace}{0.012\paperwidth}
 % Space between columns. This is manually adjusted to make the margins equal (there is a way to do this automatically, but I'm out of time.)
\setlength{\colspaceouter}{1.25\colspace}
 % Space between columns and edge of page (margins)
\setlength{\colspaceinner}{0.666667\colspace}
 % Width of one column=(1-5*0.012)/4
\setlength{\onecolwid}{(\paperwidth-5\colspace)/4}
 % Width of two columns=(2*onecolwid)+colspaceinner=(2*0.235)+0.012
% \setlength{\twocolwid}{0.482\paperwidth}
% \setlength{\threecolwid}{0.708\paperwidth} % Width of three columns
\setlength{\topmargin}{-0.5in} % Reduce the top margin size
%-----------------------------------------------------------

\usepackage{graphicx}  % Required for including images

\usepackage{booktabs} % Top and bottom rules for tables

%------------------------------------------------------------------
%	TITLE SECTION 
%------------------------------------------------------------------

\makeatletter
\title{\pwintz@title}% Poster title is defined in document_setup.tex
\makeatother
\author{% Populate author information from defnitions in document_setup.tex
   % Insert first author, if defined.
   \ifcsdef{pwintz@author1}{% If defined
        \csname pwintz@author1\endcsname%
        \textsuperscript{\csname pwintz@authorInstitute1\endcsname}%        
        \ifcsdef{pwintz@authorEmail1}{ (\csname pwintz@authorEmail1\endcsname)}{}%
    }{}%
    % Insert second author, if defined.
    \ifcsdef{pwintz@author2}{, \ifcsdef{pwintz@author3}{}{and }}{}% Add "," or ", and"
    \ifcsdef{pwintz@author2}{% If defined
        \csname pwintz@author2\endcsname%
        \textsuperscript{\csname pwintz@authorInstitute2\endcsname}%
    }{}%
    % Insert third author, if defined.
    \ifcsdef{pwintz@author3}{, \ifcsdef{pwintz@author4}{}{and }}{}% Add "," or ", and"
    \ifcsdef{pwintz@author3}{% If defined
        \csname pwintz@author3\endcsname%
        \textsuperscript{\csname pwintz@authorInstitute3\endcsname}%
    }{}%
    % Insert fourth author, if defined.
    \ifcsdef{pwintz@author4}{, \ifcsdef{pwintz@author5}{}{and }}{}% Add "," or ", and"
    \ifcsdef{pwintz@author4}{% If defined
        \csname pwintz@author4\endcsname%
        \textsuperscript{\csname pwintz@authorInstitute4\endcsname}%
    }{}
}

\institute{ % Populate institutions that are defined in document_setup.tex
    \ifcsdef{pwintz@institute1}{% If defined
        \textsuperscript{1}\csname pwintz@institute1\endcsname%
    }{}% Else nothing
    \ifcsdef{pwintz@institute2}{% If defined
       ; \textsuperscript{2}\csname pwintz@institute2\endcsname%
    }{}% Else nothing
    \ifcsdef{pwintz@institute3}{% If defined
       ; \textsuperscript{3}\csname pwintz@institute3\endcsname%
    }{}% Else nothing
    \ifcsdef{pwintz@institute3}{; }{}% Add "; " if there is another institute
    \ifcsdef{pwintz@institute4}{% If defined
       ; \textsuperscript{4}\csname pwintz@institute3\endcsname%
    }{}% Else nothing
} 

%------------------------------------------------------------------

\setlength{\parskip}{40.0 pt plus 5 pt}

\setbeamercolor{bibliography entry note}{fg=structure!85!white}

\usepackage{demo_setup}

\begin{document}

\addtobeamertemplate{block end}{}{\vspace*{2ex}} % White space under blocks
\addtobeamertemplate{block alerted end}{}{\vspace*{2ex}} % White space under highlighted (alert) blocks

\setlength{\belowcaptionskip}{2ex} % White space under figures
\setlength\belowdisplayshortskip{2ex} % White space under equations

\begin{frame}[t] % The whole poster is enclosed in one beamer frame

\begin{columns}[t] % The whole poster consists of three major columns, the second of which is split into two columns twice - the [t] option aligns each column's content to the top

\hspace{\colspaceouter} % Left Margin
\begin{column}{\onecolwid} % The first column

%------------------------------------------------------------------
%	OBJECTIVES
%------------------------------------------------------------------

\begin{block}{Summary}
    \lipsum[1]
\end{block}

%------------------------------------------------------------------
%	INTRODUCTION
%------------------------------------------------------------------

\begin{block}{Preliminaries}
    \lipsum[2]
\end{block}
% \vspace{-10pt} % Adjust space as needed.

\begin{block}{Problem Statement}
    \lipsum[2][1-2]
    $$ \ddt[f]=\lim_{h \to 0} \frac{f(t+h)-f(t)}{h} $$
    \lipsum[3-5]
\end{block}
% \vspace{-10pt} % Adjust space as needed.

\end{column} % End of the first column
\hspace{\colspaceinner}
\begin{column}{\onecolwid}% Beginning of second column

\begin{block}{[Descriptions of Solution]}
    \lipsum[6-8]
    \begin{figure}[ht]
        \centering
        \includegraphics[width=0.6\textwidth,height=0.4\textwidth]{example-image-a}
        % \setlength{\belowcaptionskip}{-100pt} % Adjust space below image.
    \end{figure}
    \lipsum[9][1-3]
\end{block}

\begin{alertblock}{First Fundamental Theorem of Calculus}
    Let \(f\) be a continuous real-valued function defined on a closed interval \([a, b]\). Let \(F\) be the function defined, for all \(x\) in \([a, b]\), by
    \[
    F(x)=\int_a^x f(t) d t
    \]
    Then \(F\) is uniformly continuous on \([a, b]\) and differentiable on the open interval \((a, b)\), and
    \[
    F^{\prime}(x)=f(x)
    \]
    for all \(x\) in \((a, b).\)
\end{alertblock}

\textit{Remark.} \lipsum[2][4]

\end{column} % End of second column
\hspace{\colspaceinner}
\begin{column}{\onecolwid} % Beginning of thrid column

\begin{block}{Example: [first example]}
    \lipsum[10]
    \begin{figure}
        \centering
        \includegraphics[width=0.7\linewidth]{example-image-b}
    \end{figure}
    \lipsum[11-14]
\end{block}

\end{column} % End of the third column
\hspace{\colspaceinner}
\begin{column}{\onecolwid} % The fourth column

\begin{block}{Example: [second example]}
    \lipsum[15]
    \begin{align*}
        \minimize{u} & \int_0^\infty \norm{z(t)}^2 + \norm{u(t)}^2 \dt \quad & 
        \subjectto & \zdot = Az + u
    \end{align*}
    \lipsum[16]
    \begin{figure}[h]
        \centering
        \includegraphics[width=0.4\linewidth]{example-image-c}
    \end{figure}
\end{block}

\begin{block}{References}
    % Make the text size smaller for the references.
    \renewcommand*{\bibfont}{\small} 

    % TODO: Replace the following citations with your own.
    \nocite{knuth:ct:a}
    \nocite{glashow}
    \nocite{aristotle:physics}
    \nocite{moore}
    \nocite{salam}
    \nocite{ctan}
    \nocite{loh}
    \printbibliography
\end{block}

%------------------------------------------------------------------
%	ACKNOWLEDGEMENTS
%------------------------------------------------------------------

\begin{block}{Acknowledgements}
    % Use small Roman text for the acknowledgments.
    \small{\rmfamily{\acknowledgementBlurb}} 
        \vspace{20pt}
        
        % Logos for funding agencies.
        \begin{minipage}{\linewidth}
            \includegraphics[height=0.16\linewidth,align=c]{example-image-a}\hfill
            \includegraphics[height=0.16\linewidth,align=c]{example-image-b}\hfill
            \includegraphics[height=0.16\linewidth,align=c]{example-image-c} 
        \end{minipage}
    \end{block}

%------------------------------------------------------------------

\end{column} % End of the fourth column
\end{columns} % End of all the columns in the poster

%% Draw an overlay with tikz to aid in getting the margins equal. 
% \begin{tikzpicture}[remember picture, overlay,
%     help lines/.append style={line width=0.05pt}]
%     \draw (current page.west) -- +(\colspaceouter, 0);
%     \draw (current page.east) -- +(-\colspaceouter, 0);
% \end{tikzpicture}%

\end{frame} % End of the enclosing frame

\end{document}
