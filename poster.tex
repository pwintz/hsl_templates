%%%%%%%%%%%%%%%%%%%%%%%%%%%%%%%%%%%%%%%%%
% Jacobs Landscape Poster
% LaTeX Template
% Version 1.1 (14/06/14)
%
% Created by:
% Computational Physics and Biophysics Group, Jacobs University
% https://teamwork.jacobs-university.de:8443/confluence/display/CoPandBiG/LaTeX+Poster
% 
% Further modified by:
% Nathaniel Johnston (nathaniel@njohnston.ca)
%
% This template has been downloaded from:
% http://www.LaTeXTemplates.com
%
% License:
% CC BY-NC-SA 3.0 (http://creativecommons.org/licenses/by-nc-sa/3.0/)
%
%%%%%%%%%%%%%%%%%%%%%%%%%%%%%%%%%%%%%%%%%

%------------------------------------------------------------------
%	PACKAGES AND OTHER DOCUMENT CONFIGURATIONS
%------------------------------------------------------------------

\documentclass[final]{beamer}

% LTeX: enable=true
% Create \ifdraft{}{} conditional that switches based on whether "draft" is passed to document class.
\usepackage{ifdraft} 
\usepackage[utf8]{inputenc}
\usepackage{amsmath}
\usepackage{amssymb}
\usepackage{amsfonts}
\usepackage{mathrsfs} % Enables \mathscr{}

% ==============
% === Layout ===
% ==============
\usepackage{multicol} % Allows enumerations over two columns
\usepackage{float}
% \usepackage[bottom]{footmisc} % Forces footnotes to be at the very bottom of the page. https://tex.stackexchange.com/a/9427/153678

% ==================
% === Formatting ===
% ==================
\usepackage{tcolorbox}
\usepackage{xcolor}
\usepackage{color}
\usepackage{verbatim} 
\usepackage{markdown}
\usepackage{bm} % Allows bold Greek letters.
\usepackage{xspace} % To fix spacing after macros. see https://tex.stackexchange.com/a/17731/153678
% allows enumerate with letters, i.e. (a), (b), (c).
%\usepackage[shortlabels]{enumitem} 

% ===============
% === SPACING ===
% ===============

% Format paragraphs. 
\makeatletter%
\@ifclassloaded{article}{%
    \setlength{\parindent}{0em}
    \setlength{\parskip}{0.8em}
}

%%%% Fix Spacing Around "\left(\right)" %%%
\usepackage{mleftright} 
\mleftright % redefine \left as \mleft and \right as \mright.

% Set the line spacing to single-spaced in the "cases" environment. 
% Source: https://tex.stackexchange.com/a/555622/153678
\usepackage{etoolbox}
\AtBeginEnvironment{cases}{\renewcommand\baselinestretch{1}\selectfont}

% Command to adjust the horizontal spacing 
%   By Ryan Johnson. 
% 
% Example usage:
% \begin{equation}
% \mbox{$\squeezespaces{0.5}
% \begin{aligned}
%     a^2 + b^2 = c^2
% \end{aligned}
% $}
% \end{equation}
\newcommand*\squeezespaces[1]{% <- #1 is a number between 0 and 1
  \thickmuskip=\scalemuskip{\thickmuskip}{#1}%
  \medmuskip=\scalemuskip{\medmuskip}{#1}%
  \thinmuskip=\scalemuskip{\thinmuskip}{#1}%
  \nulldelimiterspace=#1\nulldelimiterspace
  \scriptspace=#1\scriptspace
}
\newcommand*\scalemuskip[2]{%
  \muexpr #1*\numexpr\dimexpr#2pt\relax\relax/65536\relax
}
% === END OF SPACING ===

% Support adding "todo" comments.
\usepackage[colorinlistoftodos,prependcaption,textsize=small,textcolor=white]{todonotes}
\setlength{\marginparwidth}{2cm} % This doesn't appear to do anything, except suppress a warning related to todonotes.

%% Improved LaTeX definitions %%
\usepackage{environ}
\usepackage{xparse} % Enhanced command definitions.

%% Misc/Uncategorized Packages %%
\usepackage{comment} % enables the use of multi-line comments (\ifx \fi) 
\usepackage{cancel} % For crossing out parts of equations.
\usepackage{empheq} % For putting boxes around equations
\usepackage{mathtools}
\usepackage{pdfpages} % Allows use of \includepdf[page={page number}]{filename}
% \usepackage{siunitx} % SI Units used as '\SI{60}{\mile\per\hour} https://tex.stackexchange.com/a/509965/153678

% =========================
% === CONDITIONAL LOGIC ===
% =========================
\usepackage{etoolbox} % Used for LaTeX conditionals. See: https://tex.stackexchange.com/a/5896/153678.
\newtoggle{solutions}

% Define 'switch' environment. 
% See https://tex.stackexchange.com/a/187710/153678
\usepackage{xifthen}
\newcommand{\ifequals}[3]{\ifthenelse{\equal{#1}{#2}}{#3}{}}
\newcommand{\case}[2]{#1 #2} % Dummy, so \renewcommand has something to overwrite...
\newenvironment{switch}[1]{\renewcommand{\case}{\ifequals{#1}}}{}

% ============================
% === Figures and Graphics ===
% ============================
\usepackage{pdfpages} % Allows use of \includepdf[page={page number}]{filename}
\usepackage{rotating}% For rotating figures, tables, etc. See https://tex.stackexchange.com/a/50071/153678
\usepackage{blkarray} % allows labeling rows/columns of block matrix. 
% Setup captions
% \usepackage{caption,setspace} % needed to set caption line spacing. See: https://tex.stackexchange.com/a/186327/153678
\usepackage{caption}
% Allows removal of "Figure 1:" in caption when "\caption*{}" is used. 
% \captionsetup{font={stretch=1.2}}
\captionsetup{belowskip=-4pt}

%%%% Support subfigures %%%%
% The package 'subfigure' is deprecated. Now, either 'subfig' or 'subcaption'
% is recommend.
\usepackage{graphicx} 
% \usepackage{caption}
% \usepackage{subfig}
%% Example of how to use 'subfig' package:
% \begin{figure}
% \centering
% \begin{subfigure}{.5\textwidth}
%   \centering
%   \includegraphics[width=.6\linewidth]{}
% \end{subfigure}%
% \begin{subfigure}{.5\textwidth}
%   \centering
%   \includegraphics[width=.6\linewidth]{}
% \end{subfigure}
% \end{figure}

\usepackage{subcaption}
%% Example of how to use 'subcaption' package:
% \begin{figure*} % Star makes multi-column figure. 
%     \centering
%     \subcaptionbox{Subfigure (a) caption.\label{subfig a label}}
%     {\includegraphics{image_1}}
%     \qquad
%     \subcaptionbox{Subfigure (b) caption.\label{subfig b label}}
%     {\includegraphics{image_2}}
%     \caption{Figure caption.}
%     \label{figure label}
% \end{figure*}

% ===================
% === CODE BLOCKS ===
% ===================

\usepackage[final]{listings}

% Allows comments sections
\usepackage{verbatim} 

% \lstset{frame=tb,
%   language=Python,
%   aboveskip=3mm,
%   belowskip=3mm,
%   showstringspaces=false,
%   columns=flexible,
%   basicstyle={\small\ttfamily},
%   numbers=none,
%   numberstyle=\tiny\color{gray},
%   keywordstyle=\color{blue},
%   commentstyle=\color{dkgreen},
%   stringstyle=\color{mauve},
%   breaklines=true,
%   breakatwhitespace=true,
%   tabsize=3
% }

% ==============================
% === REFERENCES AND LINKING ===
% ==============================
% \let\inserttitle\@title
% \let\inserttitle\@title

\makeatletter
% Enable hyperlinks
\usepackage{hyperref}[implicit=true]
\hypersetup{
    colorlinks=false,
    linkcolor=blue,
    filecolor=magenta,      
    urlcolor=blue,     
    anchorcolor = blue,
    citecolor = blue,    
    % Set the metadata for the produced PDF.
    pdftitle={UGAS while Exploiting Uncertified Controller},
    pdfsubject={},
    pdfauthor={Paul K. Wintz, Ricardo G. Sanfelice, João P. Hespanha},
    pdfkeywords={Hybrid Dynamical Systems, Control Systems, Switching, Lyapunov Functions, Model Predictive Control}
}   
\makeatother

% Display Theorem/Lemma/Corollary (1) in cross references.  
% IT'S IMPORTANT TO LOAD THIS LAST AFTER ALL THE OTHER \usepackages.
\usepackage[noabbrev, capitalise]{cleveref} 
% Make ranges of references use an dash between the equation numbers. 
% See https://tex.stackexchange.com/a/18988/153678
\newcommand{\crefrangeconjunction}{--} 

% Configure Clever Reference
% Make equations referenced as "(1)" instead of "Equation 1"
\crefname{equation}{}{}
\crefformat{equation}{(#2#1#3)}
% \crefrangeformat{equation}{(#3#1#4) to~(#5#2#6)}
\crefmultiformat{equation}{(#2#1#3)}%
{ and~(#2#1#3)}{, (#2#1#3)}{ and~(#2#1#3)}

\newcommand{\setupAssumption}[1][A]{
    % Setup enumerations with the assumption environment to use enumerations 
    % (A1), (A2), etc. If the optional argument is provided, 
    % then the letter is modified.
    \renewcommand*{\theenumi}{(#1\arabic{enumi})}% From https://tex.stackexchange.com/a/11901/153678
    \renewcommand*{\labelenumi}{(#1\arabic{enumi})}%
}


% =================
% === REFERENCES ===
% =================

% Setup references for presentations

\makeatletter%
\@ifclassloaded{beamer}%
{% if using beamer
    % Import the natbib package and sets a bibliography and citation styles
    \usepackage[url = false,doi = false,isbn = false,natbib]{biblatex}
    % \usepackage[sorting=none,url=false, doi=false, isbn=false]{biblatex}
    \addbibresource{biblio.bib} %Import the bibliography file
    % \setcitestyle{authoryear,open={((},close={))}}
}
\makeatother%

% \usepackage[style=authortitle]{biblatex}

% Import the natbib package and sets a bibliography and citation styles
% \usepackage[url = false,doi = false, isbn = false]{biblatex}
% \usepackage{natbib}
% \bibliographystyle{abbrvnat}
% \setcitestyle{authoryear,open={((},close={))}}

% \addbibresource{Uniting Feedback Controller.bib} %Imports bibliography file
% \renewbibmacro*{cite:author}{%
%     \printtext[bibhyperref]{%
%         \structure{hello}
%         % \printfield[citeauthor]{labeltitle}%
%         % \setunit{\space}%
%         % \printtext[parens]{\printdate}%
%     }%
% }
% \renewbibmacro*{cite:title}{%
%     \printtext[bibhyperref]{%
%         \printfield[citetitle]{labeltitle}%
%         \setunit{\space}%
%         \printtext[parens]{\printdate}%
%     }%
% }

% =========================
% === EDITORIAL MARKUPS ===
% =========================

\ifdraft{
    \newenvironment{workingnotes}{\sffamily \color{violet}}{}
}{ 
    \newenvironment{workingnotes}
    {\expandafter\comment}
    {\expandafter\endcomment}
}

\ifdraft{
    \newcommand{\workingnote}[1]{\begin{workingnotes} #1 \end{workingnotes}}
}{ 
    \newcommand{\workingnote}[1]{}
}

\ifdraft{
    \newcommand {\finalonly}[1]{{\color{red}$<$Hidden$>$}}
}{
    \newcommand{\finalonly}[1]{#1}
}

% Progress tracking
\ifdraft{
    \newcommand{\done}{\ensuremath{\checkmark}} % Variant: \surd
    \newcommand{\undone}{\ensuremath{\square}}
    \newcommand{\doneish}{\done-ish}
}{
    \newcommand{\done}{}
    \newcommand{\undone}{}
    \newcommand{\doneish}{}
}

% Support tracking changes and adding "todo" comments.
% \usepackage[todonotes={colorinlistoftodos,prependcaption,textsize=small,textcolor=white},commandnameprefix=ifneeded]{changes}
\usepackage[commandnameprefix=ifneeded,
            final,
            todonotes={colorinlistoftodos,prependcaption,
                       textsize=small,backgroundcolor=orange!10,
                       textcolor=black,linecolor=orange,bordercolor=orange}]{changes}

% ============================================
% === ENVIRONMENTS (Theorems, lemmas, etc) ===
% ====================================f========

\makeatletter%
\@ifclassloaded{beamer}{% 
    \theoremstyle{definition}%
    \ifdefined\definition\else
        \newtheorem{definition}{Definition}
    \fi
    \ifdefined\problem\else
        \newtheorem{problem}{Problem}
    \fi
    \newtheorem{assumption}{Assumption}% Use singular even when there are multiple assumptions in a particular block.
    \crefname{assumption}{Assumption}{Assumptions}
    \theoremstyle{remark}%
    \ifdefined\remark\else
        \newtheorem{remark}{Remark}
    \fi
    \ifdefined\example\else
        \newtheorem{example}{Example}
    \fi
    % \newtheorem{example}{Example}%
    \theoremstyle{plain}%
    \ifdefined\theorem\else
        \newtheorem{theorem}{Theorem}
    \fi
    \newtheorem{proposition}{Proposition}
    \ifdefined\lemma\else
        \newtheorem{lemma}{Lemma}
        \crefname{lemma}{Lemma}{Lemmas}%
    \fi
    \ifdefined\corollary\else
        \newtheorem{corollary}{Corollary}
        \crefname{corollary}{corollary}{corollaries}%
    \fi
}
\makeatother%

\makeatletter%
\@ifclassloaded{ieeeconf}{% if using ieeeconf
    \IEEEoverridecommandlockouts
    \overrideIEEEmargins

    % Load amsthm package. The following "\relax"
    % commands prevent an error that say
    % "Command \proof already defined." when we load amsthm.
    \let\proof\relax 
    \let\endproof\relax
    \usepackage{amsthm}
    
    \theoremstyle{definition}%
    \newtheorem{definition}{Definition}
    \newtheorem{assumption}{Assumption}
    \crefname{assumption}{Assumption}{Assumptions}
    \newtheorem{example}{Example}

    \theoremstyle{plain}%
    \newtheorem{theorem}{Theorem}
    \newtheorem{proposition}{Proposition}
    \newtheorem{lemma}{Lemma}

    \theoremstyle{remark}
    \newtheorem{remark}{Remark}

    % Set the bibliography style.
    \bibliographystyle{ieeetr}
}{% if not using ieeeconf
    \newcommand{\authorblockN}[1]{#1}
    % The amsthm package provides {theorem}, {lemma} and {proof} environments
    \usepackage{amsthm}
    \usepackage[toc,page]{appendix}
}
\makeatother%

\makeatletter%
\@ifclassloaded{article}{%
    \usepackage{amsthm}
    \theoremstyle{definition}
    \newtheorem{definition}{Definition}
    \crefname{definition}{Definition}{Definitions}
    \newtheorem{assumption}{Assumption}
    \crefname{assumption}{Assumption}{Assumptions}
    \theoremstyle{remark}
    \newtheorem{remark}{Remark}
    \newtheorem{example}{Example}
    \theoremstyle{plain}
    \newtheorem{theorem}{Theorem}
    \newtheorem{proposition}{Proposition}
    \newtheorem{lemma}{Lemma}
    \crefname{lemma}{Lemma}{Lemmas}
    \newtheorem{corollary}{Corollary}
    \crefname{corollary}{corollary}{corollaries}
    \newtheorem{conjecture}{Conjecture}
    \crefname{conjecture}{conjecture}{conjectures}
}
\makeatother%

% =================================
% === BEAMER PRESENTATION STYLE ===
% =================================
\makeatletter%
\@ifclassloaded{beamer}{% 
    % Hide Roman numerals from continued frames with breaks
    \setbeamertemplate{frametitle continuation}{}

    % Define Beamer colors.
    \definecolor{lightblue}{RGB}{60,60,200}
    \setbeamercolor{block title}{bg=lightblue!10!white}
    \setbeamercolor{block body}{bg=lightblue!5!white}
    \setbeamercolor{block title alerted}{bg=blue!10!white}
    \setbeamercolor{block body alerted}{bg=blue!3!white}
    \setbeamerfont{block title}{size={}}
    \setbeamertemplate{bibliography item}{\insertbiblabel}
    % Select font
    \usefonttheme{serif}
    \setbeamerfont{footnote}{size=\tiny}
    \setbeamercolor{page number in head/foot}{fg=lightblue}
    \setbeamercolor{author in head/foot}{fg=lightblue}
    \setbeamertemplate{navigation symbols}{}
    \setbeamertemplate{theorems}[numbered]
    \setbeamertemplate{frametitle}[default][center]
    % ======== Table of Contents ========
    % Set the TOC colors
    \setbeamercolor{section in toc}{fg=structure}
    \setbeamercolor{subsection in toc}{fg=structure}
    % Define Table of Contents slide at the beginning of each section.
    \AtBeginSection[]{%
    \begin{frame}<beamer>
        \frametitle{Outline}
        % Use 'show'/'hide' to show/hide subsections, and 'shaded' to gray them out. 
        \tableofcontents[currentsection,
                         currentsubsection,
                         subsectionstyle=show/show/shaded]
    \end{frame}
    }%
    % 
    % Set the size of page numbers.
    \setbeamerfont{footline}{size=\small}
    \setbeamertemplate{footline}{
        \hfill%
        \usebeamercolor[fg]{page number in head/foot}%
        \usebeamerfont{page number in head/foot}%
        \setbeamertemplate{page number in head/foot}[framenumber]%
        \usebeamertemplate*{page number in head/foot}\kern1em\vskip6pt%
    }
}
\makeatother%

% ======================================
% === ENVIRONMENTS FOR LECTURE NOTES ===
% ======================================
% The amsthm package provides {theorem}, {lemma} and {proof} environments
% \newenvironment{problem}[1][]
%     {\medskip\large \textbf{Problem \!(#1)}:\normalsize \itshape }
%     {\par}
\newenvironment{subproblem}[1][]
    {\normalsize \textbf{Part \!(#1)}:\normalsize \itshape }
    {\par}    
\newenvironment{hint}{\ttfamily\textup{[}}{\textup{]}}

% Don't print section numbers
% \setcounter{secnumdepth}{0}

% Spaced example
\newenvironment{spexample}[1][]{
\par
\vbox\bgroup
\begin{example}[#1]
}{
\end{example}\egroup
}

% Spaced exercise
\newenvironment{spexercise}[1][]{
\par
\vbox\bgroup\begin{exercise}[#1]
}{
\end{exercise}\egroup%
}

% Spaced definition
\newenvironment{spdefinition}[1][]{
\par
\vbox\bgroup
\begin{definition}[#1]
}{
\end{definition}\egroup
}

% Spaced theorem
\newenvironment{sptheorem}[1][]{
\par
\vbox\bgroup
\begin{theorem}[#1]
}{
\end{theorem}\egroup
}

% Spaced remark
\newenvironment{spremark}[1][]{
\par
\vbox\bgroup
\begin{remark}[#1]
}{
\end{remark}\egroup
}

% Spaced reminder
\newenvironment{spreminder}[1][]{
\par
\vbox\bgroup
\begin{reminder}[#1]
}{
\end{reminder}\egroup
}

% Define document colors
\definecolor{blue}{rgb}{0.169, 0.243, 0.714}
\definecolor{orange}{rgb}{1, 0.652, 0}
\definecolor{q0}{rgb}{0.169, 0.243, 0.714}
\definecolor{q1}{rgb}{1, 0.652, 0}
\newcommand{\qcolor}[2]{{\color{q#1}#2}}

\usepackage{bbm}
\usepackage{xspace}

% =====================
% ======= TEXT ========
% =====================
%
\newcommand{\textfor}{\textup{for }}%
\newcommand{\textif}{\textup{if }}%
\newcommand{\textotherwise}{\textup{otherwise\xspace}}%
\newcommand{\otherwise}{\textotherwise}%
\newcommand{\midand}{\quad \textup{and}\quad}%
\newcommand{\midor}{\quad \textup{or}\quad}%
% \renewcommand {\if}{&\quad \text{if }}%
\newcommand{\st}{\ensuremath{^{st}}}%
\newcommand{\nd}{\ensuremath{^{nd}}}
\newcommand{\rd}{\ensuremath{^{rd}}}
\renewcommand{\th}{\ensuremath{^{th}}} % for ith, jth, etc.
\newcommand*{\circled}[1]{\raisebox{.5pt}{\textcircled{\raisebox{-.9pt} {#1}}}} % Creates a number enclosed in a circle
\newcommand{\lhs}{left-hand side }
\newcommand{\rhs}{right-hand side }
\newcommand{\wrt}{with respect to }
\newcommand{\Matlab}{\textsc{Matlab}\xspace}
%
% =============================
% ======= GENERAL MATH ======== 
% =============================
%
\newcommand*{\invs}{^{-1}}
\NewDocumentCommand{\hypt}{s m m}{
	% \hypt{a}{b} generates sqrt(a^2 + b^2). 
	% \hypt*{a}{b} puts parentheses around a and b.
    \IfBooleanTF #1%
        % Star=Parentheses.
        {\sqrt{\left(#2\right)^2 + \left(#3\right)^2}}%
        % No star=No Parentheses.
        {\sqrt{#2^2 + #3^2}}%
   }
\newcommand{\definedas}{\equiv}
\newcommand{\identifyas}{\equiv}
\newcommand{\iso}{\cong}
\newcommand{\evaluate}{\bigg\rvert}
\newcommand{\at}{\bigg\rvert}
\newcommand*{\evaluateat}[1]{\bigg\rvert_{#1}}
\newcommand*{\evaluatefromto}[2]{\bigg\rvert_{#1}^{#2}}
\newcommand{\dom}{\operatorname{dom}}
% 
% Create a system of equations.
% Usage:
% \system{x + y &= 2 \\ x - y &= 0}
\newcommand{\system}[1]{\left\{
    \begin{aligned}
        #1
    \end{aligned}\right.}
\newcommand{\neginfty}{{-\infty}}
%
% **** FUNCTIONS ****
\renewcommand{\arctan}{\tan\invs}
\newcommand{\atan}{\arctan}
\newcommand{\atantwo}{\mathtt{atan2}}
\renewcommand{\arcsin}{\sin\invs}
\newcommand{\asin}{\arcsin}
\renewcommand{\arccos}{\cos\invs}
\newcommand{\acos}{\arccos}
\newcommand{\abs}[1]{\left|#1\right|}
\newcommand{\absq}[1]{\abs{#1}^2}
%
% **** COMPLEX **** 
\newcommand{\real}{\,\mathfrak{Re}}
\newcommand{\imag}{\,\mathfrak{Im}}
\newcommand{\conj}[1]{\overline{#1}}
%
% ==========================
% ======= Fractions ======== 
% ==========================
%
\newcommand{\fracshort}[2]{\left.#1 \:\middle/\: #2\right.}
\newcommand{\oneover}[1]{\frac{1}{#1}}
\newcommand{\doneover}[1]{\dfrac{1}{#1}}
\newcommand{\oneovershort}[1]{\fracshort{1}{#1}}
\newcommand{\half}[1][1]{\frac{#1}{2}}
\newcommand{\halfshort}[1][1]{\fracshort{#1}{2}}
\newcommand{\third}[1][1]{\frac{#1}{3}}
\newcommand{\thirdshort}[1][1]{\fracshort{#1}{3}}
\newcommand{\quarter}[1][1]{\frac{#1}{4}}
\newcommand{\quartershort}[1][1]{\fracshort{#1}{4}}
\newcommand{\fifth}[1][1]{\frac{#1}{5}}
\newcommand{\fifthshort}[1][1]{\fracshort{#1}{5}}
\newcommand{\sixth}[1][1]{\frac{#1}{6}}
\newcommand{\sixthshort}[1][1]{\fracshort{#1}{6}}
\newcommand{\twelfth}[1][1]{\frac{#1}{12}}
\newcommand{\twelfthshort}[1][1]{\fracshort{#1}{12}}
%
% ===========================
% ======= SET THEORY ======== 
% ===========================
%
% Set name
\newcommand{\setname}{\mathcal} 
% Define a set in the form {<arg 1>|<arg 2>}
% \newcommand{\suchthat}{\mathrel{} \middle| \mathrel{} } % Alternative: \mid
\newcommand{\suchthat}{\mathrel{}\ifnum\currentgrouptype=16 \middle\fi|\mathrel{}}
\newcommand*{\setdef}[1]{\left\{#1 \right\}} 
\newcommand*{\closure}[1]{\overline{#1}}
\newcommand{\interior}{\operatorname{int}}
\newcommand{\union}{\cup}
\newcommand{\Union}{\bigcup}
\newcommand{\inter}{\cap}
\newcommand{\Inter}{\bigcap}
\newcommand{\intersect}{\inter}
\newcommand{\Intersect}{\Inter}
\newcommand{\boundary}{\partial}
\newcommand{\bnd}{\boundary}
%
% **** SETS **** 
\newcommand{\reals}{\mathbb{R}}
\newcommand{\realsn}{\reals^{n}}
\newcommand{\positivereals}{\reals_{>0}}
\newcommand{\preals}{\positivereals}
\newcommand{\nonnegativereals}{\reals_{\geq0}}
\newcommand{\nnreals}{\nonnegativereals}
\newcommand{\rationals}{\mathbb{Q}}
\newcommand{\integers}{\mathbb{Z}}
\newcommand{\naturals}{\mathbb{N}}
\newcommand{\complexes}{\mathbb{C}}
\newcommand*{\squarematrices}[1][n]{\reals^{#1 \times #1}}
\newcommand*{\symmetricmatrices}[1][n]{\mathbb{S}^{#1}}
\newcommand*{\pdmatrices}[1][n]{\symmetricmatrices[#1]_{++}}
\newcommand*{\psdmatrices}[1][n]{\symmetricmatrices[#1]_{+}}
\newcommand*{\unitball}[1][n]{\mathbb{B}^{#1}}
\newcommand*{\sphere}[1][n]{\mathbb{S}^{#1}}
\newcommand*{\torus}[1][n]{\mathbb{T}^{#1}}
\newcommand*{\realprojective}[1][n]{\reals \mathbb{P}^{#1}}
\newcommand*{\complexprojective}[1][n]{\complexes\mathbb{P}^{#1}}
%
% =============================
% ====== LINEAR ALGEBRA =======
% ============================= 
%
% **** VECTORS **** 
% If a value is not a vector don't use the "\"
\renewcommand*{\vec}[1]{\bm{\mathrm{#1}}} % Vector variable
\newcommand{\x}{{\vec{x}}}
\newcommand{\y}{{\vec{y}}} 
\newcommand{\z}{{\vec{z}}}
\renewcommand{\r}{{\vec{r}}}
\renewcommand{\u}{{\vec{u}}}
\renewcommand{\v}{{\vec{v}}}
\renewcommand{\a}{{\vec{a}}}
\renewcommand{\b}{{\vec{b}}}
\newcommand{\0}{\vec 0}
\newcommand{\onevec}{\mathbb{1}}
\newcommand{\RRF}{\operatorname{RRF}}
\newcommand{\RREF}{\operatorname{RREF}}
\newcommand*{\proj}[1]{\operatorname{proj}_{#1}}
\newcommand*{\comp}[1]{\operatorname{comp}_{#1}}
\newcommand*{\orth}[1]{\operatorname{orth}_{#1}}
%
% **** VECTOR AND MATRIX FUNCTIONS **** 
\newcommand*{\innerproduct}[2]{\left\langle #1, #2 \right\rangle}
\newcommand*{\ip}[2]{\innerproduct{#1}{#2}}
\newcommand*{\norm}[1]{\left\lvert#1\right\rvert} % double bars "\lVert ... \rVert" are used for signal norms so we use singal bars for euclidean vector norms.
\newcommand*{\normsq}[1]{\norm{#1}_{1}^2}
\newcommand*{\onenorm}[1]{\norm{#1}_{1}}
\newcommand*{\twonorm}[1]{\norm{#1}_{2}}
\newcommand*{\pnorm}[1]{\norm{#1}_{p}}
\newcommand*{\frobeniusnorm}[1]{\norm{#1}_{F}}
\newcommand*{\inftynorm}[1]{\norm{#1}_{\infty}}
\newcommand{\diag}{\operatorname{diag}}
\newcommand{\rank}{\operatorname{rank}}
\newcommand{\trace}{\operatorname{tr}}
%DON'T USE: \renewcommand {\null}{\text{null}}
\newcommand{\range}{\operatorname{range}}
\newcommand{\trans}[1][]{^{#1 \top}} % Matrix transpose
\newcommand{\transstar}{^{*\top}} % Vector or matrix transpose notation (\intercal is an alternative)
\newcommand{\herm}[1][]{^{#1 H}} % Hermitian 
\newcommand{\eig}{\operatorname{eig}}
%
% **** MATRICES **** 
% Create a large matrix. Usage \mat[<vertical spacing>]{a & b \\ c & d}
\newcommand{\mat}[2][1]{\begingroup
	\renewcommand*{\arraystretch}{#1}
	\begin{bmatrix}#2\end{bmatrix}
\endgroup}
% Create a small matrix suitable to use in the middle of text. 
% Usage \smallmat{a & b \\ c & d}
\newcommand*{\smallmat}[1]{\bigl[ \begin{smallmatrix} #1 \end{smallmatrix} \bigr]}
\newcommand{\diagmatrix}[2]{
\mat{
{#1} & & \\
& \ddots & \\
& & {#2}
}}
\newcommand{\tridiagmatrix}[3]{
\mat{
#2     & #3     &         \\
#1     & \ddots & \ddots  \\
       & \ddots & \ddots & #3 \\
       &        & #1     & #2
}}
\newcommand{\rotationmat}[1]{\mat{\cos{#1} & \sin{#1} \\ 
                                  -\sin{#1} & \cos{#1} }}
%
% =========================
% ======= CALCULUS ========
% =========================
%
\newcommand{\diff}{\mathop{}\!d}
\newcommand{\dt}{\diff t}
\newcommand{\dx}{\diff x}
\newcommand{\dy}{\diff y}
\newcommand{\dz}{\diff z}
\newcommand*{\derivative}[2][]{\frac{d#1}{d#2}}
\newcommand*{\derivativeshort}[2][]{\fracshort{d#1}{d#2}}
\newcommand*{\dd}[2][]{\derivative[#1]{#2}} % Shortcut for \derivative
\newcommand*{\ddshort}[2][]{\derivativeshort[#1]{#2}}
\newcommand*{\partialderivative}[2][]{\frac{\partial{#1}}{\partial{#2}}}
\newcommand*{\partialderivativeshort}[2][]{\fracshort{\partial{#1}}{\partial{#2}}}
\newcommand*{\pd}[2][]{\partialderivative[#1]{#2}}
\newcommand*{\pdshort}[2][]{\partialderivativeshort[#1]{#2}}
\newcommand*{\pdx}[1][]{\pd[#1]{x}}
\newcommand*{\pdy}[1][]{\pd[#1]{y}}
\newcommand*{\pdz}[1][]{\pd[#1]{z}}
\newcommand*{\ddx}[1][]{\derivative[#1]{x}}
\newcommand*{\ddy}[1][]{\derivative[#1]{y}}
\newcommand*{\ddz}[1][]{\derivative[#1]{z}}
\newcommand*{\ddt}[1][]{\derivative[#1]{t}}
\newcommand*{\dds}[1][]{\derivative[#1]{s}}
\newcommand{\dxdt}{\derivative[x]{t}}
\newcommand{\dydt}{\derivative[y]{t}}
\newcommand{\dfdx}{\derivative[f]{x}}
\newcommand{\dxdy}{\derivative[x]{y}}
\newcommand{\xdot}{\dot{x}}
\newcommand{\xddot}{\ddot{x}}
\newcommand{\ydot}{\dot{y}}
\newcommand{\yddot}{\ddot{y}}
\newcommand{\zdot}{\dot{z}}
\newcommand{\zddot}{\ddot{z}}
\newcommand{\rdot}{\dot{r}}
\newcommand{\rddot}{\ddot{r}}
\newcommand{\thetadot}{\dot{\theta}}
\newcommand{\thetaddot}{\ddot{\theta}}
\newcommand{\omegadot}{\dot{\omega}}
\newcommand{\omegaddot}{\ddot{\omega}}
\newcommand*{\secondderivative}[2][]{\frac{d^2{#1}}{d{#2}^2}}
\newcommand{\del}{\nabla}
\newcommand{\grad}{\nabla}
\newcommand{\laplace}{\del^2}
\newcommand{\hessian}{\del^2}
\newcommand{\Jacobian}[1]{\mat{
    \partialderivative[#1_1]{x_1} 
        & \cdots 
        & \partialderivative[#1_1]{x_n} \\
    \vdots & \ddots & \vdots \\
    \partialderivative[#1_n]{x_1} 
        & \cdots 
        & \partialderivative[#1_n]{x_n} }}
\newcommand{\Hessian}[1]{\mat{
    \partialderivative[^2#1]{x_1^2} 
        & \cdots 
        & \partialderivative[^2#1]{x_1\partial x_n} \\
    \vdots & \ddots & \vdots \\
    \partialderivative[^2#1]{x_n\partial x_1} 
        & \cdots 
        & \partialderivative[^2#1]{x_n^2} }}
% Exponent Taylor Series Definition
\newcommand*{\taylorexponent}[1]{\sum^\infty_{k=1} \frac{#1^k}{k!}}
%
% **** TRANSFORMS ****
\newcommand{\Lagrangian}{\mathcal{L}}
\newcommand*{\Fourier}[1][*]{ \mathcal{F}\left[\,#1\,\right] }
\newcommand*{\Fourierinvs}[1][*]{ \mathcal{F}\invs\left[\,#1\,\right] }
\newcommand*{\Laplace}[1][*]{ \mathcal{L}\left[\,#1\,\right] }
\newcommand*{\Laplaceinvs}[1][*]{ \mathcal{L}\invs\left[\,#1\,\right] }
%
% =============================
% ======= OPTIMIZATION ========
% =============================
%
% Example usage: \convexcombo[\alpha]{x}{y}
\newcommand*{\convexcombo}[3][\theta]{{#1}{#2} + \left(1 - {#1}\right){#3}}
\newcommand{\convex}{\operatorname{conv}}
\newcommand*{\minimize}[1]{\underset{#1}{\textup{minimize}}\quad} % Usage: \minimize{\x \in \reals}
\newcommand*{\maximize}[1]{\underset{#1}{\textup{maximize}}\quad} % Usage: \maximize{\x \in \reals}
\newcommand{\subjectto}{\textup{subject to}\quad}
\newcommand{\epi}{\operatorname{epi}}
\newcommand*{\argmax}[1]{\underset{#1}{\operatorname{argmax}}\:\:}
\newcommand*{\argmin}[1]{\underset{#1}{\operatorname{argmin}}\:\:}
%
% ===============================
% ======= Control Theory ========
% ===============================
%
\newcommand{\Kinfty}{\calK_{\infty}}
\newcommand{\KL}{\mathcal{KL}}
\newcommand{\KLL}{\mathcal{KLL}}
% 
%
% ===========================================
% ======= PROBABILITY AND STATISTICS ========
% ===========================================
%
\newcommand{\expectedvalue}{\mathbb{E}}
\newcommand{\E}{\expectedvalue}
\newcommand*{\Egiven}[2]{\E\left(#1 \suchthat #2\right)}
\renewcommand{\P}{\mathbb{P}}
\renewcommand*{\choose}[2]{_{#1}C_{#2}}
% Produces P(A|B)
\newcommand{\Pgiven}[2]{\P\left(#1 \suchthat #2\right)}
% Produces P(A intesect B) / P(B)
\newcommand{\Pgivenformula}[2]{\frac{\P(#1 \cap #2)}{\P(#2)}}
\newcommand{\Var}{\operatorname{Var}}
\newcommand{\Vargiven}[2]{\Var\left(#1 \suchthat #2\right)}
\newcommand{\Cov}{\operatorname{Cov}}
\newcommand{\Covgiven}[2]{\Cov\left(#1 \suchthat #2\right)}
\newcommand{\pdf}{p.d.f. }
\newcommand{\cdf}{c.d.f. }
%
% ===========================================
% ======= REAL ANALYSIS ========
% ===========================================
%
\newcommand{\limittoinfy}[1][n]{\lim_{#1 \to \infty}}
\newcommand{\sumtoinfty}[1][n]{\sum_{#1=1}^{\infty}}
%
% ===========================================
% ======= Manifolds ========
% ===========================================
%
\newcommand{\Lie}{\mathscr{L}}
%
% =============================
% ===== PROGRESS TRACKING =====
% =============================
%
\newcommand{\boxedeq}[1]{\begin{empheq}[box={\fboxsep=6pt\fbox}]{align}#1\end{empheq}}

% Misc
\newcommand{\xaxis}{$x$-axis }
\newcommand{\yaxis}{$y$-axis }
\newcommand{\zaxis}{$z$-axis }

\newcommand{\xvalue}{$x$-value }
\newcommand{\yvalue}{$y$-value }
\newcommand{\zvalue}{$z$-value }

\newcommand{\xcoord}{$x$-coordinate }
\newcommand{\ycoord}{$y$-coordinate }
\newcommand{\zcoord}{$z$-coordinate }

% \usepackage[dvipsnames]{xcolor}
\definecolor{darkgreen}{rgb}{0.0, 0.5, 0.0}
\definecolor{darkred}{rgb}{0.8, 0.0, 0.0}

\newcommand{\red}[1]{{\color{red}#1}}
\newcommand{\green}[1]{{\color{green}#1}}
\newcommand{\blue}[1]{{\color{blue}#1}}
\newcommand{\darkred}[1]{{\color{darkred}#1}}
\newcommand{\darkgreen}[1]{{\color{darkgreen}#1}}
\newcommand{\update}[1]{\blue{#1}}
\newcommand{\updatetwo}[1]{\green{#1}}
\newcommand{\delete}[1]{\darkred{#1}}
\newcommand{\macrocolor}[1]{{\ifdraft{\color{magenta}#1}{#1}}}
\newcommand{\correction}[1]{{\color{purple}#1}}

% Iterations
\newcommand{\I}{\textrm{I}}
\newcommand{\II}{\textrm{II}}
\newcommand{\III}{\textrm{III}}
\newcommand{\IV}{\textrm{IV}}
\newcommand{\V}{\textrm{V}}

% Specific to this paper
% Caligraphy letters
\newcommand{\calA}{{\mathcal{A}}}
\newcommand{\calB}{{\mathcal{B}}}
\newcommand{\calH}{{\mathcal{H}}}
\newcommand{\calK}{{\mathcal{K}}}
\newcommand{\calS}{{\mathcal{S}}}
\newcommand{\calT}{{\mathcal{T}}}
\newcommand{\calU}{{\mathcal{U}}}
\newcommand{\calX}{{\mathcal{X}}}
\newcommand{\calY}{{\mathcal{Y}}}
\newcommand{\calZ}{{\mathcal{Z}}}
\newcommand{\calAX}{\macrocolor{\widetilde{\calA}}}

% Zero-vector
\newcommand{\zerovec}[1]{0_{#1}}

\newcommand{\sigmaone}{\macrocolor{\sigma_1}}
\newcommand{\sigmazero}{\macrocolor{\sigma_0}}
\newcommand{\epsilonv}{\macrocolor{\varepsilon_v}}
\newcommand{\taudot}{\dot{\tau}}
\newcommand{\vdot}{\dot{v}}
\newcommand{\qdot}{\dot{q}}
\newcommand{\basin}{\macrocolor{\calB_{\calA}}}
\newcommand{\vdiffcoeff}{\macrocolor{\mu}}

%  From https://tex.stackexchange.com/questions/161050/is-there-a-painless-way-to-get-self-loop-decorations-on-mathematical-symbols/161060?noredirect=1#comment1489152_161060
\usepackage{tikz}
\usepackage{stackengine}
\newcommand{\LoopFunction}[1]{%
  \renewcommand\stacktype{L}%
  \renewcommand\useanchorwidth{T}%
  \renewcommand\stackalignment{l}%
  \setbox0\hbox{$#1$}%
  \stackon[0pt]{\usebox0}{\hspace{\the\wd0}\hspace{-1.6ex}%
    \tikz \node {} edge [in=80,out=30,loop] node {} ();}%
}

\newcommand{\plantstatespace}{\macrocolor{\reals^\nplant}}
\newcommand{\plantinputspace}{\macrocolor{\reals^\mplant}}
\newcommand{\calZswitchtoone}{\macrocolor{\calZ_{0\mapsto 1}}}
\newcommand{\calZholdzero}{\macrocolor{\calZ_{0}}}
\newcommand{\calZholdone}{\macrocolor{\calZ_{1}}}
\newcommand{\calZswitchtozero}{\macrocolor{\calZ_{1\mapsto 0}}}

% Distance function
\newcommand{\distA}[1]{\macrocolor{\abs{#1}_{\calA}}}
\newcommand{\distAX}[1]{\macrocolor{\abs{#1}_{\calAX}}}

% Dynamics
\newcommand{\fp}{\macrocolor{f_P}}
\newcommand{\fv}{\macrocolor{f_v}}
\newcommand{\ftau}{\macrocolor{f_\tau}}
\newcommand{\nplant}{\macrocolor{n}}
\newcommand{\mplant}{\macrocolor{m}}

% This can be set to either contain text "or" or the logic symbol \vee.
\newcommand{\OR}{\textup{ or }}
\newcommand{\AND}{,}

% Set-specific subscripts
\newcommand{\sszero}{^{\macrocolor{0}}}
\newcommand{\ssone}{^{\macrocolor{1}}}
\newcommand{\ssb}{^{\macrocolor{b}}}

% Generalized derivatives
\newcommand{\gdd}[1]{{#1}^\circ} % directional derivative
\newcommand{\ggrad}{\macrocolor{\partial^\circ}} % gradient

\newcommand{\tangentcone}[1]{\macrocolor{T}_{#1}}

\newcommand{\proofsketch}{\par\textit{Proof sketch.} }

\renewcommand{\sec}{\operatorname{s}}
\newcommand{\msec}{\operatorname{ms}}

\newcommand{\zzero}{z_0}
\newcommand{\zone}{z_1}

% Text
\newcommand{\UGAS}{GAS\xspace}

\newcommand{\VonedotDefinition}{
  \begin{center}
    \begin{tcolorbox}[colback=white,colframe=structure,box align=center, width=0.5\linewidth]
      Let $\dot V_{1}(z) := \ip{\del V(z)}{\fp(z, \qcolor{1}{\kappa_{1}}(z))}.$\qquad
    \end{tcolorbox}
  \end{center}
  \smallskip 
}
\newcommand{\calZholdonedef}{\{z \in \plantstatespace \mid \dot V_1(z) \leq -\sigmaone(\distA{z})\}}
\newcommand{\calZswitchtozerodef}{\{z \in \plantstatespace \mid \dot V_1(z) \geq -\sigmaone(\distA{z})\}}

\newcommand{\tV}{\macrocolor{\widetilde{V}}}
\newcommand{\tVc}{\macrocolor{\gdd{\widetilde{V}}}}
\newcommand{\eqss}{{eq}}
\newcommand{\neqss}{{neq}}
\newcommand{\xneq}{\macrocolor{x_\neqss}}
\newcommand{\zneq}{\macrocolor{z_\neqss}}
\newcommand{\vneq}{\macrocolor{v_\neqss}}
\newcommand{\qneq}{\macrocolor{q_\neqss}}
\newcommand{\xeq}{\macrocolor{x_\eqss}}
\newcommand{\zeq}{\macrocolor{z_\eqss}}
\newcommand{\veq}{\macrocolor{v_\eqss}}
\newcommand{\qeq}{\macrocolor{q_\eqss}}
\newcommand{\flowdir}{\macrocolor{w}}
% \renewcommand{\calAX}


\graphicspath{{images/}{../images/}}

\newcommand{\mysubsection}[1]{\vspace{20pt plus 5pt}{\color{structure} \textbf{#1}}\par}

\usepackage[scale=0.95]{beamerposter} % Use the beamerposter package for laying out the poster

% The graphbox package adds an "align" option to the \includegraphics command that makes this very easy to certically align images. https://tex.stackexchange.com/a/350374/153678
\usepackage{graphbox}


\usepackage{paracol}

\usetheme{confposter} % Use the confposter theme supplied with this template

%-----------------------------------------------------------
% Define the column widths and overall poster size
% To set effective colspaceinner, onecolwid and twocolwid values, first choose how many columns you want and how much separation you want between columns
% In this template, the separation width chosen is 0.024 of the paper width and a 4-column layout
% onecolwid should therefore be (1-(# of columns+1)*colspaceinner)/# of columns e.g. (1-(4+1)*0.024)/4 = 0.22
% Set twocolwid to be (2*onecolwid)+colspaceinner = 0.464
% Set threecolwid to be (3*onecolwid)+2*colspaceinner = 0.708

\usepackage{calc}% http://ctan.org/pkg/calc
\newlength{\colspace}
\newlength{\colspaceinner}
\newlength{\colspaceouter}
\newlength{\onecolwid}
\newlength{\twocolwid}
\newlength{\threecolwid}
\setlength{\paperwidth}{48in} % A0 width: 46.8in
\setlength{\paperheight}{36in} % A0 height: 33.1in
% The normative space between columns/the edge of the paper.
\setlength{\colspace}{0.012\paperwidth}
 % Space between columns. This is manually adjusted to make the margins equal (there is a way to do this automatically, but I'm out of time.)
\setlength{\colspaceouter}{1.25\colspace}
 % Space between columns and edge of page (margins)
\setlength{\colspaceinner}{0.666667\colspace}
 % Width of one column=(1-5*0.012)/4
\setlength{\onecolwid}{(\paperwidth-5\colspace)/4}
 % Width of two columns=(2*onecolwid)+colspaceinner=(2*0.235)+0.012
% \setlength{\twocolwid}{0.482\paperwidth}
% \setlength{\threecolwid}{0.708\paperwidth} % Width of three columns
\setlength{\topmargin}{-0.5in} % Reduce the top margin size
%-----------------------------------------------------------

\usepackage{graphicx}  % Required for including images

\usepackage{booktabs} % Top and bottom rules for tables

%------------------------------------------------------------------
%	TITLE SECTION 
%------------------------------------------------------------------

\title{Global Asymptotic
Stability of Nonlinear Systems while Exploiting Properties of Uncertified Feedback Controllers
via Opportunistic Switching%} % Poster title

\author{Paul K.\ Wintz%$^1$ (pwintz@ucsc.edu%), Ricardo G.\ Sanfelice%$^1$, and João P.\ Hespanha%$^2$} % Author(s)

\institute{
    $^1$University of California, Santa Cruz; %, Department of Applied Mathematics; 
    % $^2$UC, Santa Cruz, the Department of Electrical and Computer Engineering%,
    $^2$University of California, Santa Barbara%, the Department of Electrical and Computer Engineering%,
    } % Institution(s)

%------------------------------------------------------------------

\newcommand{\kappazero}{\qcolor0{\kappa_0}}
\newcommand{\kappaone}{\qcolor1{\kappa_1}}

\setlength{\parskip}{40.0 pt plus 5 pt}

% \setbeamercolor{bibliography entry author}{fg=red}
% \setbeamercolor{bibliography entry title}{fg=blue}
\setbeamercolor{bibliography entry note}{fg=structure!85!white}

\begin{document}

\addtobeamertemplate{block end}{}{\vspace*{2ex}} % White space under blocks
\addtobeamertemplate{block alerted end}{}{\vspace*{2ex}} % White space under highlighted (alert) blocks

\setlength{\belowcaptionskip}{2ex} % White space under figures
\setlength\belowdisplayshortskip{2ex} % White space under equations

\begin{frame}[t] % The whole poster is enclosed in one beamer frame

\begin{columns}[t] % The whole poster consists of three major columns, the second of which is split into two columns twice - the [t] option aligns each column's content to the top

\hspace{\colspaceouter} % Left Margin
% \hfill

\begin{column}{\onecolwid} % The first column

%------------------------------------------------------------------
%	OBJECTIVES
%------------------------------------------------------------------

\begin{block}{Summary}
    We introduce a switching strategy that renders a compact set globally asymptotically stable (\UGAS) for a nonlinear continuous-time plant by switching between a Lyapunov-certified feedback controller and an uncertified controller. 
    Our switching strategy allows for the opportunistic use of a controller that has desirable performance but lacks a Lyapunov certificate.
    A pair of tunable threshold functions determine conditions for switching between the controllers.
\end{block}

%------------------------------------------------------------------
%	INTRODUCTION
%------------------------------------------------------------------

\begin{block}{Preliminaries}
    Consider a continuous-time plant 
    \begin{equation}
        \dot z = \fp(z,u), \quad 
        z \in \plantstatespace, u \in  \plantinputspace.
        \label{eq:plant}
    \end{equation}
    and a nonempty set $\calS \subset \plantstatespace$.
    A controller ${\kappa : \plantstatespace \to \plantinputspace}$ is called \emph{Lyapunov-certified} for $\calS$ if $\calS$ is \UGAS for the closed-loop system 
    \begin{equation}
        \zdot = f(z, \kappa(z))
        \label{eq:closed-loop with kappa}
    \end{equation}
    and a Lyapunov function is known that that certifies $\calS$ is \UGAS.
    % The controller $\kappa$ is called \emph{uncertified} if either no Lyapunov function exists (because $\calA$ is not \UGAS for \cref{eq:closed-loop with kappa}) or no Lyapunov function is known.
\end{block}

\begin{block}{Problem Statement}
    
    Suppose we are given a continuous plant as in \cref{eq:plant}, a compact set $\calA \subset \plantstatespace$, and two continuous controllers:
    \begin{itemize}
        \item[] \qcolor0{$\kappa_0$:} a Lyapunov-certified controller 
            % that renders $\calA$ to be \UGAS for $\dot z = \fp(z, \kappa_0(z))$
        \item[] \qcolor1{$\kappa_1$:} any continuous controller (i.e., an \emph{uncertified} controller).
    \end{itemize}
    We want to design a switching logic for ${q\in\{0,1\}}$ (which determines whether $u = \kappazero(z)$ or $u=\kappaone(z)$ is used) such that $\calA$ is \UGAS, $\kappaone$ is preferred over $\kappazero$, and the switching does not exhibit chattering. 
    \medskip

    \begin{center}
        \hspace{-12pt} % Make image better centered
        \includegraphics[width=\linewidth]{feedback_diagram_switching_logic.drawio.pdf}
    \end{center}
    \vspace{-10pt}
    The switching logic passes $q$ into a switch that determines whether $\kappazero$ or $\kappaone$ is applied to the plant.
\end{block}
\vspace{-10pt}

\begin{block}{Why Use an Uncertified Controller?}

An uncertified controller may have ``better'' properties compared to available certified controllers, such as

\begin{itemize}
    \item Less fuel use
    \item Faster convergence 
    \item Fewer computational demands 
\end{itemize}
% \begin{figure}
%     \begin{tabular}{lll}
%         $\bullet$
%     \end{tabular}
% \end{figure}

\mysubsection{Examples}
\begin{itemize}
  \item Linear quadratic regulator for 
  the linearization of a system with an unknown basin of attraction.
  \item Model predictive control that 
  occasionally fails to compute an update.
  \item Black box controllers (e.g., neural network controllers).
\end{itemize}%
Our switching logic could be particularly useful for reinforcement learning control, which demonstrates good empirical results, but for which it is often difficult to produce Lyapunov certificates.
\end{block}


% \begin{center}
%     % \includegraphics[width=0.95\linewidth]{UCSC_BaskinEng_Logo_wide}

%     \begin{tabular}{lcr}
%         \includegraphics[height=0.15\linewidth]{DEVCOM_ARL_LOGO.png} &
%         \includegraphics[height=0.23\linewidth]{AFOSR_logo.png} \\
%         \includegraphics[height=0.15\linewidth]{ONR_logo.png} &\includegraphics[height=0.23\linewidth]{nsf_logo}\\ 
%     \end{tabular}
% \end{center}

% \begin{figure}
%     \centering
%     \includegraphics[width=0.4\linewidth]{HSLlogo.eps}
% \end{figure}

\end{column} % End of the first column
\hspace{\colspaceinner}
\begin{column}{\onecolwid}% Beginning of second column

\begin{block}{Switching Strategy}
    Let $V$ be the Lyapunov function for $\dot z = f_P(z, \kappazero(z))$. 
    The rate of change of $V(z)$ under $\qcolor1{\kappa_1}$ is central to our discussion. For each $z \in \plantstatespace$, we define 
    $$\dot V_{1}(z) := \ip{\del V(z)}{\fp(z, \qcolor1{\kappa_1}(z))}.$$

    Pick any two continuous \emph{threshold} functions ${\sigmazero, \sigmaone : \nonnegativereals \to \nonnegativereals}$ such that
    $\sigmaone$ is positive definite and $\sigmazero(s) > \sigmaone(s)$ for all $s \geq 0$.
    % \begin{itemize}%
    %     \item $\sigmaone$ is positive definite
    %     \item $\sigmazero(s) > \sigmaone(s)$ for all $s \geq 0$.
    % \end{itemize}% 
    % These functions define \emph{thresholds} for switching.
    % \smallskip

    For \qcolor0{$q = 0$}, $\dot V_1$ is ``small enough to switch to \qcolor1{$q = 1$}'' if $$\dot V_1(z) \leq -\sigmazero(\distA{z}).$$
    For \qcolor{1}{$q = 1$}, $\dot V_1$ is ``large enough to switch to~\qcolor0{$q = 0$}''~if $$\dot V_1(z) \geq -\sigmaone(\distA{z}).$$

    % For \qcolor0{$q = 0$}, $\dot V_1$ is \dots
    % \begin{itemize}
    %     \item[] \dots ``small enough to switch to \qcolor1{$q = 1$}'' if $\dot V_1(z) \leq -\sigmazero(\distA{z})$.
    %     \item[] \dots ``large enough to hold \qcolor0{$q = 0$}'' if $\dot V_1(z) \geq -\sigmazero(\distA{z}).$
    % \end{itemize}
    % For \qcolor{1}{$q = 1$}, $\dot V_1$ is \dots
    % \begin{itemize}
    %     \item[]\dots ``small enough to hold~\qcolor{1}{$q = 1$}'' if $\dot V_1(z) \leq -\sigmaone(\distA{z})$.
    %     \item[]\dots ``large enough to switch to~\qcolor0{$q = 0$}''~if $\dot V_1(z) \geq -\sigmaone(\distA{z}).$
    % \end{itemize}
    \begin{figure}[ht]
        \centering
        \includegraphics[width=\linewidth]{switching_regions_poster.pdf}
        % \setlength{\belowcaptionskip}{-100pt}
    \end{figure}
    A switch from $\qcolor0{q = 0}$ to $\qcolor1{q = 1}$ occurs immediately when $\dot V_1$ is small enough to switch (i.e., $\kappaone$ will perform well, as measured by decrease in $V(z)$).
    A switch from $\qcolor1{q = 1}$ to $\qcolor0{q = 0}$ does \textbf{not} occur immediately when $\dot V_1$ is large enough to switch.
    To prevent chattering via hysteresis, we introduce an auxiliary memory variable $v \in \nnreals$ that records the value of $V(z)$ at each switch and decreases along flows, converging to zero. A switch from $\qcolor1{q = 1}$ to $\qcolor0{q = 0}$ occurs if and only if $V(z) \geq v$ and $\dot V_1$ is large enough to switch.
    In practice, a switch from $\qcolor1{q = 1}$ to $\qcolor0{q = 0}$ almost always occurs when $V(z) = v$ and $\dot V_1(z) > -\sigmaone(\distA{z})$.
    Thus, $v$ acts as an upper bound for $V(z)$, restricting how much $V(z)$ can grow before triggering a switch to $\qcolor0{q = 0}$. 
    Because $v$ converges to zero, $V(z)$ is squeezed to zero as well.
% \end{block}
%------------------------------------------------------------------
% \begin{block}{Dynamics of the Closed-Loop System}

    \mysubsection{At each switch:}
    \begin{columns}
        \begin{column}{0.26\linewidth}
            \begin{itemize}
                \item $z$ is unchanged
                \item $v$ is set to $V(z)$ % to record $V$'s value
            \end{itemize}
        \end{column}
        \begin{column}{0.69\linewidth} 
            \begin{itemize}
                \item $q$ is toggled to the opposite value in $\setdef{0, 1}$
                \item[\ ]
            \end{itemize}
        \end{column}
    \end{columns}

    \mysubsection{Between switches:}
    \begin{columns}
        \begin{column}{0.65\linewidth}
            \begin{itemize}
                \item $z$ evolves according to $\dot z = \fp(z, \kappa_q(z))$  
            \end{itemize}
        \end{column}
        \begin{column}{0.31\linewidth} 
            \begin{itemize}
                \item $q$ is constant
            \end{itemize}
        \end{column}
    \end{columns}
    \begin{itemize}
        \item $v$ evolves according to 
        \begin{equation*}
            \vdot := 
            \begin{cases}
                {-}v,& \textif \qcolor0{q = 0 \text{ (certified)} } , \\
                {-}\sigmaone(\distA{z}) + \vdiffcoeff(V(z) - v), & \textif \qcolor1{q = 1 \text{ (uncertified)} },
            \end{cases}
        \end{equation*}
        where $\mu > 0$ is parameter.
    \end{itemize} 

    The dynamics of $v$ are designed such that:
    \begin{itemize}
        \item While $\qcolor1{q=1}$, if $\dot V_1$ is small enough to switch to $\qcolor0{q=0}$, then $v$ will (if given enough time) catch up to $V(z)$, causing a switch to $\qcolor0{q=0}$. 
        
        \item While $\qcolor1{q=1}$, if $\dot V_1$ is not small enough to switch to $\qcolor0{q=0}$, then $V(z)$ will not catch up to $v$.
    
        \item After every switch from $\qcolor0{q=0}$ to $\qcolor1{q=1}$, there is an interval where no switches can occur because $V(z) < v$. This prevents chattering.
    \end{itemize}%
    \vspace{-20pt}
\end{block}

\begin{alertblock}{Theorem}
    Suppose that $\fp,$ $\kappa_0,$ and $\kappa_1$ are continuous and $V$ is continuously differentiable.
    Then, the set $\calAX := \setdef{(z, v, q) \mid z\in \calA, v = 0}$ is (uniformly) globally asymptotically stable for the closed-loop system.
\end{alertblock}
\vspace{-20pt}

\textit{Remark.} The \UGAS property of $\calAX$ is robust to small~perturbations.

\end{column} % End of second column
\hspace{\colspaceinner}
\begin{column}{\onecolwid} % Beginning of thrid column

\begin{block}{Example: Switch Due to Slow Convergence}
    Consider the plant $\dot z = u$ 
    with $z, u\ {\in}\ \reals$ and controllers $\kappazero(z) := -z,$ 
    $\kappaone(z) := -z^3$. 
    The feedback $\kappaone$ produces fast convergences far from the origin $\calA := \setdef{0}$ but slows as $z$ approaches zero, whereas the convergence is slower for $\kappazero$ far from the origin, but faster near it.
    The switched solution uses $\kappaone$ far from the origin and 
    $\kappazero$ near the origin, producing overall faster convergence.
    % and pick $\sigmaone(s) := s^2$ and $\sigmazero(s) := 1.5 s^2 + 10^{-3}$ for all $s \geq 0.$
    \begin{figure}
        \centering
        \includegraphics[width=0.9\linewidth]{languish_example_comparison_poster}
    \end{figure}
    % A plot of $V$, $v$, $\dot V_1$ and the threshold functions show that $\dot V_1$ passes above $-\sigma_1(\distA{z})$ at $t=0.4$ but the switch to $\qcolor0{q=0}$ does not occur until $V(z) = v$ at $t=0.95$.
    The switch occurs when $V(z) = v$ (\emph{after} $\dot V_1$ moves above~$-\sigma_1(\distA{z})$).

    \begin{figure}[h]
        \centering
        \includegraphics[width=0.9\linewidth]{languish_example_poster}
    \end{figure}
\end{block}
\vspace{-25pt}

\newcommand{\fsampled}{{f_{s}}}%
\newcommand{\zsampled}{{z_{s}}}%
\newcommand{\usampled}{{u_{s}}}%
\newcommand{\Tsample}{{T}}%
\newcommand{\Tactual}{{T_{c}}}%
\begin{block}{Example: Model Predictive Controller}
    Consider a nonlinear plant 
    $\zdot = u$
    and two controllers:
    \begin{itemize}
        \item[] \qcolor{0}{$\kappa_0$}: Lyapunov-certified controller 
        \item[] \qcolor{1}{$\kappa_1$}: Model predictive controller (MPC) with $1 \msec$ sampling period 
    \end{itemize}

    Suppose the MPC algorithm fails to compute the next input by $t=1\msec$, so it reuses the previous input. This causes $z$ to move away from $\calA$.

    \begin{center}
        \includegraphics[width=\linewidth]{mpc_example_kappa0_and_kappa1}
    \end{center}
    Our switching scheme uses $\kappaone$ until ${V(z) = v}$, then it switches to $\kappazero$.
    
    \hspace{30pt}\includegraphics[width=0.9\linewidth]{mpc_example_switched_poster}
\end{block}

\end{column} % End of the third column
\hspace{\colspaceinner}
\begin{column}{\onecolwid} % The fourth column

\begin{block}{Example: Linear Quadratic Regulator}
    Consider the nonlinear plant
    \begin{equation*}
    \zdot = A z + u + \underbrace{f(z, u) }_{\mathclap{\text{Nonlinear component}}}
    % \zdot = A_1 z + \min(\norm{z}, 1) A_2 z + u
    \qquad z, u \in \reals^2.
    \end{equation*}%
    Suppose the origin is \UGAS for a linear high-gain Lyapunov-certified controller $\qcolor0{\kappa_0}(z) := Kz$ and that $\qcolor1{\kappa_1}$ is the LQR feedback 
    that solves 
    \begin{align*}
        \minimize{u} & \int_0^\infty \norm{z(t)}^2 + \norm{u(t)}^2 \dt \quad & 
        \subjectto & \zdot = \underbrace{Az + u}_{\mathclap{\text{Linearized dynamics}}}.
    \end{align*}

    For a small choice of $\mu > 0$, $v$ decreases slowly while $\qcolor1{q=1}$, allowing $\dot V_1$ to briefly pass above $-\sigmaone(\distA{z})$ without triggering a switch.
    \begin{center}
        \includegraphics[width=\linewidth]{lqr_example_1_poster} 
    \end{center}
    When $\mu$ is larger, $v$ decreases faster, causing $V(z)$ to reach $v$ while $\dot V_1(z)$ is above the threshold $-\sigmaone(\distA{z}),$ triggering a switch to $\qcolor0{q=0}$.
    \begin{center}
        \includegraphics[width=\linewidth]{lqr_example_16_poster}
    \end{center}
    \vspace{-30pt}
\end{block}

\begin{block}{References}
    \renewcommand*{\bibfont}{\small}
    \nocite{wintz_global_2022}
    \nocite{sanfelice_hybrid_2021}
    \nocite{sanfelice_invariance_2007}
    % \nocite{clarke_optimization_1990}
    % \nocite{sanfelice_toolbox_2013}
    % \nocite{aubin_tangent_2009}
    \printbibliography
\end{block}
\vspace{-20pt}

%------------------------------------------------------------------
%	ACKNOWLEDGEMENTS
%------------------------------------------------------------------

\begin{block}{Acknowledgements}
    \small{\rmfamily{This research was supported by the National Science Foundation under grant nos.\ ECS-1710621, CNS-1544396, and CNS-2039054; by the Air Force Office of Scientific Research under grant nos.\ FA9550-19-1-0053, FA9550-19-1-0169, and FA9550-20-1-0238; by the Army Research Office under grant no.\ W911NF-20-1-0253; and by the U.S. Office of Naval Research under the MURI grant no.\ N00014-16-1-2710.}} 
    \vspace{20pt}
    
    \begin{minipage}{\linewidth}
        \includegraphics[height=0.16\linewidth,align=c]{nsf_logo}\hfill
        \includegraphics[height=0.17\linewidth,align=c]{AFOSR_logo.png}\hfill
        \includegraphics[height=0.11\linewidth,align=c]{DEVCOM_ARL_LOGO.png}\hfill 
        \includegraphics[height=0.13\linewidth,align=c]{ONR_logo.png} 
    \end{minipage}
\end{block}

%------------------------------------------------------------------

\end{column} % End of the fourth column

% \hspace{100\colspaceouter}
\hfill
\end{columns} % End of all the columns in the poster

%% Draw an overlay with tikz to aid in getting the margins equal. 
% \begin{tikzpicture}[remember picture, overlay,
%     help lines/.append style={line width=0.05pt}]
%     \draw (current page.west) -- +(\colspaceouter, 0);
%     \draw (current page.east) -- +(-\colspaceouter, 0);
% \end{tikzpicture}%

\end{frame} % End of the enclosing frame

\end{document}
