\documentclass{beamer}
% PRESENTATION 
%\documentclass{beamer}
% TO MAKE HANDOUT use
%\documentclass[notes=hide,handout]{beamer}
%\pgfpagesuselayout{4 on 1}[letterpaper,border shrink=5mm] 

%%%%%%%%%%%%%%%%%%%%%%%%%%%%%%%%
%%%%%%%%%%%% PACKAGES %%%%%%%%%%%%%
%%%%%%%%%%%%%%%%%%%%%%%%%%%%%%%%
\usepackage{psfrag}
\usepackage{amsmath}
\usepackage{lipsum}
\usepackage{ifthen}
\usepackage{verbatim}
\usepackage{calc}
\usepackage{color}
\usepackage{colortbl}
\usepackage{graphics,graphicx,amssymb}
\usepackage{amsxtra}
\usepackage{epsfig}
\usepackage{movie15}
\usepackage{pdfpages}
\usepackage{pgfpages}
\usepackage{tikz}
\usepackage{pgflibraryshapes}
\usepackage{lipsum}

%%%%%%%%%%%%%%%%%%%%%%%%%%%%%%%%
%%%%%%%%%%%% MACROS %%%%%%%%%%%%%%
%%%%%%%%%%%%%%%%%%%%%%%%%%%%%%%%

\usepackage{pwintz_configuration}
% \usepackage{pwintz_definitions}
\usepackage{rgsMacros}
\usepackage{rgsBeamerv01}

\renewcommand{\em}{\it \color{blue}}

\definecolor{lightblue}{RGB}{60,60,200}
\definecolor{mygreen}{RGB}{0,128,0}
\definecolor{catch}{RGB}{0,128,0}
\definecolor{throw}{RGB}{0,0,255}
\definecolor{recovery}{RGB}{255,0,0}

\setbeamercolor{block title}{bg= lightblue!10!white}
\setbeamercolor{block body}{bg= lightblue!5!white}
\setbeamercolor{block title alerted}{bg=blue!10!white}
\setbeamercolor{block body alerted}{bg=blue!3!white}
\setbeamerfont{block title}{size={}}

\renewcommand{\comment}[1]{\vspace{1in}{\color{red}\noindent \tt \tiny #1}}
\newcommand{\BAn}[1]{{\cal B}_{#1}}
\newcommand{\B}{{\cal B}}
\renewcommand{\S}{{\cal S}}
\newcommand{\E}{{\cal E}}
\renewcommand{\P}{{\cal P}}
\renewcommand{\Q}{{\cal Q}}
\renewcommand{\K}{{\cal K}}
\renewcommand{\U}{{\cal U}}

%\newcommand{\location}{\ - CDC2008\ }
\newcommand{\location}{}

%%%%%%%%%%%%%%%%%%%%%%%%%%%%%%%%
%%%%%%%%%%%% SETTINGS %%%%%%%%%%%%%%
%%%%%%%%%%%%%%%%%%%%%%%%%%%%%%%%

\usetheme{rgsv02nopic}
%\pgfpagesuselayout{4 on 1}[letterpaper,border shrink=5mm] 

\mode <all>
\graphicspath{{images}}

\makeatletter
% Populate title using value defined in document_setup.tex
\title{\pwintz@title}
\subtitle{\pwintz@subtitle}

% \title{\LARGE Title}
\author{\csname pwintz@author1\endcsname}
\institute{Hybrid Systems Lab\\ University of California\\ Department of Computer Engineering\\ Santa Cruz, CA %\\
%\vspace{0.1in}{Andrew R. Teel (UCSB) and Rafal Goebel (LUC)}
}
\makeatother

\def\cmd#1{\texttt{\textbackslash #1}}
\def\env#1{\texttt{#1}}

% Temporarily out
%\addlogo{\ucsblogo}
%\addlogo{\mitlogo}
%\addlogo{\ensmplogo}

% Use this line to get some text in the logo bar as well
\logobartext{\csname pwintz@author1\endcsname{} --- University of California}

\setbeamertemplate{navigation symbols}{}

% \setwhitesheetcolors

\begin{document}
\makeatletter

%-------------------------------------------------------------------------------
%                                                                     title page

\begin{frame}
    \setbluesheetcolors
    \normalcolorhack
    \titlepage
    \includegraphics[height = 11mm, align=c]{images/UCSC_BaskinEng_Logo_wide.eps}
    \hfill 
    \includegraphics[height = 13.5mm, align=c]{images/HSLlogo.eps}
    \vspace{20pt}% Increase bottom padding.
\end{frame}

\setwhitesheetcolors
\normalcolorhack

%-------------------------------------------------------------------------------
%                                                                  Outline

%\begin{frame}[t,label=outline]{{\color{black}Outline}}
%\tableofcontents
%\end{frame}

%-------------------------------------------------------------------------------
%                                                                  Outline slide repeated at the beginning of every section

%\AtBeginSection[] % Do nothing for \section* 
%{ 
%\begin{frame}[t,label=outline]{{\color{black}Outline}}
%\tableofcontents[currentsection] 
%\end{frame} 
%} 

%-------------------------------------------------------------------------------
%                                                                  Problem Outline


\begin{frame}[t]{Slide With Top-aligned Text}
  \lipsum[1][1-3]
  \begin{itemize}
    \item \lipsum[2][1]
    \item \lipsum[2][2]
    \item \lipsum[2][3]
    \item \lipsum[2][4]
    \item \lipsum[2][5]
  \end{itemize}
\end{frame}
  
\begin{frame}{Slide with Center-Aligned Text}
  \lipsum[2]
\end{frame}
  
\begin{frame}[t]{Slide with Columns}

  \begin{minipage}[t][\textheight][t]{\linewidth}
    \begin{columns}
      \begin{column}[T]{0.48\textwidth}
        \lipsum[3][1-10]
      \end{column}
      \hfill
      \begin{column}[T]{0.48\textwidth}
        \lipsum[4][1]
        \begin{center}
          \includegraphics[width=0.6\linewidth]{example-image-a}
        \end{center}
        \lipsum[4][2-5]
      \end{column}
    \end{columns}
  \end{minipage}

\end{frame}


\begin{frame}[plain]
  \vspace{0.4in}
  \begin{center}
    \vfill
    % \only<0| handout:1>{
      {\Huge\structure{Questions?}}
    % }
    % \only<1| handout:0>{
    %   \animategraphics[autoplay,width=0.8\linewidth]{30}{animation/QuestionsScene}{0000}{0052}
    % }
    \vfill
  %   \begin{description}
  %     \item[NSF] CNS-2039054 and CNS-2111688; 
  %     \item[AFOSR] FA9550-19-1-0169, FA9550-20-1-0238, FA9550-23-1-0145, and FA9550-23-1-0313; 
  %     \item[AFRL] FA8651-22-1-0017 and FA8651-23-1-0004; 
  %     \item[ARO] W911NF-20-1-0253; 
  %     \item[DOD] W911NF-23-1-0158.
  %   \end{description}
  \end{center}

  

  \begin{columns}
    \hfill
    \begin{column}[T]{0.48\textwidth}
      \centering
      \structure{\textbf{Slides and paper available at \url{paulwintz.com/publications}.}}
        \begin{center}
          \qrset{height=1.4in,nolink}%
          \qrcode{paulwintz.com/publications/wintz-forward-2023}
        \end{center}
    \end{column}
    \hfill
    \begin{column}[T]{0.48\textwidth}
      \centering
      \structure{\textbf{Funding}}
      \medskip

      \begin{minipage}{\linewidth}%
        \centering
        \hfill
        \includegraphics[height=0.30\linewidth,align=c]{nsf_logo.png}% The 'align=c' argument needs '\usepackage{graphbox}' 
        \hfill
        \includegraphics[height=0.10\linewidth,align=c]{AFRL_Logo.png} 
        \hfill
      \end{minipage}%

      \begin{minipage}{\linewidth}%
        \centering
        \hfill
        \includegraphics[height=0.30\linewidth,align=c]{AFOSR_logo.png} 
        \hfill
        \includegraphics[height=0.16\linewidth,align=c]{DEVCOM_ARL_LOGO.png} % The Army Research Laboratory includes the Army Research Office.
        \hfill
      \end{minipage}%
    \end{column}
    \hfill
  \end{columns}
  
\end{frame}

%-------------------------------------------------------------------------------
% To substract some of the final slides
\addtocounter{framenumber}{0}
\end{document}