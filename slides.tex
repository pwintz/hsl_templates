% For rendering each frame as one page, use 'handout'.
\documentclass[notheorems, aspectratio=169, presentation]{beamer}

% % For presentation mode
% \setbeamertemplate{note page}[plain] % Don't show outline or next slide.
% \setbeameroption{show notes on second screen=right}


% For handout mode
%   \setbeamertemplate{note page}[plain] % Don't show outline or next slide.
%   \setbeameroption{show notes}

%%%%%%%%%%%%%%%%%%%%%%%%%%%%%%%%
%%%%%%%%%%%% PACKAGES %%%%%%%%%%%%%
%%%%%%%%%%%%%%%%%%%%%%%%%%%%%%%%
\usepackage{amsmath}
\usepackage{ifthen}
\usepackage{adjustbox}
\usepackage{color}
\usepackage{graphics,graphicx,amssymb}
\usepackage{amsxtra}
\usepackage{amsfonts}
\usepackage{epsfig}
\usepackage{multicol}
\usepackage{tikz}
\usepackage{animate}
% \usepackage{multimedia} % Provides \movie.
\usetikzlibrary{calc}

\usepackage{pwintz_configuration}
\usepackage{pwintz_definitions}
% Insert definitions used in your document here. 
\usepackage{demo_setup} % TODO: Delete this line.

% Set the space between paragraphs, with flexible glue.
\setlength{\parskip}{7pt plus2pt minus4pt}
\newcommand{\setparskip}{\setlength{\parskip}{7pt plus2pt minus4pt}}
\AtBeginEnvironment{column}{\setlength{\parskip}{7pt plus2pt minus4pt}}

\graphicspath{{images/}{../images/}}

\newcommand{\mycolorbox}[1]{
  \begin{tcolorbox}[colback=white,colframe=structure]
    #1
  \end{tcolorbox}
}

%%%%%%%%%%%%%%%%%%%%%%%%%%%%%%%%
%%%%%%%%%%%% MACROS %%%%%%%%%%%%%%
%%%%%%%%%%%%%%%%%%%%%%%%%%%%%%%%

\usepackage{rgsMacros}

\renewenvironment{remark}{\textit{\structure{Remark.}}}{}
\newcommand{\mstructure}[1]{{\color{structure} #1}}
\newcommand{\shortfootcite}[2][1-]{\footnote<#1>{\citet{#2}.}}
\let\footnoterule\relax % Hide line above footnotes.

\makeatletter
% Populate title using value defined in document_setup.tex
\title{\pwintz@title}
\makeatother

\author{% 
% Populate author information from definitions in document_setup.tex
    % Insert first author, if defined.
    \csname pwintz@author1\endcsname%
    \textsuperscript{\csname pwintz@authorInstitute1\endcsname}%
    %
    % Insert second author, if defined.
    \ifcsdef{pwintz@author2}{, \ifcsdef{pwintz@author3}{}{and }}{}% Add "," or ", and"
    \ifcsdef{pwintz@author2}{% If defined
        \qquad \csname pwintz@author2\endcsname%
        \textsuperscript{\csname pwintz@authorInstitute2\endcsname}%
    }{}%
    % Insert third author, if defined.
    \ifcsdef{pwintz@author3}{, \ifcsdef{pwintz@author4}{}{and }}{}% Add "," or ", and"
    \ifcsdef{pwintz@author3}{% If defined
        \qquad \csname pwintz@author3\endcsname%
        \textsuperscript{\csname pwintz@authorInstitute3\endcsname}%
    }{}%
    % Insert fourth author, if defined.
    \ifcsdef{pwintz@author4}{, \ifcsdef{pwintz@author5}{}{and~}}{}% Add "," or ", and"
    \ifcsdef{pwintz@author4}{% If defined
        \qquad\csname pwintz@author4\endcsname%
        \textsuperscript{\csname pwintz@authorInstitute4\endcsname}%
    }{}
}

\institute{ % Populate institutions that are defined in document_setup.tex
    \ifcsdef{pwintz@institute1}{% If defined
        \textsuperscript{1}\csname pwintz@institute1\endcsname%
    }{}% Else nothing
    \ifcsdef{pwintz@institute2}{% If defined
       \\ \textsuperscript{2}\csname pwintz@institute2\endcsname%
    }{}% Else nothing
    \ifcsdef{pwintz@institute3}{% If defined
       \\ \textsuperscript{3}\csname pwintz@institute3\endcsname%
    }{}% Else nothing
    \ifcsdef{pwintz@institute3}{; }{}% Add "; " if there is another institute
    \ifcsdef{pwintz@institute4}{% If defined
       \\ \textsuperscript{4}\csname pwintz@institute3\endcsname%
    }{}% Else nothing
} 


\newcommand{\itemnote}[1]{\note[item]<.->{#1}}

\begin{document}
\makeatletter

% TITLE FRAME
\begin{frame}[plain]
  \maketitle
  \includegraphics[height = 11mm,align=c]{UCSC_BaskinEng_Logo_wide.eps}
  \hfill 
  \includegraphics[height = 13.5mm,align=c]{HSLlogo.eps}
  \hspace{25pt}
\end{frame}


\begin{frame}[t]{Slide With Top-aligned Text}
  \lipsum[1][1-3]
  \begin{itemize}
    \item \lipsum[2][1]
    \item \lipsum[2][2]
    \item \lipsum[2][3]
    \item \lipsum[2][4]
    \item \lipsum[2][5]
  \end{itemize}
\end{frame}
  
\begin{frame}{Slide with Center-Aligned Text}
  \lipsum[2]
\end{frame}
  
\begin{frame}[t]{Slide with Columns}

  \begin{minipage}[t][\textheight][t]{\linewidth}
    \begin{columns}
      \begin{column}[T]{0.48\textwidth}
        \lipsum[3][1-10]
      \end{column}
      \hfill
      \begin{column}[T]{0.48\textwidth}
        \lipsum[4][1]
        \begin{center}
          \includegraphics[width=0.6\linewidth]{example-image-a}
        \end{center}
        \lipsum[4][2-5]
      \end{column}
    \end{columns}
  \end{minipage}

\end{frame}


\begin{frame}[plain]
  \vspace{0.4in}
  \begin{center}
    \vfill
    % \only<0| handout:1>{
      {\Huge\structure{Questions?}}
    % }
    % \only<1| handout:0>{
    %   \animategraphics[autoplay,width=0.8\linewidth]{30}{animation/QuestionsScene}{0000}{0052}
    % }
    \vfill
  %   \begin{description}
  %     \item[NSF] CNS-2039054 and CNS-2111688; 
  %     \item[AFOSR] FA9550-19-1-0169, FA9550-20-1-0238, FA9550-23-1-0145, and FA9550-23-1-0313; 
  %     \item[AFRL] FA8651-22-1-0017 and FA8651-23-1-0004; 
  %     \item[ARO] W911NF-20-1-0253; 
  %     \item[DOD] W911NF-23-1-0158.
  %   \end{description}
  \end{center}

  

  \begin{columns}
    \hfill
    \begin{column}[T]{0.48\textwidth}
      \centering
      \structure{\textbf{Slides and paper available at \url{paulwintz.com/publications}.}}
        \begin{center}
          \qrset{height=1.4in,nolink}%
          \qrcode{paulwintz.com/publications/wintz-forward-2023}
        \end{center}
    \end{column}
    \hfill
    \begin{column}[T]{0.48\textwidth}
      \centering
      \structure{\textbf{Funding}}
      \medskip

      \begin{minipage}{\linewidth}%
        \centering
        \hfill
        \includegraphics[height=0.30\linewidth,align=c]{nsf_logo.png}% The 'align=c' argument needs '\usepackage{graphbox}' 
        \hfill
        \includegraphics[height=0.10\linewidth,align=c]{AFRL_Logo.png} 
        \hfill
      \end{minipage}%

      \begin{minipage}{\linewidth}%
        \centering
        \hfill
        \includegraphics[height=0.30\linewidth,align=c]{AFOSR_logo.png} 
        \hfill
        \includegraphics[height=0.16\linewidth,align=c]{DEVCOM_ARL_LOGO.png} % The Army Research Laboratory includes the Army Research Office.
        \hfill
      \end{minipage}%
    \end{column}
    \hfill
  \end{columns}
  
\end{frame}


\end{document}