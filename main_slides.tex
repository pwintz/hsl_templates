% For rendering each frame as one page, use 'handout'.
\documentclass[notheorems, aspectratio=169, presentation]{beamer}

\makeatletter
\def\input@path{{./packages/}}
\makeatother

% % For presentation mode
% \setbeamertemplate{note page}[plain] % Don't show outline or next slide.
% \setbeameroption{show notes on second screen=right}

% For handout mode
%   \setbeamertemplate{note page}[plain] % Don't show outline or next slide.
%   \setbeameroption{show notes}

%%%%%%%%%%%%%%%%%%%%%%%%%%%%%%%%%%%
%%%%%%%%%%%% PACKAGES %%%%%%%%%%%%%
%%%%%%%%%%%%%%%%%%%%%%%%%%%%%%%%%%%

\usepackage{ifthen}
\usepackage{adjustbox}
\usepackage{amsfonts}
\usepackage{multicol}
\usepackage{animate}
% \usepackage{multimedia} % Provides \movie.
\usepackage{tikz}
\usetikzlibrary{calc}

\usepackage{pwintz_configuration}
\usepackage{pwintz_definitions}
%%%%%%%%%%%%%%%%%%%%%%%%%%%%%%%%%%%%%%%%%
%%%%%%%%%% DOCUMENT PACKAGES %%%%%%%%%%%%
%%%%%%%%%%%%%%%%%%%%%%%%%%%%%%%%%%%%%%%%%
% Include packages specific to this document here.

%%%%%%%%%%%%%%%%%%%%%%%%%%%%%%%%%%%%%%%%%%%
%%%%%%%%%% TYPESETTING OPTIONS %%%%%%%%%%%%
%%%%%%%%%%%%%%%%%%%%%%%%%%%%%%%%%%%%%%%%%%%

% Define hyphenation for technical words.
\hyphenation{}

%%%% Use \varphi instead of \phi. Comment to disable. %%%%
\let\phi\varphi

%%%% Use \varepsilon instead of \epsilon. Comment to disable. %%%%
\let\epsilon\varepsilon

%%%%%%%%%%%%%%%%%%%%%%%%%%%%%%%%%%%%%%%%%%%
%%%%%%%%%%%%%%% IMAGE PATH %%%%%%%%%%%%%%%%
%%%%%%%%%%%%%%%%%%%%%%%%%%%%%%%%%%%%%%%%%%%
% Define the search path for images.
\graphicspath{% List directories to search for images.
    {images/}%
}

%%%%%%%%%%%%%%%%%%%%%%%%%%%%%%%%%%%%%%%%%%%
%%%%%%%%%%%%% COLOR OPTIONS %%%%%%%%%%%%%%%
%%%%%%%%%%%%%%%%%%%%%%%%%%%%%%%%%%%%%%%%%%%

% Colors for annotations.
\ifdraft{% Is draft
    \colorlet{unimportant}{black!35!white}
    \colorlet{not ready}{black!35!white}
    \colorlet{added}{blue!80!black} 
    \colorlet{added unimportant}{blue!35!white}
    \colorlet{added important}{darkred}
}{% Not draft
    \colorlet{unimportant}{black}
    % \colorlet{not ready}{black!35!white}
    \colorlet{added}{black} 
    \colorlet{added unimportant}{black}
    \colorlet{added important}{black}
}

%%%%%%%%%%%%%%%%%%%%%%%%%%%%%%%%%%%%%%%%%%%
%%%%%%%%%%%%% References %%%%%%%%%%%%%%%%%%
%%%%%%%%%%%%%%%%%%%%%%%%%%%%%%%%%%%%%%%%%%%

\makeatletter
    \@ifpackageloaded{biblatex}{% If biblatex package...
        % Import the bibliography files (BibLaTeX)
        \addbibresource{biblio.bib} % Add in biblio.bib for most citations.
    }{}%
\makeatother

% Add in biblio_report.bib for citing the report from the conference version.
% \addbibresource{biblio_report.bib} 

%%%%%%%%%%%%%%%%%%%%%%%%%%%%%%%%%%%%%%%%%%%%%%%%%%%%%%%%%%%
%%%%%%%%%%%%% DOCUMENT HIDE/SHOW TOGGLES %%%%%%%%%%%%%%%%%%
%%%%%%%%%%%%%%%%%%%%%%%%%%%%%%%%%%%%%%%%%%%%%%%%%%%%%%%%%%%

% WORK IN PROGRESS
% \setbool{showWorkingNotesInDraft}{true}
% \setbool{showProofsInDraft}{true}

% \ifdraft{
%     % Hide Proofs
%     \ifbool{showProofsInDraft}{% If true, show proofs
%     }{% If false, hide proofs
%         \renewenvironment{proof}{\emph{Proof.} \workingnote{[Proof omitted from draft.]}\commentsection}{\endcommentsection\par\medskip}%
%     }
%     % % Hide Examples
%     % \renewenvironment{example}{\textbf{Example.} \workingnote{[Example omitted from draft.]}\commentsection}{\endcommentsection\par\medskip}
%     % Hide paragraph headings.
%     \providecommand{\paragraph}[1]{\workingnote{\textbf{#1} }}
% }{
%     \providecommand{\paragraph}[1]{}
% }


%%%%%%%%%%%%%%%%%%%%%%%%%%%%%%%%%%%%%%%%%%%%%%%%%%%%%%%%%%%%
%%%%%%%%%%%%% CHANGES PACKAGE ANNOTATIONS %%%%%%%%%%%%%%%%%%
%%%%%%%%%%%%%%%%%%%%%%%%%%%%%%%%%%%%%%%%%%%%%%%%%%%%%%%%%%%%


% Insert definitions used in your document here. 

\usepackage{demo_setup} % TODO: Delete this line.

% Use the presentation theme defined in packages/beamerthemepwintz_slides.sty.
\usetheme{pwintz_slides}


% Set the space between paragraphs, with flexible glue.
\setlength{\parskip}{7pt plus2pt minus4pt}
\newcommand{\setparskip}{\setlength{\parskip}{7pt plus2pt minus4pt}}
\AtBeginEnvironment{column}{\setlength{\parskip}{7pt plus2pt minus4pt}}

\graphicspath{{images/}{../images/}}

\newcommand{\mycolorbox}[1]{
  \begin{tcolorbox}[%
      colback=white,
      colframe=structure
    ]
    #1
  \end{tcolorbox}
}

%%%%%%%%%%%%%%%%%%%%%%%%%%%%%%%%%%
%%%%%%%%%%%% MACROS %%%%%%%%%%%%%%
%%%%%%%%%%%%%%%%%%%%%%%%%%%%%%%%%%

\renewenvironment{remark}{\textit{\structure{Remark.}}}{}
\newcommand{\mstructure}[1]{{\color{structure} #1}}
\newcommand{\shortfootcite}[2][1-]{\footnote<#1>{\citet{#2}.}}
\let\footnoterule\relax % Hide line above footnotes.

\usepackage{universal_frontmatter_loader}

\newcommand{\itemnote}[1]{\note[item]<.->{#1}}

\usepackage{qrcode}

\begin{document}


\begin{frame}[t]{Slide With Top-aligned Text}
  \lipsum[1][1-3]
  \begin{itemize}
    \item \lipsum[2][1]
    \item \lipsum[2][2]
    \item \lipsum[2][3]
    \item \lipsum[2][4]
    \item \lipsum[2][5]
  \end{itemize}
\end{frame}
  
\begin{frame}{Slide with Center-Aligned Text}
  \lipsum[2]
\end{frame}
  
\begin{frame}[t]{Slide with Columns}

  \begin{minipage}[t][\textheight][t]{\linewidth}
    \begin{columns}
      \begin{column}[T]{0.48\textwidth}
        \lipsum[3][1-10]
      \end{column}
      \hfill
      \begin{column}[T]{0.48\textwidth}
        \lipsum[4][1]
        \begin{center}
          \includegraphics[width=0.6\linewidth]{example-image-a}
        \end{center}
        \lipsum[4][2-5]
      \end{column}
    \end{columns}
  \end{minipage}

\end{frame}


\begin{frame}[plain]
  \vspace{0.4in}
  \begin{center}
    \vfill
    % \only<0| handout:1>{
      {\Huge\structure{Questions?}}
    % }
    % \only<1| handout:0>{
    %   \animategraphics[autoplay,width=0.8\linewidth]{30}{animation/QuestionsScene}{0000}{0052}
    % }
    \vfill
  %   \begin{description}
  %     \item[NSF] CNS-2039054 and CNS-2111688; 
  %     \item[AFOSR] FA9550-19-1-0169, FA9550-20-1-0238, FA9550-23-1-0145, and FA9550-23-1-0313; 
  %     \item[AFRL] FA8651-22-1-0017 and FA8651-23-1-0004; 
  %     \item[ARO] W911NF-20-1-0253; 
  %     \item[DOD] W911NF-23-1-0158.
  %   \end{description}
  \end{center}

  

  \begin{columns}
    \hfill
    \begin{column}[T]{0.48\textwidth}
      \centering
      \structure{\textbf{Slides and paper available at \url{paulwintz.com/publications}.}}
        \begin{center}
          \qrset{height=1.4in,nolink}%
          \qrcode{paulwintz.com/publications/wintz-forward-2023}
        \end{center}
    \end{column}
    \hfill
    \begin{column}[T]{0.48\textwidth}
      \centering
      \structure{\textbf{Funding}}
      \medskip

      \begin{minipage}{\linewidth}%
        \centering
        \hfill
        \includegraphics[height=0.30\linewidth,align=c]{nsf_logo.png}% The 'align=c' argument needs '\usepackage{graphbox}' 
        \hfill
        \includegraphics[height=0.10\linewidth,align=c]{AFRL_Logo.png} 
        \hfill
      \end{minipage}%

      \begin{minipage}{\linewidth}%
        \centering
        \hfill
        \includegraphics[height=0.30\linewidth,align=c]{AFOSR_logo.png} 
        \hfill
        \includegraphics[height=0.16\linewidth,align=c]{DEVCOM_ARL_LOGO.png} % The Army Research Laboratory includes the Army Research Office.
        \hfill
      \end{minipage}%
    \end{column}
    \hfill
  \end{columns}
  
\end{frame}

\end{document}
