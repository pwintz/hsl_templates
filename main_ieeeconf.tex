\documentclass[letterpaper, 10pt, conference, final]{ieeeconf}

\pdfminorversion=4

% % Change the search path to include "./packages/".
\makeatletter
\def\input@path{{./packages/}}
\makeatother

\usepackage{pwintz_configuration}
\usepackage{pwintz_definitions}
\usepackage{demo_setup} % TODO: Delete this.

%%%
%%% STEP 1
%%%
%%% Use the "hslTR" package. Make sure you have the
%%% "hslTR.sty" file in the packages folder or your current directory.
%%%
%%%  HSL Technical Report PACKAGE
\usepackage{hslTR}

%%%%%%%%%%%%%%%%%%%%%%%%%%%%%%%%%%%%%%%%%%%%%%%%%%%%%%%%%%%%%%%%%%%%%%
%%% VERSION CONTROL COMMAND
%%% For Conference, Change to "true"
%%% For Report, Change to "false"
\newbool{conf}
\setbool{conf}{true}
\ifdraft{
  %% Show conference-only text in green.
  % \newcommand{\confonly}[1]{\darkgreen{#1}}
  % \newcommand{\reportonly}[1]{}
  \newcommand{\confonly}[1]{\darkgreen{#1}}
  \newenvironment{confonlyblock}{\color{darkgreen}}{\color{black}}

  % Show report-only content (which will be hidden in the final version) in red.
  \newcommand{\reportonly}[1]{\red{#1}}
  \newenvironment{reportonlyblock}{\color{darkred}}{\color{black}}

  % % Alternatively, use the following lines to hide the report-only content in drafts.
  % \newcommand{\reportonly}[1]{\red{#1}}
  % \newenvironment{reportonlyblock}{\commentsection}{\endcommentsection}
}{
  % Show conference-only content.
  \newcommand{\confonly}[1]{#1}

  % Hide report-only content.
  \newenvironment{confonlyblock}{}{}

  % Hide report-only content.
  \newcommand{\reportonly}[1]{}
  \newenvironment{reportonlyblock}{\commentsection}{\endcommentsection}
}

% Insert definitions used in your document here. 

%%%%%%%%%%%%%%%%%%%%%%%%%%%%%%%%%%%%%%%%%%%%%%%%%%%%%%%%%%%%%%%%%%%%%%

%%%%%%%%%%%%%%%%%%%%%%%%%%%%%%%%%%%%%%%%%%%%%%%%%%%%%%%%%%%%%%%%%%%%%%
% For speed compilation purposes (process figures when needed)
%\includeonly{} % comment this if you want to process the figures 
% (or select which tex file you want to process - one per figure)
% \let\clearpage\relax 
%%%%%%%%%%%%%%%%%%%%%%%%%%%%%%%%%%%%%%%%%%%%%%%%%%%%%%%%%%%%%%%%%%%%%% 

% Create \ifdraft{}{} conditional 
% that switches based on whether "draft" 
% is passed to document class.
\usepackage{ifdraft} 
\ifdraft{
    % Adjust spacing to fit margin notes.
    \usepackage[
            inner=0.75in,
            outer=0.75in,
            marginparwidth=0.6in]{geometry} 
}{}

\begin{document}

% Populate poster title from value defined in document_setup.tex.
\makeatletter
\title{\pwintz@title}
\makeatother

\author{% 
% Populate author information from definitions in document_setup.tex
   % Insert first author, if defined.
    \authorblockN{\csname pwintz@author1\endcsname}%
    % Insert second author, if defined.
    \ifcsdef{pwintz@author2}{% If defined
        \and\authorblockN{\csname pwintz@author2\endcsname}%
    }{}%
    % Insert third author, if defined.
    \ifcsdef{pwintz@author3}{% If defined
        \and\authorblockN{\csname pwintz@author3\endcsname}%
    }{}%
    % Insert fourth author, if defined.
    \ifcsdef{pwintz@author4}{% If defined
        \and\authorblockN{\csname pwintz@author4\endcsname}%
    }{}
    \thanks{
        \csname pwintz@author1\endcsname{} is with
        \csname pwintz@longAffiliation1\endcsname{}
        (\insertemailfromcs{pwintz@authorEmail1})%
        %
        \ifcsdef{pwintz@author2}{% If defined
          ; \csname pwintz@author2\endcsname{} is with
          \csname pwintz@longAffiliation2\endcsname{}
          (\insertemailfromcs{pwintz@authorEmail2})%
        }{}%
        %
        \ifcsdef{pwintz@author3}{% If defined
          ; \csname pwintz@author3\endcsname{} is with 
          \csname pwintz@longAffiliation3\endcsname{}
          (\insertemailfromcs{pwintz@authorEmail3})%
        }{}%
        %
        \ifcsdef{pwintz@author4}{% If defined
          ; \csname pwintz@author4\endcsname{} is with 
          \csname pwintz@longAffiliation4\endcsname{}
          (\insertemailfromcs{pwintz@authorEmail4})%
        }{}.
    }
    \thanks{
        \acknowledgementBlurb{}
    }
}

\maketitle

% Uncomment to compile.
% \update{Significant updates since the last draft are shown in blue.}
% \confonly{Sections included only in the conference version are formatted like this.}
% \reportonly{Sections included only in the report version are formatted like this.}

\begin{abstract}%
    \lipsum[1]
\end{abstract}
 
% \begin{keywords}
%     Switching, hybrid system, stability \todo[inline]{Remove from PDF before submission.}
% \end{keywords}

\section{Introduction}
\label{sec:intro}
\lipsum[2]

\workingnote{This is a working note}
% \begin{workingnotes}
% This comment only appear in \texttt{draft} mode.
% \end{workingnotes}
Here is \added{added}, \deleted{deleted} and \replaced{replaced}{replaysed} text.

\lipsum[3]

%%%%%%%%%%%%%%%%%%%%%%%%%%%%%%%%%%%%%%%%%%%%%%%
\section{Preliminaries}
%%%%%%%%%%%%%%%%%%%%%%%%%%%%%%%%%%%%%%%%%%%%%%%
\label{sec:preliminaries}

\lipsum[4-6]

\subsection{Hybrid Systems}
% Hybrid System
We consider hybrid systems modeled in the form \cite{goebel_hybrid_2012}
\begin{equation}
    \calH{:}\: \left\{\begin{aligned}
        \xdot \hspace{4pt}&= f(x) & x \in C \\
        x^+ &= g(x) & x \in D
    \end{aligned}\right. 
    \label{eq:hybrid system}
\end{equation}
with state variable $x\in \realsn$, 
flow map $f : C \to \realsn$, 
jump map $g : D \to \realsn$, 
flow set $C \subset \realsn,$ and
jump set $D \subset \realsn$.
See \cref{fig:this is a figure}.

\begin{figure}[htbp]
    \centering
    \includegraphics[width=0.5\linewidth]{example-image-a}
    \caption{This is my caption}
    \label{fig:this is a figure}
\end{figure}

\lipsum[7]

%%%%%%%%%%%%%%%%%%%%%%%%%%%%%%%%%%%%%%%%%%%%%%%
\section{[Section Heading]}
%%%%%%%%%%%%%%%%%%%%%%%%%%%%%%%%%%%%%%%%%%%%%%%
% \label{sec:<insert label>} % TODO: Insert label

\lipsum[9]
\begin{equation}
    \label{eq:my equation}
    \dot z = f(z,u) \quad z \in \reals^n.
\end{equation}
The following is clear from \cref{eq:my equation}:
\lipsum[10-11]

\section{Conclusion}
\label{sec:conclusion}
\lipsum[12]

%%% BIBLIOGRAPHY

% TODO: Remove the following fake citations.
\nocite{knuth:ct:a}
\nocite{glashow}
\nocite{aristotle:physics}
\nocite{moore}
\nocite{salam}
\nocite{ctan}
\nocite{loh}
\printbibliography

\appendix
\section{Appendix}
\label{sec:appendix}

\lipsum[1]

\end{document}

