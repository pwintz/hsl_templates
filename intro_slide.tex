%%%%%%%%%%%%%%%%%%%%%%%%%%%%%%%%%%%%%%%%%
% Jacobs Landscape Poster
% LaTeX Template
% Version 1.1 (14/06/14)
%
% Created by:
% Computational Physics and Biophysics Group, Jacobs University
% https://teamwork.jacobs-university.de:8443/confluence/display/CoPandBiG/LaTeX+Poster
% 
% Further modified by:
% Nathaniel Johnston (nathaniel@njohnston.ca)
%
% This template has been downloaded from:
% http://www.LaTeXTemplates.com
%
% License:
% CC BY-NC-SA 3.0 (http://creativecommons.org/licenses/by-nc-sa/3.0/)
%
%%%%%%%%%%%%%%%%%%%%%%%%%%%%%%%%%%%%%%%%%

%------------------------------------------------------------------
%	PACKAGES AND OTHER DOCUMENT CONFIGURATIONS
%------------------------------------------------------------------

\documentclass[final]{beamer}

\usepackage{text/pwintz_configuration}
\usepackage{text/pwintz_definitions}
% Enter the title of your work.
\setTitle{This is my title}

%  ==== Define the authors for this project ==== 
% Each author is defined using 
%    \setAuthorX{Y}{Z}
% where 
%  - X is a unique author-index
%  - Y is the author's name
%  - Z is the institution-index for the author's institution.
% Up to four authors are supported, currently.

% Author 1 info (required)
\setAuthor1{Paul V.\ Smith}{1}
\setAuthorEmail1{psmith@ucsc.edu}
\setLongAffiliation1{the Department of Applied Mathematics, University of California, Santa Cruz, CA 95064}

% Author 2 info (optional)
\setAuthor2{Small D.\ Smith}{1}
\setAuthorEmail2{ssmith@ucsc.edu}
\setLongAffiliation2{the Department of Unapplied Mathematics, University of California, Santa Cruz, CA 95064}

% Author 3 info (optional)
\setAuthor3{Tall C.\ Smith}{1}
\setAuthorEmail3{tsmith@ucsc.edu}
\setLongAffiliation3{the Department of Reapplied Mathematics, University of California, Santa Cruz, CA 95064}

% Author 4 info (optional)
\setAuthor4{Pterodactyl Z.\ Smith}{2}
\setAuthorEmail4{psmith@pangaea.edu}
\setLongAffiliation4{the Department of Prehistoric Mathematics, University of California, Santa Cruz, CA 95064}

% Define institutions using 
%    \setInstituteX{Y}
%  - X is a unique institution-index
%  - Y is the institution's name
% Up to four institutes are supported, currently
\setInstitute1{University of California, Santa Cruz}
\setInstitute2{University of Pangaea}

% Enter acknowledgements, such as the sources of funding for your project.
% - The abbreviation of numbers (plural) is "nos." 
% - When writing a period that does not end sentence (e.g., after "nos."), follow it with "\ " to create a space of the correct width. 
\setAcknowledgementBlurb{

    This research was supported by the National Science Foundation under grant nos.\ ECS-1710621, CNS-1544396, and CNS-2039054; by the Air Force Office of Scientific Research under grant nos.\ FA9550-19-1-0053, FA9550-19-1-0169, and FA9550-20-1-0238; by the Army Research Office under grant no.\ W911NF-20-1-0253; and by the U.S. Office of Naval Research under the MURI grant no.\ N00014-16-1-2710.
}

% Insert definitions used in your document here. 
\usepackage{demo_setup} % TODO: Delete this.

\graphicspath{{images/}{../images/}}

\newcommand{\mysubsection}[1]{\vspace{20pt plus 5pt}{\color{structure} \textbf{#1}}\par}

\usepackage[scale=1.8]{beamerposter} % Use the beamerposter package for laying out the poster

% The graphbox package adds an "align" option to the \includegraphics command that makes this very easy to certically align images. https://tex.stackexchange.com/a/350374/153678
\usepackage{graphbox}

\usepackage{paracol}

\usetheme{confposter} % Use the confposter theme supplied with this template

%-----------------------------------------------------------
% Define the column widths and overall poster size
% To set effective sepwid, onecolwid and twocolwid values, first choose how many columns you want and how much separation you want between columns
% In this template, the separation width chosen is 0.024 of the paper width and a 4-column layout
% onecolwid should therefore be (1-(# of columns+1)*sepwid)/# of columns e.g. (1-(4+1)*0.024)/4 = 0.22
% Set twocolwid to be (2*onecolwid)+sepwid = 0.464
% Set threecolwid to be (3*onecolwid)+2*sepwid = 0.708

\newlength{\sepwid}
\newlength{\onecolwid}
\newlength{\twocolwid}
\newlength{\threecolwid}
\setlength{\paperwidth}{48in} % A0 width: 46.8in
\setlength{\paperheight}{27in} % A0 height: 33.1in
% Separation width between columns
\setlength{\sepwid}{0.01\paperwidth} 
% Width of one column=(1-(# of columns+1)*sepwid)/# of columns=(1-4*0.01)/3
\setlength{\onecolwid}{0.32\paperwidth} 
% Width of two columns=(2*onecolwid)+sepwid=(2*0.32)+0.01
\setlength{\twocolwid}{0.65\paperwidth} 
% Width of three columns
\setlength{\threecolwid}{0.708\paperwidth} 
\setlength{\topmargin}{-0.5in} % Reduce the top margin size
%-----------------------------------------------------------

\usepackage{graphicx}  % Required for including images

\usepackage{booktabs} % Top and bottom rules for tables

%------------------------------------------------------------------
%	TITLE SECTION 
%------------------------------------------------------------------

\title{(Use header in PPT presentation)} % Poster title
\author{\ } % Author(s)
\institute{\ } % Institution(s)

%------------------------------------------------------------------

\setlength{\parskip}{40.0 pt plus 5 pt}

\begin{document}

\addtobeamertemplate{block end}{}{\vspace*{2ex}} % White space under blocks
\addtobeamertemplate{block alerted end}{}{\vspace*{2ex}} % White space under highlighted (alert) blocks

\setlength{\belowcaptionskip}{2ex} % White space under figures
\setlength\belowdisplayshortskip{2ex} % White space under equations

\begin{frame}[t] % The whole poster is enclosed in one beamer frame

\begin{columns}[t] % The whole poster consists of three major columns, the second of which is split into two columns twice - the [t] option aligns each column's content to the top

\hspace{\sepwid}
\begin{column}{0.85\onecolwid} % The first column

\begin{block}{Problem Setting}
    \lipsum[4][1-2]
    $$\xdot = f(x)$$
    \lipsum[4][3]
    \begin{enumerate}
        \item \lipsum[4][5]
        \item \lipsum[4][6]
        \item \lipsum[4][7]
        \item \lipsum[4][8]
    \end{enumerate}
    
    \lipsum[4][9-10]
\end{block}

\end{column} % End of the first column
\hspace{\sepwid}
\begin{column}{\onecolwid} % The first column

    \vfill
    \begin{center}
        \includegraphics[width=0.5\linewidth]{example-image-a}
    \end{center}
    \vfill

    \begin{block}{[Section Heading]}
        \lipsum[2][1]
        \begin{itemize}
            \item \lipsum[2][2]
            \item \lipsum[2][3]
            \item \lipsum[2][4]
        \end{itemize}
    \end{block}

\end{column} % End of the third column
\hspace{\sepwid}
\begin{column}{\onecolwid} % The first column
    
    \begin{block}{Example}
        % Take \qcolor{1}{$\kappa_1$} to be a model predictive controller that occasionally fails to compute an update.
        % \vspace{-20pt}
        \lipsum[3][1-2]
        \begin{center}
            \includegraphics[align=c,width=0.49\linewidth]{example-image-b}
            % \vspace{1em}
            \includegraphics[align=c,width=0.49\linewidth]{example-image-c}
            \vfill
        \end{center}
    \end{block}

    \begin{block}{Acknowledgements}
        \begin{center}
            \begin{minipage}{0.9\linewidth}
                % Adjust the height of each icon to make their relative size look right. The option "align=c" ensures that they are all vertically centered
                \includegraphics[height=0.16\linewidth,align=c]{nsf_logo}\hfill
                \includegraphics[height=0.17\linewidth,align=c]{AFOSR_logo.png}\hfill
                \includegraphics[height=0.11\linewidth,align=c]{DEVCOM_ARL_LOGO.png}\hfill 
                \includegraphics[height=0.13\linewidth,align=c]{ONR_logo.png} 
            \end{minipage}
        \end{center}
    \end{block}
\end{column}
\end{columns} % End of all the columns in the poster

\end{frame} % End of the enclosing frame

\end{document}
