% \update{Significant updates since the last draft are shown in blue.}
% \confonly{Sections included only in the conference version are formatted like this.}
% \reportonly{Sections included only in the report version are formatted like this.}

\begin{abstract}%
    A hybrid control strategy is introduced
    that renders a compact set uniformly globally 
    asymptotically stable for a continuous-time plant
    by switching between a Lyapunov-certified feedback controller 
    and an uncertified controller.
    This control strategy
    allows for the opportunistic use of a controller that 
    has desirable performance but lacks a Lyapunov certificate.
    A pair of tunable threshold functions determine 
    conditions for switching between the controllers.
    To establish global uniform asymptotic stability, 
    a nonsmooth Lyapunov function is constructed for the 
    closed-loop hybrid system using 
    an auxiliary memory variable
    and the Lyapunov certificate 
    associated with the certified controller.
    Examples illustrate improvements 
    to control effort and rate of convergence 
    resulting from the proposed hybrid control strategy 
    when applied to state-feedback 
    and model-predictive control.
\end{abstract}
 
% \begin{keywords}
%     Switching, hybrid system, stability \todo[inline]{Remove from PDF before submission.}
% \end{keywords}

\section{Introduction}
\label{sec:intro}
For some control design problems, 
a single continuous state-feedback controller 
cannot simultaneously satisfy all design requirements. 
In particular, the design of a globally asymptotically 
stabilizing controller for a nonlinear system
requires multiple controllers 
if the system violates Brockett's 
conditions \cite{brockett_asymptotic_1983} or
has certain topological obstructions,   
such as systems with states that live 
in certain topological manifolds~\cite{mayhew_quaternion-based_2011,sanfelice_robust_2006}.
Such challenges have motivated the use 
of supervisory algorithms that
selects between multiple 
controllers~\cite{liberzon_switching_2012,battistelli_supervisory_2012,hespanha_logic-based_1998,hespanha_supervision_1996}.
For a system with one controller that renders %TODO: This sentence could be improved.
a set-point only locally asymptotically stable, 
and another controller that guides
the system into the vicinity of the set-point, 
a class of supervisors called \emph{uniting controllers}
produce global asymptotic stability of the set-point 
by selecting the first controller near the set-point 
and the second controller away from it
\cite{prieur_uniting_2001,sanfelice_hybrid_2021,teel_uniting_1997}.
Similarly, a switching strategy for a family of 
Lyapunov-certified controllers to achieve asymptotic stability 
is presented in \cite{el-farra_output_2005}.
For systems with constraints,
supervisors are used to
provide a backup controller that guarantees 
safety when the primary controller 
lacks such a guarantee \cite{seto_simplex_1998}.
We do not, however, know of a 
control strategy that allows for
the opportunistic use of an uncertified controller 
to improve performance while preserving asymptotic stability. 

In this paper, a hybrid control strategy 
is proposed to fill that gap by uniting a 
Lyapunov-certified controller with an uncertified controller
via opportunistic switching.
We consider an unconstrained 
continuous-time nonlinear plant
with state space $\reals^\nplant$ and a given compact set 
$\calA\subset\reals^\nplant$ that must be rendered
uniformly globally asymptotically stable (UGAS).
If, for a given controller, $\calA$ is rendered UGAS 
and a Lyapunov function is known for the closed-loop system, 
then we call the controller \emph{Lyapunov-certified}.
A Lyapunov function exists for every 
sufficiently regular
closed-loop system such that $\calA$ is 
UGAS \cite[Theorem 4.17]{khalil_nonlinear_2014},
but construction of such a function 
is often difficult. 
Some controllers that do not cause $\calA$ to be UGAS may, however, 
have otherwise desirable properties.
A controller for 
which a Lyapunov function is unavailable
is called \emph{uncertified}.
The novel contribution of this paper is the introduction of
a hybrid control strategy, such that---%
given a continuous Lyapunov-certified 
feedback controller $\kappa_0$
and a continuous uncertified controller $\kappa_1$---%
the set $\calA$ is UGAS for the
resulting closed-loop system, 
the controller $\kappa_1$ is preferred over $\kappa_0,$
% \footnote{ % ADD "%" on previous line if this is restored.
%     The controller $\kappa_1$ can be thought of as the primary
%     controller and $\kappa_0$ as a fallback, but we do not use 
%     this terminology.
% }
and Zeno behavior does not occur.

As an example where using an uncertified controller is advantageous, 
suppose $\kappa_1$ is a linear quadratic regulator (LQR) 
for the linearization of a nonlinear system about the origin. 
Because an LQR feedback is an optimal control law, 
$\kappa_1$ is approximately optimal \added{(by some measure)} near the origin. 
The basin of attraction under $\kappa_1$ 
is an open neighborhood of the origin, but
far from the origin, nonlinear dynamics dominate, 
so the linearization is inaccurate and
$\kappa_1$ will generally not produce global stability.
Our switching logic lets us use $\kappa_1$---without knowledge 
of the actual basin of attraction---in conjunction with a Lyapunov-certified
controller to achieve global convergence to the origin and minimize costs locally.
A detailed consideration of this example is given 
in \cref{ex:lqr}.
We envision
that our switching logic could be particularly useful % "We envision.." per Ricardo
for reinforcement learning control,
which often demonstrates good results empirically, 
but for which it is often difficult to produce Lyapunov 
certificates \cite{dai_lyapunov-stable_2021}.

The remainder of the paper proceeds as follows. 
\Cref{sec:preliminaries} introduces notation and preliminary concepts.
\Cref{sec:switching_logic} describes our \added{proposed} switching logic
and the resulting closed-loop system.
\Cref{sec:analysis} contains theoretical results.
Several examples, throughout, illustrate
the behavior of the closed-loop system.

%%%%%%%%%%%%%%%%%%%%%%%%%%%%%%%%%%%%%%%%%%%%%%%
\section{Preliminaries}
%%%%%%%%%%%%%%%%%%%%%%%%%%%%%%%%%%%%%%%%%%%%%%%
\label{sec:preliminaries}

% \subsection{Notation}
We denote
the nonnegative real numbers by $\nonnegativereals$, 
and the natural numbers 
($0$ inclusive) by $\naturals.$ 
\reportonly{The zero vector in $\reals^n$ is written $\zerovec{n}.$}
For $x, y \in \reals^n,$ $\ip{x}{y}$ denotes the 
inner product between $x$ and $y.$ 
We write $[x\trans \: y\trans]\trans$ as $(x, y)$. 
For a set $S,$ the interior of $S$ is denoted 
$\interior S$. % and the boundary is denoted $\bnd S.$
% For a function $f: \reals^m \to \realsn$ and $D\subset \reals^m,$
% $f(D) := \setdef{f(x) \suchthat x \in D}.$
For a continuously differentiable function $f : \realsn \to \reals,$ 
the gradient of $f$ at $x$ is denoted $\grad f(x).$
Given $x\in \realsn$ and a nonempty set $\calA \subset \realsn,$ 
the distance from $x$ to $\calA$ is
$\distA{x} := \inf_{y\in \calA}\abs{y - x}.$
% Let $S \in \reals^n$ and take any point $x\in S.$ 
% \deleted{
% The \emph{tangent cone} to $S$ at $x,$ denoted $\tangentcone{S}(x),$ 
% is the set of 
% all vectors $v\in \reals^n$ such that there exists a sequence of 
% real numbers $\alpha_{i} \searrow 0$ and sequence of vectors $ v_i \to v$ 
% that satisfy $x + \alpha_i v_i \in S$ for all $i \in \naturals.$
% }
A continuous function $\alpha : \nnreals \to \nnreals$ 
is said to be in class $\Kinfty$ if $\alpha(0) = 0,$
$\alpha$ is strictly increasing, and $\lim_{r \to \infty} \alpha(r) = \infty.$ 
Given a nonempty set $\calA \subset \realsn,$ 
a function $V : \realsn \to \nnreals$ is said to be 
\emph{positive definite} with respect
to $\calA$ if $V(x) > 0$ for all 
$x \in \realsn \setminus \calA$ 
and $V(x) = 0$ for all $x\in\calA$. %$V(\calA) = \{0\}.$

\reportonly{
    For nonsmooth functions, we use the Clarke generalized 
    gradient and Clarke generalized directional derivative described in 
    \cite{clarke_optimization_1990}.
    For a locally Lipschitz function $V : \reals^n \to \reals,$ 
    the \emph{Clarke generalized gradient} of $V$ at $x \in \realsn,$
    denoted $\ggrad V(x),$ is the convex hull of all limits of sequences
    $\grad V(x_i)$ where $x_i$ is a sequence converging to $x$ 
    while avoiding the set where $V$ is not differentiable 
    (due to Lipschitz continuity, $V$ is differentiable almost everywhere).
    The \emph{Clarke generalized directional derivative} of $V$ at $x$ in the 
    direction $w$ is given by 
    $\gdd{V}(x, w) = \max_{\zeta \in \ggrad V(x)}\ip{\zeta}{w}.$
    
}

\subsection{Hybrid Systems}
% Hybrid System
We consider hybrid systems modeled in the form \cite{goebel_hybrid_2012,sanfelice_hybrid_2021}
\begin{equation}
    \calH \left\{\begin{aligned}
        \xdot \hspace{4pt}&= f(x) & x \in C \\
        x^+ &= g(x) & x \in D
    \end{aligned}\right. 
    \label{eq:hybrid system}
\end{equation}
with state variable $x\in \realsn$, 
flow map $f : \added{C} \to \realsn$, 
jump map $g : \added{D} \to \realsn$, 
flow set $C \subset \realsn,$ and
jump set $D \subset \realsn$.
A solution $x$ to $\calH$ is defined on a hybrid time 
domain $\dom x \subset \nnreals \times \naturals,$ 
which parameterizes the solution by ordinary time 
$t \in \nnreals$ and discrete time $j \in \naturals.$
A hybrid time domain is a subset of $\nnreals \times \naturals$
such that for every $(T, J) \in \dom x,$ 
there exists a sequence \(\left\{t_{j}\right\}_{j=0}^{J+1}\) such that 
\(t_{0}= 0,\) \(t_{j+1} \geq t_{j}\) 
for each \(j \in\{0,1, \ldots, J\},\) and
$\dom x \cap ([0, T] \times \{0, 1, \dots, J\})
=\cup_{j=0}^{J}\left(\left[t_{j}, t_{j+1}\right], j\right);$ 
see \cite{goebel_hybrid_2012}.
A solution $x$ is said to be \emph{complete} if $\dom x$ is unbounded, 
and is said to be \emph{Zeno} if it is complete 
and the $t$ component of $\dom x$ is bounded 
(implying $j \to \infty$ in finite ordinary time). 
A solution $x$ is said to be \emph{maximal} if there does not exist a solution $y$
to $\calH$ such that $x$ is a truncation of $y$ to a strict subset of $\dom y.$   
% "A solution \(x\) to \(\calH\) is defined on a hybrid time domain denoted 
% \(\dom x \subset \nnreals \times \mathbb{N} .\) 
% The solution \(x\) is parametrized by the ordinary time variable \(t \in \nnreals\) 
% and the discrete jump variable \(j \in \mathbb{N}\).
% Its domain of definition dom \(x\) is such that for each \((T, J) \in \dom x\), 
% \(\dom x \cap([0, T] \times\{0,1, \ldots, J\})
% =\cup_{j=0}^{J}\left(\left[t_{j}, t_{j+1}\right], j\right)\)
% for a sequence \(\left\{t_{j}\right\}_{j=0}^{J+1}\), such that \(t_{j+1} \geq t_{j}\) 
% for each \(j \in\{0,1, \ldots, J\}\) and \(t_{0}=0 ;\) see \([31]\)
% A solution \(x\) to \(\calH\), as defined in [30, Definition 12], 
% starting from \(x_{o}\) is said to be complete if it is defined on an unbounded hybrid time domain; 
% that is, the set \(\dom x\) is unbounded. 
% Furthermore, it is said to be maximal if there is no solution \(y\) to \(\calH\) 
% such that \(x(t, j)=y(t, j)\) for all \((t, j) \in \dom x\) 
% with dom \(x\) a proper subset of \(\dom y\)." From [236].
\reportonly{
A hybrid system is called \emph{well-posed} when it is robust
to certain types of perturbations; 
see \cite{sanfelice_hybrid_2021}. 
The following conditions are sufficient 
for a system in the form \cref{eq:hybrid system} to be well-posed.
\begin{assumption}[Hybrid Basic Conditions] Given a hybrid system $\calH$ as in \cref{eq:hybrid system}, 
    its data $(f, g, C, D)$ satisfies the following properties:
    \setupAssumption[A]
    \begin{enumerate}
        \label{assump:hbc}
        \item $C$ and $D$ are closed sets;
        \item $f$ is a continuous function on $C$; and
        \item $g$ is a continuous function on $D$.
    \end{enumerate}
\end{assumption}
}

% COMPARISON FUNCTIONS
\subsection{Stability Properties}

% Lyapunov function
Given a differential equation $\dot z = f(z)$ with $f : \realsn \to \realsn$ continuous 
and $z$ evolving in $\realsn,$ and a nonempty compact set $\calA \subset \reals^n,$
then a continuously differentiable function 
$V : \realsn \to \reals$ is called a \emph{Lyapunov function}
if there exist $\alpha_1, \alpha_2 \in \Kinfty$ 
and a continuous positive definite function $\rho$ such that
\begin{align*}
    \alpha_1(\distA{z}) \leq V(z) &\leq \alpha_2(\distA{z} ) &\forall z \in \realsn, \\ 
    \ip{\nabla V(z)}{f(z)} &\leq -\rho(\distA{z}) &\forall z \in \realsn.
\end{align*}

% TYPES OF STABILITY (globally uniformly stable and globally uniformly asymptotically stable)
% We use the following types of stability from :
\begin{definition}[{\cite[Definition 3.7]{sanfelice_hybrid_2021}}]%[Uniform global stability, pre-attractivity, and pre-asymptotic stability]
    For a hybrid system \(\calH\) as in \cref{eq:hybrid system}, 
    a nonempty set \(\calA \subset \realsn\) is said to be
        \emph{uniformly globally stable} for \(\calH\) if there exists 
        a class-\(\Kinfty\) function \(\alpha\) 
        such that every solution \(x\) to \(\calH\) satisfies 
        \(\distA{x(t, j)} \leq \alpha\left(\distA{x(0,0)}\right)\) 
        for each \((t, j) \in \dom x\); and
        \emph{uniformly globally attractive} for \(\calH\) if 
        every maximal solution is complete and
        for all \(\varepsilon>0\) and \(r>0\), there exists \(T>0\) such that
        every solution \(x\) to \(\calH\) with \(|x(0,0)|_{\calA} \leq r\)
        satisfies $|x(t, j)|_{\calA} \leq \varepsilon$ for all $(t, j) \in \dom x$
        such that $t+j \geq T$. If $\calA$ is 
        both uniformly globally stable and uniformly globally attractive for $\calH$,
        then it is said to be
        \emph{uniformly globally asymptotically stable} (UGAS) for \(\calH\).
        % uniformly globally stable and uniformly globally attractive.
\end{definition} 

% Forward invariant.
Given a hybrid system \(\calH\) as in \cref{eq:hybrid system}, 
a nonempty set \(K \subset \realsn\) is said to be
\emph{forward invariant} for \(\calH\) if each maximal solution 
\(x\) to \(\calH\) from \(K\) is complete and satisfies 
\(x(t, j) \in K\) for all \((t, j) \in \dom x\)
\cite[Definition 3.13]{sanfelice_hybrid_2021}.

%%%%%%%%%%%%%%%%%%%%%%%%%%%%%%%%%%%%%%%%%%%%%%%
\section{Hybrid Control Strategy}
%%%%%%%%%%%%%%%%%%%%%%%%%%%%%%%%%%%%%%%%%%%%%%%
\label{sec:switching_logic}

We consider a nonlinear continuous-time plant 
\begin{equation}
    \dot z = \fp(z,u)
    \label{eq:plant}
\end{equation}
with state space $\plantstatespace.$ 
Let $\calA \subset \plantstatespace$ be a given nonempty
compact set to asymptotically stabilize.
Suppose $\kappa_0$ is a continuous
Lyapunov-certified controller that renders 
$\calA$ to be UGAS for $\dot z =
\fp(z, \kappa_0(z))$ and has an associated Lyapunov function $V$,
and suppose $\kappa_1$ is a 
continuous uncertified controller for \cref{eq:plant}.
We write the pair of feedback control laws as
$u = \kappa_q(z)$ with $q \in Q := \{0, 1\}.$
The problem to solve consists of designing 
a \correction{switching logic for $q$} such 
that $\calA$ is UGAS for the resulting
closed-loop system, 
Zeno behavior does not occur, 
and the controller $\kappa_1$ is preferred over $\kappa_0.$
To solve this problem, 
we design a hybrid control strategy that determines 
when to switch between $\kappa_0$ and $\kappa_1$, 
as shown in \cref{fig:feedback diagram}.
The resulting closed-loop system is hybrid, which we
model as in \cref{eq:hybrid system}.
\begin{figure}[htbp]
    \centering
    \includegraphics{feedback_diagram.pdf}
    \setlength{\belowcaptionskip}{-8pt}
    \caption{The switching logic passes $q$ as an output to 
    a switch, which determines whether 
    $\kappa_0$ or $\kappa_1$ is applied to the plant.}
    \label{fig:feedback diagram}
\end{figure}

%%%%%%%%%%%%%%%%%%%%%%%%%%%%%%%%%%%%%%%%%%%%%%%%    
        \subsection{Outline of Hybrid Control Strategy}        
%%%%%%%%%%%%%%%%%%%%%%%%%%%%%%%%%%%%%%%%%%%%%%%%
\label{sec:outline of strategy}

% Paragraph break per Ricardo (10/10/2021)
Our hybrid control strategy uses 
the plant state variable $z,$
the logic variable $q$ described above, and
a memory variable $v \in \nonnegativereals.$
The purpose of each variable is summarized here:
\begin{itemize}
    \item $z \in \reals^\nplant$ is the state of the plant. 
    Our goal is to steer $z$ asymptotically to $\calA.$
    \item $q \in Q$ determines the current feedback controller. 
    When $q = 0$, controller $\kappa_0$ is used
    and when $q = 1,$ $\kappa_1$ is used.
    \item $v \in \nnreals$ records the value of $V(z)$ at each switch, 
    and then decreases along flows, converging to zero 
    (the dynamics of $v$ are designed in 
    \cref{sec:construction of system}). 
    When using the $\kappa_1$ controller, 
    \added{$V(z)$} can increase because $\kappa_1$ is uncertified,
    so $v$ is used as an upper bound for $V\added{(z)}$, 
    restricting how much $V\added{(z)}$ can grow before triggering a
    switch to $q = 0$. Because $v$ converges to zero, 
    $V\added{(z)}$ will be squeezed to zero as well.
\end{itemize}
Hence, the state of the closed-loop system is
$$x := (z, v, q) \in \calX := \plantstatespace \times \nnreals \times Q,$$
and we aim to uniformly globally 
asymptotically stabilize the compact set  
\begin{equation}
    \label{eq:Ax}
    \calAX := \setdef{x \in \calX \mid z\in \calA, v = 0} 
            = \calA \times \{0\} \times Q.
\end{equation}
The rate \added{of change of} $V\added{(z)}$
is central to our discussion, so for each 
$z \in \plantstatespace$ and each $q\in Q$, we define 
$$\dot V_{q}(z) := \ip{\del V(z)}{\fp(z, \kappa_{q}(z))}.$$
Because $V$ is a Lyapunov function for $\dot z = f_P(z, \kappa_0(z)\added{)}$, 
there exists a continuous positive definite function $\rho$ such
that $\dot V_0(z) \leq -\rho(|z|_A)$ 
for all $z\in \reals^n$.

The basic idea of our hybrid control strategy is as follows. 
Our strategy implements a switching logic that 
uses two continuous functions 
$\sigmazero, \sigmaone : \nonnegativereals \to \nonnegativereals$
chosen such that
$\sigmaone$ is positive definite 
and $\sigmazero(s) > \sigmaone(s)$ for all $s \geq 0.$%
\footnote{The function $\sigmazero$ is strictly 
positive---not positive definite---because $\sigmazero(0) > \sigmaone(0) = 0.$} 
These functions define thresholds on
$\dot V_1\added{(z)}$ for switching between the 
feedback controllers $\kappa_0$ and $\kappa_1.$
{%
\setupAssumption[S]
\begin{enumerate}
    \addtocounter{enumi}{-1}
    \item While the feedback controller $\kappa_0$ is applied to 
    the plant, due to $q$ being equal to $0$, we monitor $\dot V_1(z).$
    We say that $\dot V_1$ is 
    ``small enough to switch to $q=1$'' at $\zzero \in\plantstatespace$~if
    \begin{equation}
        \label{eq:Zswitchtoone}
        \zzero\in \calZswitchtoone 
        := \{z \in \plantstatespace \mid \dot V_1(z) \leq -\sigmazero(\distA{z})\}.
    \end{equation} 
    If $\dot V_1(z)$ is small enough to switch to $q=1,$
    then $\kappa_1$ will produce convergence toward $\calA,$ 
    so the switching logic updates 
    $q$ from $0$ to $1$ and records the value of $V(z)$ in $v.$ 
    Conversely, we say that $\dot V_1$ is  ``large enough to hold $q=0$'' 
    at $\zzero \in\plantstatespace$ if
    \begin{equation}
        \label{eq:Zholdzero}
        \zzero\in \calZholdzero 
        := \{z \in \plantstatespace \mid \dot V_1(z) \geq -\sigmazero(\distA{z})\}.
    \end{equation} 
    The system is allowed to flow if $q=0$ and $z\in \calZholdzero.$
    \label{item:q=0 switching criterion}

    \item While the feedback controller $\kappa_1$ is applied, 
    due to $q$ being equal to $1$, 
    the values of $v$, $V(z),$ and $\dot V_1(z)$ are monitored.
    We say that $\dot V_1$ is ``large enough to switch to $q=0$'' 
    at $\zone \in \plantstatespace$ if
    \begin{equation}
        \zone\in \calZswitchtozero 
        := \{z \in \plantstatespace \mid \dot V_1(z) \geq -\sigmaone(\distA{z})\},
        \label{eq:Zswitchtozero}
    \end{equation} 
    and ``small enough to hold $q=1$'' if
    \begin{equation}
        \zone\in\calZholdone  
        := \{z \in \plantstatespace \mid \dot V_1(z) \leq -\sigmaone(\distA{z})\}.
        \label{eq:Zholdone}
    \end{equation}
    If $\dot V_1(z)$ is large enough to switch to $q=0,$
    then $\kappa_1$ is performing poorly. 
    Rather than switching immediately, however, we wait to switch 
    until ${V(z) \geq v}.$ This provides leeway
    in case $\kappa_1$ briefly causes a small increase to $V(z)$. 
    (The dynamics of $v$ are designed, below, 
    such that if $z$ remains in $\calZswitchtozero$ long enough, 
    then $V(z)$ will eventually equal $v$.) 
    While \added{$q=1$ and either $z \in \calZholdone$ or $V(z) < v$}, 
    \added{the system flows and} we continue to use~$\kappa_1$.%
    \label{item:q=1 switching criterion}
\end{enumerate}
}
\Cref{fig:switching regions} shows a representative plot of 
$-\sigma_0,$ $-\sigma_1,$ $\calZswitchtoone,$ and $\calZswitchtozero.$ 
Note that $\calZswitchtoone \subset \interior \calZholdone$,
which ensures that if $\dot V_1(\zzero)$ is small enough to switch to $q=1,$
then there is a neighborhood of 
$\zzero$ where $\dot V_1$ is small enough to hold $q = 1$.
Similarly, $\calZswitchtozero \subset \interior \calZholdzero,$ so
if $\dot V_1(\zone)$ is large enough to switch to $q = 0,$
then there is a neighborhood of
$\zone$ where $\dot V_1$ is large enough to hold $q = 0$.
Furthermore, 
$\calZholdzero \union \calZholdone = \reals^\nplant,$
so either holding $0$ or holding $1$ is possible everywhere. The 
sets $\calZswitchtozero$ and $\calZswitchtoone$ are closed and disjoint,
precluding Zeno solutions (see \cref{lem:no Zeno}).
\begin{figure}[ht]
    \centering
    \includegraphics{switching_regions.pdf}
    \setlength{\belowcaptionskip}{-6pt}
    \caption{A plot of $-\sigma_0(\distA{z}), -\sigma_1(\distA{z}),
    \calZswitchtoone,$ and $\calZswitchtozero.$}
    \label{fig:switching regions}
\end{figure}

% \todo[inline]{Place this somewhere:
% When $\dot V_1(z)$ is ``small enough to hold $q = 1$'', 
% then the gap between $v$ and $V$ increases,
% otherwise it shrinks until $V(z) = v,$ prompting a switch. 
% This limits how much $V$ can increase if $\dot V_1$ is positive
% and how long $V$ can languish if $\dot V_1$ is negative but 
% decreasing too slowly.
% }

Before we formulate the hybrid closed-loop system, 
    we demonstrate the switching logic
    with an example.
\begin{example}[Switching Logic]
    \label{ex:languishing}
    Consider the plant $\dot z = u$ 
    with $z, u\ {\in}\ \reals$, controllers $\kappa_0(z) := -z,$ 
    $\kappa_1(z) := -z^3,$ and pick 
    $\sigmaone(s) := s^2$ and
    $\sigmazero(s) := 1.5 s^2 + 10^{-3}$
    for all $s \geq 0.$
    \Cref{fig:languishing example comparison} shows
    plots of solutions to $\zdot = \kappa_0(z),$ 
    $\zdot = \kappa_1(z),$ and
    $\zdot = \kappa_q(z)$ with $q$ switching 
    according to our hybrid control strategy.%
    \footnote{Simulations are computed in \Matlab 
    with the \emph{HyEQ Toolbox} \cite{sanfelice_toolbox_2013}.} 
    Initially, the solution with the feedback $\kappa_1$ converges quickly 
    but slows as $z$ approaches zero.
    On the other hand, the solution with the feedback $\kappa_0$
    converges slowly far from the origin, 
    but accelerates relative to the solution with the feedback $\kappa_1$,
    becoming smaller than it at $t=1.7 \sec$.
    The switched solution uses 
    $\kappa_1$ far from the origin and 
    $\kappa_0$ near the origin, 
    producing overall faster convergence.
    \begin{figure}[htbp]
        \centering
        \includegraphics{languish_example_comparison}
        \setlength{\belowcaptionskip}{-10pt}
        \caption{Solutions from \cref{ex:languishing} 
        using $\kappa_0$ only, $\kappa_1$ only, 
        and opportunistic switching between
        $\kappa_0$ and $\kappa_1$.}
        \label{fig:languishing example comparison}
    \end{figure}

    Plots of $V$, $v$, $-\sigmazero$, $-\sigmaone$, $\dot V_1$, 
    and $q$ for a solution from initial values
    ${(z_0, v_0, q_0) = (2, \added{0}, 0)}$  
    are shown in \cref{fig:languishing example}.
    At $t = 0 \sec$, $\dot V_1(z_0) < -\sigmazero(\distA{z_0}),$ 
    so, 
    per \ref{item:q=0 switching criterion},
    the system immediately switches to $q=1.$
    As time progresses, $\dot V_1\added{(z)}$ increases until 
    it surpasses $-\sigmaone(\distA{z})$
    at $t = 0.4 \sec$.
    This indicates $V\added{(z)}$ is not decreasing fast enough \added{to hold $q=1$}, 
    but because $V(z)$ is less than $v,$
    the switch to $q = 0$ is delayed until $v$ equals $V(z)$ 
    at $t \approx 0.9 \sec,$
    as required in \ref{item:q=1 switching criterion}.%
    \begin{figure}[htbp]
        \centering
        \includegraphics{languish_example}
        \setlength{\belowcaptionskip}{-12pt}
        \caption{In \cref{ex:languishing},
        the controller $\kappa_1$ initially has good performance, 
        causing $V\added{(z)}$ to decrease quickly, but after 
        $\dot V_1\added{(z)}$ moves above $-\sigmaone\added{(\distA{z})}$,
        $v$ starts to catch up with $V\added{(z)}$ and a switch is 
        triggered when \added{$V(z) = v$}. 
        While $q = 0$, $v$ does not affect switching, so it is hidden from plots.}
        \label{fig:languishing example}
    \end{figure}
\end{example}

%%%%%%%%%%%%%%%%%%%%%%%%%%%%%%%%%%%%%%%%%%%%%%%%%%%%%%%%%%%%%%%
        \subsection{Construction of the Closed-Loop System}        
%%%%%%%%%%%%%%%%%%%%%%%%%%%%%%%%%%%%%%%%%%%%%%%%%%%%%%%%%%%%%%%
\label{sec:construction of system}

We are now equipped to define the hybrid closed-loop system.
Following \ref{item:q=0 switching criterion}, 
if $q = 0$, then the system 
\added{flows while $z$ is in $\calZholdzero$ and}
jumps when $z$ enters $\calZswitchtoone$.  
Thus, if $q=0$, jumps occur when $\added{x = (z, v, q)}$ belongs to 
\begin{align} 
    D_{0} 
    % :\!&= \{x \in \calX_0 \mid \dot{V}_1(z) 
    % \leq -\sigmazero(\distA{z}) \} \nonumber \\
    %= \{x\in \calX_0 \mid z\in\calZs witchtoone\} 
    &:= \calZswitchtoone \times \nnreals \times \{0\}
    \label{eq:D_0}
\end{align} 
and flows occur when $x$ belongs to
\begin{equation}
    C_{0} := \closure{\calX_0 \setminus D_0} 
    = \calZholdzero \times \nnreals \times \setdef{0} 
    \label{eq:C_0}
\end{equation}
where $\calX_0:= \plantstatespace \times \nnreals \times \setdef{0}.$ % No paragraph break (per Ricardo)
Similarly, following \ref{item:q=1 switching criterion},
when $q = 1,$ jumps occur only when $z \in \calZswitchtozero,$ 
and $V(z) \geq v,$ and flows occur if 
either $z\in \calZholdone$ or $\added{V(z) \leq v}.$
Hence, if $q = 1$, then the system jumps when $x$ is in
\begin{align} 
    D_{1} 
    % :\!&= \{x \in \calX_1 \mid V(z) \geq v 
    %     \AND  \dot V_1(z) \geq -\sigmaone(\distA{z}) \} \nonumber\\
    &:= \{x \in \calX_1 \mid V(z) \geq v\} 
    \intersect \left( \calZswitchtozero\times\nnreals\times\setdef{1} \right)
    \label{eq:D_1}
\end{align} 
and flows when $x$ is in 
\begin{equation}
    \squeezespaces{0.5}
    C_{1} := \closure{\calX_1 \!\setminus\! D_1} 
    % &= \{x \in \calX_1 \mid V(z) \leq v \OR \dot V_1(z) \leq -\sigmaone(\distA{z}) \} \\
    % &= \{x \in \calX_1 \mid V(z) \leq v \OR z\in \calZholdone\}\\
    = \{x \in \calX_1 \mid V(z) \leq v\} 
        \!\union\! \left( \calZholdone \!\times\!\nnreals\!\times\!\setdef{1} \right)
    \label{eq:C_1}
\end{equation}
where $\calX_1:= \plantstatespace \times \nnreals \times \setdef{1}.$ 
Then, the jump set is $D := D_0 \union D_1$ 
and the flow set is $C := C_0 \union C_1.$
% Paragraph break (per Ricardo 8/22/21), No paragraph break (per Ricardo 8/02/21)
Note that the flow set is 
the closed complement of the jump set:
$C = \closure{\calX \setminus D}$.

Next, we define the discrete and continuous dynamics
of the hybrid closed-loop system.
At each jump, $z$ is constant, 
since the plant state is continuous in time;
$v$ is set equal to $V(z)$ to \added{record} its value; 
and $q$ is toggled to the opposite value in $\setdef{0, 1}.$
During flows, $z$ evolves according to $\fp(z, \kappa_q(z))$ 
and the logic variable $q$ is held constant. 
The continuous dynamics of $v$ \added{are designed} next.

% Paragraph per Ricardo (10/9/2021)
We write the continuous dynamics for $v$ as $\dot v = \fv(z, v, q).$  
When $ q = 0,$ the value of $v$ does not affect the switching scheme;
we simply pick $\fv(z, v, 0) = -v$
so that $v$ exponentially converges to zero. % No paragraph break (per Ricardo)
When $q=1,$ however, the behavior of $v$ is crucial to 
ensuring that solutions to the 
closed-loop system converge to $\calAX$.
We design $\fv(z, v, 1)$
to satisfy the following rules:
{% Localize the effect of '\setupAssumption'
\setupAssumption[R] % Update "(R1)--(R5)" if this changes.
\newcounter{rulesCounter}
\begin{enumerate}
    \item If $V(z) = v = 0,$ then $\fv(z, v, 1) = 0$ 
    because $x\in \calAX$ has already been achieved.
    \label{rule:Ax achieved}
    \item If $V(z) \leq v \neq 0,$ then 
    $\fv(z, v, 1) < 0$ \added{is such} that $v$ converges to zero.
    The motivation for this choice is that $V\added{(z)}$ 
    is allowed to increase while $V(z) < v,$ so
    by making $v$ converge to zero, $V\added{(z)}$ is squeezed from 
    above, forcing either convergence or a switch to $q=0.$%
    \label{rule:v must decrease when V(z) leq v}%
    \setcounter{rulesCounter}{\value{enumi}}%
\end{enumerate}
If $V(z) > v,$ then $v$ is allowed to increase because,
eventually, one of the following must occur:
\begin{itemize}
    \item $z$ enters $\calZswitchtozero$ 
    prompting a switch to $q=0$;
    \item $V(z) = v,$ in which case \ref{rule:Ax achieved} or
    \ref{rule:v must decrease when V(z) leq v} applies; or
    \item $z \in \calZholdone$ and $V(z) > v$ hold for the rest of time,
    so $V\added{(z)}$ converges to $0$ and $v$ is squeezed to $0$ as well.
\end{itemize}

The behavior of $v$ while $V(z) \leq v \neq 0$ is 
crucial to the performance of the closed-loop system.
We prescribe the following cases:
\begin{enumerate}%[\ \ a)] %"\ \ % adds spacing to make left-edge line up with previous enumerate.
    \setcounter{enumi}{\value{rulesCounter}}
    \item If $z\in \calZholdone$ and $V(z) < v,$
    then $v$ remains greater than $V\added{(z)}$ 
    as long as $z$ remains in $\calZholdone.$ 
    This guarantees that the leeway above $V\added{(z)}$ 
    is maintained while $V\added{(z)}$ is decreasing fast enough to hold $q=1.$
    % ($v$  to decrease faster than
    % $V,$ however, as this is necessary if, say, $V$ is already zero.)
    \label{rule:V(z) < v while dot V_1(Z) small enough}
    
    \item If $z \in \calZswitchtozero$ \added{and $V(z) < v$,
    then $\dot V_1(z) > \fv(z, v, 1)$ holds}
    and, furthermore, if $z$ remains in $\calZswitchtozero,$ 
    then $v$ decreases until it reaches $V\added{(z)}$ in finite time, 
    causing a switch to $q=0.$
    This acts as a fail-safe in case $V\added{(z)}$ 
    otherwise fails to converge to zero.
    \label{rule:fv < dot V1}

    \item 
    % After a switch to $q=1$ (where $v$ is set to $V(z)$),
    % then $v$ must initially decrease slower than $V$. In particular, 
    If $z\in \calZswitchtoone$ and $V(z) = v$ 
    (as is the case immediately after every switch to $q=1$),
    then $\dot V_1(z) < \fv(z, v, 1) $ must hold. 
    This condition, in combination with 
    \ref{rule:V(z) < v while dot V_1(Z) small enough}, 
    ensures that ${z\in \calZswitchtoone}$ and ${V(z) = v}$ 
    only occur simultaneously immediately after a switch to $q = 1,$
    and the switch is immediately followed by an open interval $I$ 
    of ordinary time such that $V(z) < v$ for all $t\in I.$ 
    During $I,$ a switch to $q = 0$ is impossible, due to the design~of~$D_1.$% <- This "%" is important for preventing an extra space after the list.
    \label{rule:dot V1 < fv}
\end{enumerate}%
}%
To satisfy \ref{rule:Ax achieved}--\ref{rule:dot V1 < fv}, %TODO: Check before flight
we define $\fv$ at each $(z, v, 1)$ as
\begin{equation}
    \fv(z, v, 1) = -\sigmaone(\distA{z}) + \vdiffcoeff(V(z) - v)
    \label{eq:fv q = 1}
\end{equation}
with $\vdiffcoeff > 0$.
Clearly, \cref{eq:fv q = 1} satisfies 
\ref{rule:Ax achieved} and \ref{rule:v must decrease when V(z) leq v}.
Inspecting $\calZholdone,$ $\calZswitchtozero,$ 
and $\calZswitchtoone,$ we see that
\cref{eq:fv q = 1} also satisfies 
\ref{rule:V(z) < v while dot V_1(Z) small enough}--\ref{rule:dot V1 < fv}. %TODO: Check before flight
% It can be seen that rule \ref{rule:V(z) < v while dot V_1(Z) small enough}
% is satisfied by noting that if $z\in \calZholdone,$ then 
% $\dot V_1(z) \leq -\sigmaone(\distA{z}),$ 
% and $V(z) - v \to 0$ as $v \to V(z),$
% so 
The term $\vdiffcoeff(V(z) - v)$ pushes $v$ toward $V\added{(z)}$ 
at a rate proportional to the difference $V(z) - v$, 
which helps $v$ to ``catch up'' if $V\added{(z)}$ has dropped quickly.
% When $V(z) < v$, then $\vdiffcoeff(V(z) - v)$ is negative 
% and $-\sigmaone(\distA{z})$ is less than or equal to zero
% (with equality achieved when $z \in \calA$). 
% Together, they cause $v$ to converge to zero. 

Combining the cases for $q=0, 1$, the dynamics of $v$ are 
\begin{equation}
    \squeezespaces{0.2}
    \dot v 
    = \fv(z, v, q)
    := 
    \begin{cases}
                                                  -v, & \textif q = 0, \\
        -\sigmaone(\distA{z}) + \vdiffcoeff(V(z) - v), & \textif q = 1.
    \end{cases}
    \label{eq:vdot}
\end{equation}
The system parameters $\sigmaone$ and $\vdiffcoeff$ 
affect the rate at which $v$ converges toward zero while $q=1$.
Larger choices of $\sigmaone$ and $\vdiffcoeff$ cause $v$ to decay faster, 
which reduces the amount $V\added{(z)}$ can increase before switching back to $q=0$, 
whereas smaller choices of $\sigmaone$ and $\vdiffcoeff$ 
correspond with a stronger preference for $\kappa_1$ 
(see \cref{ex:MPC}).
% \deleted{The effect of changing $\vdiffcoeff$ 
% is illustrated in }{\color{blue}\cref{ex:MPC}.}

%%%%%%%%%%%%%%%%%%%%%%%%%%%%%%%%%%%%%%%%%%%
%     \subsection{System Specification}            
%%%%%%%%%%%%%%%%%%%%%%%%%%%%%%%%%%%%%%%%%%%

The construction above leads to the hybrid closed-loop system 
$\calH = (C, f, D, g)$ \added{with state $x = (z, v, q) \in \calX$ and data} given by 
\begin{equation} \squeezespaces{0.4}
    \hspace{-1pt}\begin{cases}
    f(x) := (\fp(z, \kappa_q(z)), \fv(x), 0)
            & \forall x \in C := C_0 \cup C_1 \\ 
    g(x) := (z,\: V(z),\: 1 - q) & \forall x\in D := D_0 \cup D_1
\end{cases}
\label{eq:switched hybrid system}
\end{equation}
with $\fv$ given in \cref{eq:vdot} 
and $D_0, D_1, C_0, C_1$ 
in \cref{eq:D_0,eq:D_1,eq:C_0,eq:C_1}.
The parameters of our hybrid control strategy are 
$\vdiffcoeff > 0,$
and continuous functions $\sigmazero, \sigmaone : \nnreals \to \nnreals$ 
such that $\sigmaone$ is positive definite and 
$\sigmaone(s) < \sigmazero(s)$ for all $s\geq 0.$

\begin{example}[LQR]
    \label{ex:lqr}
    Consider the nonlinear plant
    \begin{equation}
        % \zdot = \cos^2(\norm{z})A_1 z + \sin^2(\norm{z})A_2 z + u
        \zdot = A_1 z + h(\correction{\norm{z}}) A_2 z + u
        \label{eq:lqr plant}
    \end{equation}
    with $\squeezespaces{0.5}A_1 := \smallmat{0 & 2 \\ -2 & 0}$, 
    $\squeezespaces{0.5}A_2 := 4I$,
    and $h(s) = \correction{\min\{s, 1\}}$ for $s \geq 0$.
    This system behaves like $\zdot = A_1z$ near the origin and 
    like $\zdot = (A_1+A_2)z$ far from it.
    The origin of \cref{eq:lqr plant} is UGAS for
    $\kappa_0(z) := \smallmat{-5 & 0 \\ 0 & -6}z.$ 
    For $\kappa_1$, we linearize \cref{eq:lqr plant} about the origin
    and use the linear quadratic regulator (LQR) feedback that solves
    the following infinite-horizon optimal control problem:
    \begin{equation}
        \begin{aligned}
            \minimize{u} & \int_0^\infty \norm{z(t)}^2 + \norm{u(t)}^2 \dt \\
            \subjectto & \zdot = A_1z \correction{{} + u}.
        \end{aligned}
        \label{eq:lqr problem}
    \end{equation}
    \added{The LQR feedback is ${u = \kappa_1(z) := -z}.$}
    \Cref{fig:lqr example_1} \added{shows a solution to the hybrid closed-loop system}
    with $\squeezespaces{1.0}\vdiffcoeff = 1$, $\squeezespaces{1.0}\sigmazero(s) := 0.5s^2$, 
    and $\squeezespaces{1.0}\sigmaone(s) := 0.8s^2 + 10^{-3}$.
    \added{The switching logic} 
    uses $\kappa_1$ near the origin, 
    significantly reducing $\norm{u}.$ The leeway between $v$ and $V\added{(z)}$ 
    allows $\dot V_1\added{(z)}$ to be \added{briefly larger} than $-\sigmaone(\distA{z})$ without
    triggering a switch to $q=0.$ 
    In contrast, if $\vdiffcoeff =  4$ \added{(not shown)}, 
    then $v$ \added{decreases} faster,
    \added{causing $v$ to reach $V(z)$ and triggering} a switch to $q=0$.
    \added{The switch is followed} by a spike in control 
    effort, a period of faster convergence, \added{and}
    a subsequent switch back to $q=\added{1}.$
    \begin{figure}[htbp]
        \centering
        \includegraphics{lqr_example_1}
        \setlength{\belowcaptionskip}{-14pt}
        \caption{In \cref{ex:lqr}, the switching logic 
        uses the LQR controller $\kappa_1$ near the origin, 
        significantly reducing $\norm{u}.$ The leeway between $v$ and $V\added{(z)}$ 
        allows $\dot V_1\added{(z)}$ to be briefly greater 
        than $-\sigmaone(\distA{z})$ without 
        triggering a switch to $q=0.$
        }
        \label{fig:lqr example_1}
    \end{figure}
    % \begin{figure}[htbp]
    %     \centering
    %     \includegraphics{lqr_example_4}
    %     \caption{Plots for \cref{ex:lqr} show that compared to $\vdiffcoeff=1$,
    %     using $\vdiffcoeff=4$ 
    %     causes an additional switch to $q=0$ at $t = 1.2\sec$ because $v$ reaches $V.$
    %     The switch is followed by a spike in control 
    %     effort $\norm{u}$ and a period of faster convergence until $t=1.5\sec$ 
    %     when $\dot V_1$ hits $-\sigmazero(\distA{z}),$ prompting a switch back to $q=0.$}
    %     \label{fig:lqr example_4}
    % \end{figure}
\end{example}

%%%%%%%%%%%%%%%%%%%%%%%%%%%%%%%%%%%%%%%%%%%%%%%%%%%%%%%%%%%%%%%%
\subsection{Uniform Global Asymptotic Stability of $\calAX$}
%%%%%%%%%%%%%%%%%%%%%%%%%%%%%%%%%%%%%%%%%%%%%%%%%%%%%%%%%%%%%%%%
\label{sec:analysis}

% We now turn to the mathematical analysis of the closed-loop system.
Our results require the following assumption.
\begin{assumption} The functions $\fp, \kappa_0, \kappa_1,$ and $V$ 
    % satisfy: %the following properties. %%ONELINE
    satisfy the following properties.
    \label{assump:regularity}
    \setupAssumption[B]
    \begin{enumerate}
        % \item \label{assump:fp locally Lipschitz} $\fp$ is locally Lipschitz continuous.  
        \item \label{assump:fp continuous} $\fp,$ 
        $\kappa_0,$ and $\kappa_1$ are continuous;
        % Making V\inC^1 simplifies construction of Lyapunov function. 
        % We could relax it to just Lipschitz continuous 
        % at the cost of extra complexity. 
        \item \label{assump:V differentiable} 
        $V$ is continuously differentiable. 
    \end{enumerate}
\end{assumption}

\reportonly{
\subsection{Well-Posedness and Existence of Complete Solutions}
\label{sec:well-posedness}

%%%%%%%%%%% Hybrid Basic Conditions %%%%%%%%%%%%%%
In \cref{lem:hbc,lem:all maximal are complete}, we assert that 
the $\calH$ satisfies the hybrid basic conditions 
and every solution is complete.
}

% \begin{lemma} Suppose \cref{assump:regularity} holds. 
%     Then, system 
%     $\calH$ given in \cref{eq:switched hybrid system}
%      satisfies the hybrid basic conditions.
%     \label{lem:hbc}
% \end{lemma}

\reportonly{
\begin{proof}
    We check each item of the hybrid basic conditions:
    \begin{itemize}
        \setupAssumption[A]
        % \item[(A0)] The entire space $\calX$ is trivially open.
        \item The flow and jump sets can be written 
        as finite unions and intersections
        of sublevel sets of continuous functions, 
        hence the sets are closed.
        \item Function $f$ is continuous because $\fp, \kappa_0, \kappa_1,$ 
        and $V$ are continuous by assumption 
        and $\sigma_1$ is continuous by definition.
        \item Function $g$ is continuous because $V$ is continuous.\qedhere
    \end{itemize}
\end{proof}
}

%%%%%%%%%%%%%%%

\reportonly{
\begin{lemma}
    Suppose \cref{assump:regularity} holds. 
    Then, every maximal solution to 
    $\calH$ given in \cref{eq:switched hybrid system} is complete.
    \label{lem:all maximal are complete}
\end{lemma}
}
% \confonly{
%     \proofsketch The union $C\union D$ is the entire space $\calX,$ 
%     so at every point it is possible to flow or jump. 
%     Per \cref{lem:UGS}, $\calAX$ is uniformly globally stable, 
%     so $x$ cannot go to infinity in finite time.
%     Thus, by \cite[Proposition 2.34]{sanfelice_hybrid_2021}, 
%     every maximal solution is complete.
%     \todo[inline]{The order of the lemmas doesn't work well here because we haven't proven \cref{lem:UGS} yet.}
% }
\reportonly{
    The proof of \cref{lem:all maximal are complete} is contained in 
    \cref{sec:appendix}.
}

\reportonly{
    \subsection{Asymptotic Stability}
    \label{sec:asymptotic stability}
}

% To prove uniform global asymptotic 
% stability of $\calAX$ for $\calH$,
% we use the following functions defined in
% \cite[Equations 9 and 10]{sanfelice_invariance_2007} 
% that provide upper bounds on the change 
% in $\tV$ during flows and jumps, respectively: 
% \begin{align} 
%     v_C(x) &:= 
%     \begin{cases}
%         \tVc(x, f(x)) & \textif x\in C, f(x) \in T_C(x), \\
%               -\infty & \otherwise,
%     \end{cases}
%     \label{eq:v_C definition} \\
%     u_D(x) &:= 
%     \begin{cases} 
%         \tV(g(x)) - \tV(x) & \textif x \in D, \\
%                    -\infty & \otherwise
%     \end{cases}
%     \label{eq:u_D definition}
% \end{align}
% for each $x \in \calX.$ 
% We proceeded as follows:
\reportonly{%
    In \cref{lem:Clarke directional derivative},
    we compute the generalized directional derivative $\tVc$. 
}%
% \confonly{%
%    In \cref{lem:v_C < 0} and [Wintz, 2021]}%\cite[Lemma 5]{wintz_global_2021}, 
% }%
\reportonly{
    In \cref{lem:v_C < 0,lem:u_D leq 0}, 
}%
% we show that $v_C(x) < 0$ for all 
% $x \in \calX \setminus \calAX$ % NOT the same as negative definite.
% and $u_D(x) \leq 0$ for all $x\in \calX.$
% In \cref{lem:UGS,lm:forward invariance,lem:no Zeno}, 
% In \cref{lm:forward invariance,lem:no Zeno},
% we show
% that $\calAX$ is % globally uniformly stable and
% forward invariant for $\calH$, and that the 
% hybrid closed-loop system does not experience Zeno 
% behavior.
% Finally, \cref{the:UGAS} 
% asserts that $\calAX$ is 
% UGAS.
% \confonly{%
%     Omitted proofs are found in \cite{wintz_global_2021}.
% }%

%%%%%%%%%%% Generalized Directional Derivative %%%%%%%%%%%%%%

\reportonly{
\begin{lemma} % Clarke generalized directional derivative 
    Suppose \ref{assump:V differentiable} from \cref{assump:regularity} holds. 
    The Clarke generalized directional derivative 
    of $\tV$ at $x = (z, v, q) \in \calX$ 
    in the direction of 
    $\flowdir = (\flowdir_z, \flowdir_v, \flowdir_q) 
            \in \reals^{\nplant} \times \reals \times \{0\}$ is 
    \begin{equation}
        \label{eq:Clarke directional derivative (report)}
        \tVc(x, \flowdir)
        = \begin{cases}
                                    \ip{\nabla_z V(z)}{\flowdir_z} & \textif V(z) > v, \\
            \max\setdef{\ip{\nabla_z V(z)}{\flowdir_z}, \flowdir_v} & \textif V(z) = v, \\
                                                        \flowdir_v & \textif V(z) < v.
        \end{cases}
    \end{equation}
    \label{lem:Clarke directional derivative}
\end{lemma}

\begin{proof}
    By assumption, $V$ is differentiable, so $\tV(x)$ is smooth everywhere 
    except where $V(z) = v.$ 
    Thus, the Clarke generalized gradient of $\tV$ at $x$ is 
    \begin{equation}
        \ggrad \tV(x) = \begin{cases}
            (\nabla_z V(z), 0, 0)                            & \textif V(z) > v, \\
            \begin{aligned}\convex\{&(\nabla_z V(z), 0, 0), 
                (\zerovec{\nplant}, 1, 0)\} \end{aligned} & \textif V(z) = v, \\
                (\zerovec{\nplant}, 1, 0)                  & \textif V(z) < v.
        \end{cases} 
    \end{equation}
    % Then the Clarke generalized directional derivative of 
    % $\tV$ in the direction of $\flowdir$ is defined as
    % $$ \tVc(x, \flowdir) = \max_{\zeta \in \ggrad V(x)} \ip{\zeta}{\flowdir}.$$ 

    Take $\xeq = (\zeq, \veq, \qeq) \in \calX$ such that $V(\zeq) = \veq$ 
    and $\xneq = (\zneq, \vneq, \qneq) \in \calX$ such that $V(\zneq) \neq \vneq.$
    We can immediately compute,
    the Clarke generalized directional derivative of 
    $\tV$ in the direction of $\flowdir$
    because $\ggrad \tV(\xneq)$ is single valued:
    $$ \tVc(\xneq, \flowdir)
    = \begin{cases}
        \ip{\nabla_z V(\zneq)}{\flowdir_z} & \textif V(\zneq) > \vneq, \\
                                \flowdir_v & \textif V(\zneq) < \vneq.
    \end{cases}
    $$

    To compute $\tVc(\xeq, \flowdir),$ we first expand the expression of the convex hull
    \begin{align*}
        \ggrad \tV(\xeq) &= \convex\setdef{\mat{\nabla_z V(\zeq) \\ 0 \\ 0},
                                           \mat{\zerovec{\nplant} \\ 1 \\ 0}}
        \\ &= \setdef{\theta \mat{\nabla_z V(\zeq) \\ 0 \\ 0} 
                    + (1-\theta) \mat{\zerovec{\nplant} \\ 1 \\ 0} 
                                \suchthat \theta \in [0, 1]}
        \\ &= \setdef{ \mat{\theta \nabla_z V(\zeq) \\ 1-\theta \\ 0} 
                                \suchthat \theta \in [0, 1] }.
    \end{align*}
    Thus,
    \begin{align*}
        \max_{\zeta\in \ggrad \tV(\xeq)} \ip{\zeta}{\flowdir} 
        &=\max_{\theta \in [0, 1]} \ip{\mat{\theta \nabla_z V(\zeq) \\ 1-\theta \\ 0} }
                                      {\mat{\flowdir_z \\ \flowdir_v \\ \flowdir_q}} \\
        &=\max_{\theta \in [0, 1]} \theta\ip{\nabla_z V(\zeq)}{\flowdir_z} 
                                    + (1 - \theta) \flowdir_v \\
        &=\max_{\theta \in [0, 1]} \theta\left(\ip{\nabla_z V(\zeq)}{\flowdir_z} - \flowdir_v\right) + \flowdir_v \\
        &=\begin{cases}\, 
                \ip{\nabla_z V(\zeq)}{\flowdir_z} & \textif \ip{\nabla_z V(\zeq)}{\flowdir_z} \geq \flowdir_v \\
                \flowdir_v & \textotherwise
            \end{cases} \\ 
        &= \max\setdef{\ip{\nabla_z V(\zeq)}{\flowdir_z}, \flowdir_v},
    \end{align*}
    where we use the fact that the constrained 
    maximum of an affine function is obtained 
    on the boundary
    (i.e., when $\theta\in\{0, 1\}$).
    Combining the $\xneq$ and $\xeq$ cases, 
    we get \cref{eq:Clarke directional derivative (report)}.
\end{proof}
}

%%%%%%%%%%% v_C is negative %%%%%%%%%%%%%%

% \begin{lemma} % v_c < 0
%     \deleted{Suppose} \cref{assump:regularity} \deleted{holds.
%     Then, $v_C(x) < 0$ for all % NOT the same as negative definite.
%     $x \in \calX \setminus \calAX.$ }
%     \label{lem:v_C < 0}
% \end{lemma}

% \confonly{
% \proofsketch
%     Consider the function \cite[Equation 8]{sanfelice_invariance_2007}
%     \begin{equation} 
%         u_C(x) = \begin{cases}
%             \tVc(x, f(x)) & \textif x \in C, \\
%                   -\infty & \textotherwise
%         \end{cases}
%         \label{eq:u_C definition}
%     \end{equation}
%     and the set $\calB := \{x \in \calX_1 \mid 
%     V(z) = v \AND \dot V_1(z) > -\sigma_1(\distA{z})\}$.
%     Our proof proceeds in three steps:
%     (1) show that $u_C(x) <\nolinebreak 0$ for all 
%     $x \in \calX \setminus (\calAX \union \calB)$;
%     (2) show that $f(x) \not\in T_C(x)$ for all 
%     $x \in \calB$ and thus $v_C(x) = -\infty$ in $\calB$; and
%     (3) conclude $v_C(x) < 0$ for all $x\in\calX$ 
%     because $v_C(x) \leq u_C(x)$.

%     The set $\calX \setminus (\calAX \union \calB)$ is 
%     partitioned by the three sets
%     $$\calX \setminus C, \quad
%     C_0 \setminus (\calAX \union \calB), \midand 
%     C_1 \setminus (\calAX \union \calB).$$
%     We show that $u_C(x) < 0$ 
%     for all $x$ in each of these sets separately.
%     For $x \in \calX \setminus C,$ then 
%     $u_C(x) = -\infty$ because $x\not\in C.$
%     For $x\sszero := (z\sszero, v\sszero, 0) \in C_0 \setminus (\calAX \union \calB)$,
%     it can be shown that $u_C(x\sszero) < 0$ 
%     by substitution of \cref{eq:Clarke directional derivative (conf)} 
%     into \cref{eq:u_C definition}.
%     For $x\ssone := (z\ssone, v\ssone, 1) \in C_1 \setminus(\calAX \union \calB),$
%     we consider
%     \begin{equation*}
%         u_C(x\ssone) 
%         = \begin{cases}
%                                                 \dot V_1(z\ssone) & \textif V(z\ssone) > v\ssone, \\
%             \max\{\dot V_1(z\ssone), -\sigma_1(\distA{z\ssone})\} & \textif V(z\ssone) = v\ssone, \\
%             -\sigma_1(\distA{z\ssone}) + \vdiffcoeff(V(z\ssone) - v\ssone)
%                                                                   & \textif V(z\ssone) < v\ssone
%         \end{cases}
%         % \label{eq:u_c of x_1_conf}
%     \end{equation*}
%     case-by-case.
%     % \begin{itemize}
%     %     \item[\textbf{Case 1 ($V(z\ssone) > v\ssone$):}] 
%         If $V(z\ssone) > v\ssone,$ then 
%         $\dot V_1(z\ssone) \leq -\sigma_1(\distA{z\ssone})$ 
%         (from the definition of $C_1$),
%         so $$u_C(x\ssone) = \dot V_1(z\ssone) \leq -\sigma_1(\distA{z\ssone}) < 0.$$
%         % \item[\textbf{Case 3 ($V(z\ssone) = v\ssone$):}] 
%         If $V(z\ssone) = v\ssone,$ 
%         then, $x\ssone \not\in \calB$ 
%         implies that $\dot V_1(z\ssone) \leq -\sigma_1(\distA{z\ssone})$, so
%         $u_C(x\ssone) = -\sigma_1(\distA{z\ssone}) < 0.$
%         % \item[\textbf{Case 2 ($V(z\ssone) < v\ssone$):}] 
%         If $V(z\ssone) < v\ssone,$ then
%         $$ u_C(x\ssone) = -\sigma_1(\distA{z\ssone})
%                         + \vdiffcoeff(V(z\ssone) - v\ssone)
%                         < 0.$$
%     % \end{itemize}
%     Therefore, $u_C$ is strictly negative on 
%     $\calX \setminus(\calAX \union \calB).$

%     Now consider $x\ssb = (z\ssb, v\ssb, 1) \in \calB.$ 
%     % We show that $x\ssb$ is in the boundary of $C_1,$ 
%     % and $f(x\ssb) \not\in \tangentcone{C_1}(x)$. 
%     Per~\cite[Proposition 4.3.7]{aubin_tangent_2009},
%     the tangent cone of $C_1$ at $x\ssb$ is
%     $$T_{C_1}(x\ssb) 
%     = \setdef{w \in T_{\calX_1}(x\ssb) 
%              \suchthat \ip{\nabla_x(V(z\ssb) - v\ssb)}{w} \leq 0}.$$
%     Evaluating $\ip{\nabla_x(V(z\ssb) - v\ssb)}{w}$ 
%     with $w = f(x\ssb),$ we see that $f(x\ssb) \not\in T_{C_1}(x\ssb)$ because
%     \begin{align*}
%         \ip{\nabla_x(V(z\ssb) - v\ssb)}{f(x\ssb)} 
%         &= \dot V_1(z\ssb) + \sigma_1(\distA{z\ssb}) > 0
%     \end{align*} 
%     (the last inequality comes from 
%     the criteria for $x\ssb \in \calB$).
%     Thus, $v_C(x\ssb) = -\infty.$ 
%     From the definitions of $v_C$ and $u_C$, it is clear
%     that $v_C(x) \leq u_C(x)$ for all $x \in \calX$. 
%     Therefore, $v_C(x) < 0$ for all $x \in \calX \setminus \calAX.$
%     \qed
% }

\reportonly{
\begin{proof}
    In this proof, we use the function \cite[Equation 8]{sanfelice_invariance_2007}
    \begin{equation} 
        u_C(x) = \begin{cases}
            \tVc(x, f(x)) & \textif x \in C, \\
                  -\infty & \textotherwise,
        \end{cases}
        \label{eq:u_C definition (report)}
    \end{equation}
    and the set $$\calB := \setdef{x \in \calX_1  \suchthat 
    V(z) = v \AND \dot V_1(z) > -\sigma_1(\distA{z})}.$$ 
    (Shown below, the set $\calB$ contains all the points 
    in the boundary of $C_1$ where the flow map points out of $C_1.$)

    Our proof proceeds in three steps:
    \begin{enumerate}
        \item show that $u_C(x) < 0$ for all 
        $x \in \calX \setminus (\calAX \union \calB)$;
        \item Show that $f(x) \not\in T_C(x)$ for all 
        $x \in \calB$ and thus $v_C(x) = -\infty$ in $\calB$; and
        \item conclude $v_C(x) < 0$ for all $x\in\calX$ 
    because $v_C(x) \leq u_C(x)$.
    \end{enumerate} 

    The set $\calX \setminus (\calAX \union \calB)$ is 
    partitioned by the three sets
    $$\calX \setminus C, \quad
    C_0 \setminus (\calAX \union \calB), \midand 
    C_1 \setminus (\calAX \union \calB).$$
    We show that $u_C(x) < 0$ 
    for all $x$ in each of these sets separately.
    For $x \in \calX \setminus C,$ then 
    $u_C(x) = -\infty$ because $x\not\in C.$

    For $x\sszero := (z\sszero, v\sszero, 0) \in C_0 \setminus (\calAX \union \calB) 
	= C_0 \setminus \calAX$ ($\calB$ and $C_0$ are disjoint),
    we show that $u_C(x\sszero) < 0$ 
    by substitution of \cref{eq:Clarke directional derivative (report)} 
    into \cref{eq:u_C definition (report)}.
    Note that 
    $f(x\sszero) = \left( \fp(z\sszero, \kappa_0(z\sszero)), -v\sszero \right).$ 
    Thus, 
    % by substitution 
    % into \cref{eq:Clarke directional derivative (report),eq:u_C definition (report)}
    we find
    \begin{equation}
        \begin{aligned}
        u_C(x\sszero) &= \tVc(x\sszero, f(x\sszero)) 
        = \begin{cases}
                               \dot V_0(z\sszero) & \textif V(z\sszero) > v\sszero, \\
            \max\{\dot V_0(z\sszero), -v\sszero\} & \textif V(z\sszero) = v\sszero, \\
                                        -v\sszero & \textif V(z\sszero) < v\sszero
        \end{cases}
        \end{aligned}
        \label{eq:u_c of x_0}
    \end{equation}
    where we used the definition of $\dot{V}_0(z\sszero)$ to replace 
    $\ip{\del_z V(z\sszero)}{\fp(z\sszero, \kappa_0(z\sszero))}.$
    It follows immediately that $u_C$ is negative definite relative to $\calAX$ on $\calX_0$
    because $\dot V_0$ is negative definite relative to $\calA$ and 
    $v\sszero$ is nonnegative.
    
    For $x\ssone := (z\ssone, v\ssone, 1) \in C_1 \setminus(\calAX \union \calB),$
    we consider
    \begin{equation}
        u_C(x\ssone) 
        = \begin{cases}
                                                \dot V_1(z\ssone) & \textif V(z\ssone) > v\ssone, \\
            \max\{\dot V_1(z\ssone), -\sigma_1(\distA{z\ssone})\} & \textif V(z\ssone) = v\ssone, \\
            -\sigma_1(\distA{z\ssone}) + \vdiffcoeff(V(z\ssone) - v\ssone)
                                                                  & \textif V(z\ssone) < v\ssone
        \end{cases}
        \label{eq:u_c of x_1}
    \end{equation}
    case-by-case.
    \begin{itemize}
        \item[\textbf{Case 1 ($V(z\ssone) > v\ssone$):}] 
        When $V(z\ssone) > v\ssone,$ then
        $x\ssone \in C_1$ implies that 
        $\dot V_1(z\ssone) \leq -\sigma_1(\distA{z\ssone})$. 
        Furthermore, $-\sigma_1$ is negative definite,
        so $u_C(x\ssone) = \dot V_1(z\ssone) \leq -\sigma_1(\distA{z\ssone}) < 0.$


        \item[\textbf{Case 2 ($V(z\ssone) = v\ssone$):}] When $V(z\ssone) = v\ssone,$ 
        then, $x\ssone \not\in \calB$ 
        implies that $\dot V_1(z\ssone) \leq -\sigma_1(\distA{z\ssone})$.
        By substitution into \cref{eq:u_c of x_1}, we see
        \begin{align*}
            u_C(x\ssone) 
            &= \max\left\{ \dot V_1(z\ssone), 
                                -\sigmaone(\distA{z\ssone}) \right\} 
            = -\sigma_1(\distA{z\ssone}) < 0.
        \end{align*}

        \item[\textbf{Case 3 ($V(z\ssone) < v\ssone$):}] When $V(z\ssone) < v\ssone,$ 
        it is immediately apparent that $u_C$ is strictly negative:
        $$ u_C(x\ssone) = \underbrace{-\sigma_1(\distA{z\ssone})}_{< 0} 
                        + \underbrace{\vdiffcoeff(V(z\ssone) - v\ssone)}_{< 0}
                    < 0.$$
    \end{itemize}
    Therefore, $u_C$ is strictly negative on $\calX \setminus(\calAX \union \calB).$
    
    \newcommand{\Chat}{\macrocolor{\hat C_1}}
    Now consider $x\ssb = (z\ssb, v\ssb, 1) \in \calB.$ 
    From the definition of $\calB,$ we have that 
    $V(z\ssb) = v\ssb$ and $\dot V_1(z\ssb) > -\sigma_1(\distA{z\ssb}).$
    We show, next, that the point 
    $x\ssb$ is on the boundary of $C_1$ 
    and that $f(x\ssb)$ points out of $C_1$ (i.e., $f(x\ssb) \not\in T_{C_1}(x\ssb) $),
    so the system cannot flow from $\calB.$ 
    To construct $\tangentcone{C_1}(x\ssb),$ we first rewrite $C_1$ as 
    \begin{align*}
        C_1 =& \setdef{x \in \calX_1 \suchthat V(z) - v \leq 0} 
            \union \{x \in \calX_1 \mid \dot V_1(z) \leq -\sigma_1(\distA{z})\}.
    \end{align*}
    Clearly, $x\ssb$ is not in $\{x \in \calX_1 \mid \dot V_1(z) \leq -\sigma_1(\distA{z})\}$
    and there is an open set around $x\ssb$ that is disjoint from that set, 
    so the tangent cone 
    to $C_1$ at $x\ssb$ is the same as the tangent cone to 
    $\Chat := \setdef{x \in \calX_1 \suchthat V(z) - v \leq 0}.$
    Per~\cite[Proposition 4.3.7]{aubin_tangent_2009},
    the tangent cone of $\Chat$ at $x\ssb$ is
    $$T_\Chat(x\ssb) = \setdef{w \in T_{\calX_1}(x\ssb) 
            \suchthat \ip{\nabla_x(V(z\ssb) - v\ssb)}{w} \leq 0}.$$
    Evaluating $\ip{\nabla_x(V(z\ssb) - v\ssb)}{w}$ 
    with $w = f(x\ssb),$ we see that $f(x\ssb) \not\in T_{\Chat}(x\ssb)$ because
    \begin{align*}
        \ip{\nabla_x(V(z\ssb) - v\ssb)}{f(x\ssb)} 
        &= \ip{\mat{\nabla_z V(z\ssb) \\ - 1 \\ 0}}
            {\mat{\fp(z\ssb, \kappa_1(z\ssb)) \\ -\sigma_1(\distA{z\ssb}) \\ 0}} \\ 
        &= \ip{\nabla_z V(z\ssb)}{\fp(z\ssb, \kappa_1(z\ssb))} + \sigma_1(|z\ssb|_\calA) \\ 
        &= \dot V_1(z\ssb) + \sigma_1(|z\ssb|_\calA) \\ 
        &> 0,
    \end{align*}
    with the last inequality from the criteria that $x\ssb \in \calB$. 
    Thus, $f(x\ssb) \not\in T_\Chat(x\ssb) = T_{C_1}(x\ssb),$ so 
    $v_C(x\ssb) = -\infty.$ 
    From the definitions of $v_C$ and $u_C$, it is clear
    that $v_C(x) \leq u_C(x)$ for all $x \in \calX$. 
    Therefore, $v_C(x) < 0$ for all $x \in \calX \setminus \calAX.$
\end{proof} 
}

\reportonly{
\begin{lemma} % u_D \leq 0
    % No assumptions needed.
    Consider the function $u_D : \calX \to [-\infty, \infty)$ given in 
    \cite[Equation 9]{sanfelice_invariance_2007} as
    $$u_D(x) = \begin{cases} 
        \tV(g(x)) - \tV(x) & \textif x \in D \\
                   -\infty & \otherwise
    \end{cases} $$
    for $x \in \calX.$
    Then, $u_D(x) \leq 0$ for all $x\in \calX.$
    \label{lem:u_D leq 0}
\end{lemma}

\begin{proof}
    The conclusion follows directly from the definition of $\tV$ and the fact that $z$ is unchanged over jumps:
    \begin{align*}
        \tV(g(x)) - \tV(x) 
        &= \max\setdef{V(z), V(z)} - \max\setdef{V(z), v} \\
        &= \min\setdef{0, V(z) - v}  \\
        &\leq 0. \qedhere
    \end{align*}
\end{proof}
}

% \begin{lemma} % UGS
%     \deleted{Suppose} \cref{assump:regularity} \deleted{holds.
%     Then, the set $\calAX$ is uniformly globally stable for $\calH$ 
%     given in} \cref{eq:switched hybrid system}.
%     \label{lem:UGS}
% \end{lemma}

% \confonly{
% \proofsketch
% Functions $v_C$ and $u_D$ are nonincreasing along 
% solutions. Thus, \cite[Theorem 7.6]{sanfelice_invariance_2007} applies and is used to 
% show that every sublevel set of $\tV$ is stable, 
% thereby establishing that every solution is bounded. \qed
% }

\reportonly{
\begin{proof}
    Consider $\tV(x) = \max\{V(z), v\}$. 
    By assumption, $V(z)$ is differentiable, and thus locally Lipschitz.
    Furthermore, $V$ is positive definite on $\plantstatespace$ relative to $\calA.$ 
    It follows, then, that $\tV(z)$ is locally Lipschitz 
    and positive definite relative to $\calAX.$ 
    By \cref{lem:v_C < 0,lem:u_D leq 0}, $v_C(x) \leq 0$ and $u_D(x) \leq 0$ everywhere 
    and $v_C(x) < 0$ on $\calX \setminus \calAX.$ 
    This satisfies $(*)$ for \cite[Theorem 7.6]{sanfelice_invariance_2007} 
    and thus $\calAX$ is stable.

    \newcommand{\zerolevel}{\macrocolor{\tV_r\invs(0)}}
    To show that all solutions are bounded, 
    we will apply \cite[Theorem 7.6]{sanfelice_invariance_2007}
    to sublevels sets of $\tV.$
    First, note that $\tV$ is radially unbounded (as a result of $V$ being radially unbounded)
    so every sublevel set of $\tV$ is compact.
    Now, take $\tV_r(x) = \max\setdef{0, \tV(x) - r}.$ 
    For every $r \geq 0,$ the zero level set of $\tV_r$, denoted $\zerolevel,$ is equal to 
    the $r$-sublevel set of $\tV.$
    Furthermore, $\tV_r$ is positive definite relative to $\zerolevel.$
    Replacing $\tV$ with $\tV_r$ in the definitions of $v_C$ and $u_D$  
    only changes the values of these functions on $\zerolevel$ 
    where $v_C$ and $u_D$ become zero (or $-\infty$ outside $C \union D$).
    Therefore, by \cite[Theorem 7.6]{sanfelice_invariance_2007}, 
    every sublevel set of $\tV$ is stable,
    and, thus, every solution to $\calH$ is bounded.
\end{proof}
}

%%%%%%%%%%% Forward Invariance %%%%%%%%%%%%%%

% \begin{proposition} % forward invariant
%     Suppose \cref{assump:regularity} holds.
%     Then, the set $\calAX$ is forward invariant for $\calH$ 
%     given in \cref{eq:switched hybrid system}.
%     \label{lm:forward invariance}
% \end{proposition}
% 
% \confonly{
%     \proofsketch
%     The functions $v_C$ and $u_D$ are nonpositve, 
%     so $V$ cannot increase along solutions. 
%     Thus, by positive definiteness of $V$ relative to $\calA,$ 
%     solutions remain in $\calA.$ \qed
% }

\reportonly{
\begin{proof}
    \newcommand{\init}{_{\macrocolor 0}}
    \newcommand{\os}{_{\macrocolor 1}}
    \newcommand{\edge}{_{\macrocolor *}}
    Pick $x\init \in \calAX$ 
    and let $x = (z, v, q)$ be any solution from $x\init.$ 
    Assume that for some $(t\os, j\os) \in \dom x,$ 
    that $x(t\os, j\os)$ is not in $\calAX.$ Because $\calAX$ is closed, 
    there is a maximal time $(t\edge, j\edge)\in \dom x$ prior to $(t\os, j\os)$ such that 
    $x(t\edge, j\edge) \in \calAX.$ 
    The function $\tV$ is positive definite relative to $\calAX,$ 
    so $\tV(x(t\edge, j\edge)) = 0$ and $\tV(t\os, j\os) > 0.$
    By \cref{lem:v_C < 0,lem:u_D leq 0}, the change in $\tV$ is 
    upper bounded along flows by $v_C$ and at jumps by $u_D.$
    In particular, as described in \cite{sanfelice_invariance_2007}, 
    at each jump time $(t_{j+1}, j) \in \dom x,$
    $$V(x(t_{j+1}, j+1)) - V(x(t_{j+1}, j)) \leq u_D(x(t_{j+1}, j))$$ and, 
    at almost every all $(t,j) \in \dom x,$ 
    $$ \ddt V(x(t, j)) \leq v_C(x(t, j)).$$
    % Together, these two inequalities imply 
    % \cite[Equation 2]{sanfelice_invariance_2007}
    % \begin{equation}
    %     V(x(t', j')) -  V(x(t, j)) \leq \int_t^{t'} u_C(x(s, ))
    % \end{equation}
    We have shown, however, that $v_C$ and $u_D$ are nonpositve, 
    so $\tV$ cannot increase, producing a contradiction. 
    Thus, solutions cannot leave $\calAX.$
    Additionally, by \cref{lem:all maximal are complete}, 
    all maximal solutions are complete.
    Therefore, $\calAX$ is forward invariant.
    % {\color{gray} % Old proof
    % \input{Forward Invariance Lemma.tex}}
\end{proof}
}

%%%%%%%%%%%%%%%

For solutions to $\calH$, every jump is followed by an interval of flow.
The following proposition states that for every solution to $\calH$,
the lengths of all intervals of flow have a strictly positive lower bound.
As a consequence, $\calH$ does not have Zeno solutions.%
\begin{proposition} % No Zeno behavior
    Suppose \cref{assump:regularity} holds.
    Then, for each solution $x$ 
    to $\calH$ in \cref{eq:switched hybrid system}, 
    there exists $\gamma > 0$ 
    such that $t_{j+1} - t_j > \gamma$ 
    for all $(t_j, j), (t_{j+1}, j+1) \in \dom x.$% <- This "%" is important for preventing extra space after the proposition.
    \label{lem:no Zeno}
\end{proposition} 
% \confonly{
%     \proofsketch
%     The sets $D$ and $g(D)$ are 
%     disjoint, $\calH$ satisfies
%     the hybrid basic conditions~\cite[Lemma 1]{wintz_global_2021},
%     and every solution to $\calH$ is bounded (\cref{lem:UGS}). 
%     Thus, the conclusion follows from~\cite[Lemma 2.7]{sanfelice_invariance_2007}.
%     \qed
% }
% \reportonly{
% \begin{proof}
%     First, we show that $D \intersect g(D) = \emptyset$. 
%     For sake of discussion, we separate the components of 
%     the jump map using the following notation 
%     $g(x) = (g_z(z), g_v(z), g_q(q)) = (z, V(z), 1-q).$
%     Recall that 
%     we defined $\sigmazero$ and $\sigmaone$ such that  
%     $\sigmaone(s) < \sigmazero(s)$ for all $s \geq 0.$ 
%     % and that
%     % \begin{align*} 
%     %     D_{0} &:= \{x \in \calX_0 \mid \dot{V}_1(z) \leq -\sigmazero(\distA{z}) \} \\
%     %     D_{1} &:= \{x \in \calX_1 \mid V(z) \geq v,  \dot V_1(z) \geq -\sigmaone(\distA{z}) \}.
%     % \end{align*}
%     Take $x\sszero = (z\sszero, v\sszero, 0) \in D_0$.
%     Then, 
%     $$\dot V_1(g_z(z\sszero)) 
%         \leq -\sigmazero(\distA{g_z(z\sszero)}) 
%            < -\sigmaone(\distA{g_z(z\sszero)}),$$
%     thus $g(x\sszero) \not\in D_1.$
%     Similarly, for $x\ssone = (z\ssone, v\ssone, 1) \in D_1,$ 
%     $$\dot V_1(g_z(z\ssone)) 
%         \geq -\sigmaone(\distA{g_z(z\ssone)}) 
%            > -\sigmazero(\distA{g_z(z\ssone)}),$$
%     so $g(x\ssone) \not\in D_0.$
%     Therefore, $D \intersect g(D) = \emptyset.$

%     Additionally, $\calH$ satisfies the hybrid basic conditions (\cref{lem:hbc})
%     and every solution is bounded (\cref{lem:UGS}), thus, 
%     by \cite[Lemma 2.7]{sanfelice_invariance_2007}, 
%     we have that for every solution $x$ to $\calH$, 
%     there exists $\gamma > 0$ such that $t_{j+1} - t_j > \gamma$ 
%     for all $$(t_j, j), (t_{j+1}, j+1) \in \dom x.$$ 
%     Therefore, no solution to $\calH$ exhibits Zeno behavior.
% \end{proof}
% }
%%%%%%%%%%%%%%%
\begin{figure*}[t]
    \centering
    {\includegraphics{mpc_example_1}}
    % \quad
    {\includegraphics{mpc_example_26}}
    \setlength{\belowcaptionskip}{-12pt}
    \caption{In \cref{ex:MPC} with $\vdiffcoeff = 1$,
    a slow decrease in $v$ allows $V\added{(z)}$ to return almost to 
    its value at the start of each $q = 1$ interval. When $\vdiffcoeff=26$, however, 
    $v$ decreases faster, limiting the increase in $V\added{(z)}$
    after each drop.}
    \label{fig:mpc example}
\end{figure*}

The next result asserts that every maximal solution to $\calH$
is complete and converges to $\calAX$; thus, the hybrid controller 
successfully steers the plant state $z$ to $\calA$.
\begin{theorem}
    Suppose \cref{assump:regularity} holds. 
    Then, the set $\calAX$ in \cref{eq:Ax} is UGAS 
    for $\calH$ given in \cref{eq:switched hybrid system}.
    \label{the:UGAS}
\end{theorem}

\proofsketch
Consider the function 
\begin{equation}
    \tV(x) := \max\{V(z), v\}.
\end{equation} 
Outside $\calAX$, $\tV\added{(x)}$ 
decreases along flows because
if $q = 0,$ then both $V\added{(z)}$ and $v$ are decreasing; 
if $q = 1$ and $V(z) \geq v,$ 
then $V\added{(z)}$ is decreasing; and if $q = 1$ and $V(z) \leq v,$ 
then $v$ is decreasing.
    To show that $\tV(x)$ is decreasing, let $\tVc(x, f(x))$ be 
    the Clarke generalized directional derivative~\cite{clarke_optimization_1990}
    of $\tV$ at $x \in \calX$ in the direction of $f(x)$
    and let 
        $$\calB := \{{x \in \calX} \mid {q=1}, 
                V(z) = v \AND \dot V_1(z) > -\sigma_1(\distA{z})\}.$$
    For all $\squeezespaces{0.5} x \in C \setminus (\calB \union \calAX)$, 
    $\squeezespaces{0.5} \tVc(x, f(x)) < 0$.
    For each $\squeezespaces{0.5}x \in \calB$, the flow map $f(x)$ points out of $C$, 
    so flows are \added{im}possible.
    Thus, $\tV\added{(x)}$ decreases along flows in $C \setminus \calAX$.
    Within $\calAX$, $\tV$ is \added{zero}. 
    Furthermore, $\tV\added{(x)}$ does not increase at jumps
    and Zeno behavior does not occur (see \cref{lem:no Zeno}).
    Therefore, $\tV$ is a (nonsmooth) Lyapunov function for $\calH$, and
    $\calAX$ is UGAS
    \cite[Theorem 7.8]{sanfelice_invariance_2007}, 
    \cite[Theorem 3.22]{sanfelice_hybrid_2021}.
    \qed\par
    It can also be shown that $\calH$ satisfies 
    the hybrid basic conditions \cite[Definition 2.18]{sanfelice_hybrid_2021}, 
    so the asymptotic stability of $\calAX$ is 
    robust to small
    disturbances~\cite[Theorem 3.26]{sanfelice_hybrid_2021}.

% \begin{proof}
%     By \cref{lem:no Zeno}
%     there are no Zeno solutions to 
%     $\calH$ and by \cref{lem:v_C < 0},
%     $v_C(x) < 0$ for all $x \in \calX\setminus\calAX.$ 
%     Thus, we can apply \cite[Theorem 7.8]{sanfelice_invariance_2007}
%     to conclude that every sublevel set of $\tV$ is locally asymptotically stable. 
%     Altogether, this implies $\calAX$ is globally asymptotically stable. 
%     Thus, by \cite[Theorem 3.22]{sanfelice_hybrid_2021}, 
%     $\calAX$ is $\KL$ pre-asymptotically stable on $\calX$ 
%     for $\calH$; see \cite{sanfelice_hybrid_2021}.
%     Because $\calAX$ is compact and $\calH$ satisfies the hybrid basic conditions, 
%     $\KL$ pre-asymptotic stability on $\calX$ implies 
%     uniform global pre-asymptotic stability; see 
%     \cite[Remark 3.10]{sanfelice_hybrid_2021}.
%     Finally, all maximal solutions are complete, % per \cref{lem:all maximal are complete}, 
%     therefore $\calAX$ is UGAS. 
% \end{proof}

%%%%%%%%%%%%%%%

The next example shows how
$\vdiffcoeff$ affects performance.

\begin{example}[MPC]
\label{ex:MPC}
\newcommand{\fsampled}{{f_{s}}}
\newcommand{\zsampled}{{z_{s}}}
\newcommand{\usampled}{{u_{s}}}
\newcommand{\Tsample}{\added{T}}
\newcommand{\Tactual}{{T_{c}}}
Let $\squeezespaces{0.7}\zdot = u$ with $\squeezespaces{0.7}z, u\in\reals^2$,
let $\kappa_0(z) := \half\smallmat{-1 & 2 \\ -2 & -1}z$,
and let $\kappa_1$ be a 
model predictive controller (MPC) with 
periodic updates.
Between updates, zero-order hold (ZOH) is used 
to generate the control signal.
The switching logic for $\calH$ preserves 
the stability properties of $\calAX$ 
regardless of the choice of $\kappa_1,$
so solutions to the hybrid closed-loop system converge
even if the MPC algorithm fails to compute updated control 
values by the next ZOH sample time.
Suppose $\Tsample = 1\sec$ is the ZOH sample-time used in the 
computation of the MPC feedback
(that is, the duration that each input 
value is designed to be applied for)
and suppose $\Tactual = 2\sec$
is the actual time required to compute the MPC feedback. 
Because $\Tactual > \Tsample,$ a new MPC feedback value is 
not available at every sample time, 
in which case 
the feedback values from the previous interval are reused.

\Cref{fig:mpc example} shows solutions to $\calH$ 
with $\kappa_1$ computed using the \Matlab MPC Toolbox
and with $\sigmaone(s) := 0.3 s^2,$ and 
$\sigmazero(s) := 0.36 s^2 + 0.5.$
For $\vdiffcoeff = 1,$ 
we see that $V\added{(z)}$ decreases quickly as $t$ approaches 
$1 \sec$ (the end of the interval when the 
MPC feedback value is designed to be applied), but 
since an updated value is unavailable, 
$\kappa_1$ holds the same value until $t=2\sec$. 
After $t = 1\sec$, 
$V\added{(z)}$ rises quickly until it hits $v$,
causing a switch to $q=0.$
Consequently, despite $\kappa_1$'s poor performance, 
the closed-loop system \added{achieves convergence.} % because of the switching logic.
Note, however, that $V\added{(z)}$ increases significantly 
after $t=1\sec$ and again after $t=3\sec$.
The increase is limited by $v,$ 
but a smaller increase may be desirable. 
To reduce the amount that $V\added{(z)}$ can increase,
a larger value of $\vdiffcoeff$ should be chosen, 
causing $v$ to follow $V\added{(z)}$ more closely as $V\added{(z)}$ decreases.
\end{example}

% \begin{example}[Frequent switching due to poor $\kappa_1$ design]
% \label{ex:many switches}
% This example illustrates how the choices 
% of $\sigmazero$ and $\sigmaone$ affect 
% the frequency of switching in 
% the closed-loop system when $\kappa_1$ is 
% designed poorly.
% Consider the plant $\dot z = u$ 
% with $z, u \in \reals^2.$ 
% Let $\kappa_0(z) = \smallmat{-1 & 2 \\ -2 & -1} z$ 
% and let $\kappa_1(z) = \smallmat{0 \\ -10}.$ 
% The differential equation of the closed-loop 
% system with the controller $\kappa_1$
% is $\zdot = \smallmat{10 \\ 0}$, 
% so trajectories move in the positive $z_1$ direction.
% This controller does not help in stabilizing $\calA = \{0\}$, 
% but is intended to illustrate how the design of 
% our system regulates the frequency of switching.
% \Cref{fig:many switches} shows plots for a solution 
% to this system with parameters
% $\vdiffcoeff = 1,$
% $\sigmaone(s) := 0.5 s^2,$ and 
% $\sigmazero(s) := 0.6s^2 + 10^{-2}$
% and initial condition $z_0 = (5, 5),$ $v_0 = 10,$ and $q_0 = 1.$
% The solution experiences many switches around $t = 2\sec$ 
% due to the distance between $-\sigmazero(\distA{z})$ 
% and $-\sigmaone(\distA{z})$ being small and the controllers 
% $\kappa_0$ and $\kappa_1$ in some sense working against each other 
% (see \cref{fig:many switches plant}).
% The switching becomes less frequent and stops, however, 
% because $-\sigmazero$ is always less than $-10^{-2}$ 
% whereas $\dot V_1$ trends upwards toward zero.
% If we wish to decrease the maximum frequency of switches, 
% the gap $\sigmazero$ and $\sigmaone$ can be increased.
% % This can either be done by making $\sigmaone$ smaller 
% % or $\sigmazero$ larger.
% If, to the contrary, we pick $\sigmazero(s) := 0.6s^2$ 
% (violating the requirement that $\sigmazero(0) > \sigmaone(0) = 0$), then
% Zeno behavior occurs as solutions approach the origin,
% highlighting the importance that $\sigmazero$ 
% is strictly greater than $\sigmaone$.
% \begin{figure}[htbp]
%     \centering
%     \includegraphics{many_switches}
%     \caption{The solution in \cref{ex:many switches} experiences many switches, 
%     partially due to the small distance between $\sigma_0$ and $\sigma_1.$}
%     \label{fig:many switches} 
% \end{figure}
% % \todo{We don't include a plot of $V(z)$ and $v$ 
% % because it is hard to read and doesn't add much to the example.}
% \begin{figure}[htbp]
%     \centering
%     \includegraphics{many_switches_plant_state}
%     \caption{Phase plot for the solution in \cref{ex:many switches}. 
%     The controllers $\kappa_0$ and $\kappa_1$ push the state in 
%     opposite directions, producing frequent---but non-Zeno---switching.}
%     \label{fig:many switches plant}
% \end{figure}
% \end{example}

% \reportonly{
% \begin{example}
%     \label{ex:multiple switches}
% Consider the plant $\dot z = \fp(z, u) = u$ 
% with $z, u \in \reals^2$. Let $\kappa_0(z) = K_0z$ 
% where $K_0 = \smallmat{-1 & 2 \\ -2 & -1}$ and let $\kappa_1(z) = K_1x$ where 
% $K_1 = c \smallmat{-1 & 0 \\ -1 & -0.1}$ with $c > 0.$ 
% The eigenvalues of $K_1$ are $\lambda = -1 \pm 2i,$ so solutions to 
% $\dot z = \fp(z, \kappa_1(z))$ spiral toward the origin.
% On the other hand, the eigenvalues of $K_1$ are 
% $\left\{-0.1c, -c\right\},$
% producing a faster direction of travel and a slower direction of travel 
% for solutions to $\zdot = \fp(z, \kappa_1(z)).$
% As a result, in solutions to the closed-loop system $\calH,$ 
% we expect multiple switches back and forth 
% between $q=0$ and $q=1$ depending on variations in the strength of $\kappa_1.$  
% Let $\sigma_0$ and $\sigma_1$ be defined as in \cref{ex:languishing} 
% with $\gamma = 10,$ $\epsilonv = 1,$ and $\vdiffcoeff = 3.$
% \Cref{fig:multiple_switches} shows plots for a solution from initial state 
% $z_0 = (10, 0),$ $v_0 = 10,$ and $q_0 = 1.$
% The solution is controlled by $\kappa_1$ until $\dot V_1$ 
% becomes greater than $-\sigmaone(\distA{z})$ at $t = 0.2\sec.$ 
% The feedback $\kappa_0$ is then applied until $\dot V_1$ 
% drops to $-\sigmazero(\distA{z}).$ 
% In the next interval, the value of $V$ initially falls quickly, 
% so $v$ becomes much larger than $V.$ 
% Consequently, a switch does not occur immediately when $\dot V_1$ rises above 
% $-\sigmaone(\distA{z}),$ but instead happens later at $t = 3.7\sec$, 
% when $v$ decreases enough to reach $V.$
% After this point, the solution never switches again because $\dot V_1$ 
% continues trending upward whereas $-\sigmazero(\distA{z})$ 
% approaches $-\epsilonv = -1$ from below.
% \begin{figure}[htbp]
%     \centering
%     \includegraphics{multiple_switches}
%     \caption{The solution in \cref{ex:multiple switches} switches 
%     between $q = 0$ and $q=1$ several times.}
%     \label{fig:multiple_switches}
% \end{figure}
% \end{example}
% }

\section{Conclusion}
Future work includes analyzing our hybrid control 
strategy when applied to systems with disturbances 
and noisy measurements and inputs.
We are also interested in integrating our strategy
with existing supervisory control strategies that ensure
constraint satisfaction by switching between 
a primary controller that is not provably safe 
and a backup controller with safety guarantees
to create a hybrid closed-loop system 
that is provably safe and convergent.

%%% BIBLIOGRAPHY

\bibliography{biblio_report,biblio}

\reportonly{%
\appendix
\section{Appendix}
\label{sec:appendix}
    \begin{proof}[Proof of \cref{lem:all maximal are complete}] 
        Our proof uses \cite[Proposition 2.34]{sanfelice_hybrid_2021}. 
        A point $x_0 \in C\union D$ is 
        said to satisfy the viability condition (VC) if there exists a neighborhood $\calU$
        of $x_0$ such that $f(x') \in T_C(x')$ for every $x' \in \calU \intersect C$.
        If $\calH$ satisfies \cref{assump:hbc} and (VC) holds everywhere in $C \setminus D,$ 
        then every maximal solution $x$ to 
        $\calH$ satisfies exactly one of the following conditions:
        \begin{enumerate}
            \item the norm of $x$ goes to infinity in finite time; \label{item:finite time blowup}
            \item $x$ leaves $C\union D;$ or \label{item:leaves C union D}
            \item $x$ is complete. \label{item:complete}
        \end{enumerate} 
    
        To prove that (VC) is satisfied in $C\setminus D$, 
        we first show that $C\setminus D = \interior C.$ 
        The sets $C_0$ and $C_1$ are disconnected and form a partition of $C,$ 
        so $\bnd C = \bnd C_0 \union \bnd C_1.$
        For all $x_0 \in \bnd C_0,$ 
        $$x_0 \in \{x \in \calX \mid \dot{V}_1(z) = -\sigmazero(\distA{z})\} \subset D_0.$$
        Similarly, for every $x_1 \in \bnd C_1,$  
        $$x_1 \in \{x \in \calX \mid \dot{V}_1(z) = -\sigmaone(\distA{z})\} \subset D_1.$$
        Thus, $\bnd C \subset D,$ so $\interior C = C \setminus D.$ 
        On the interior of a set, the tangent cone is equal to the
        tangent cone of the entire space,
        therefore $T_C(x) = T_{\calX}(x)$ for all $x \in C\setminus D.$ 
        It follows that for every $x_0 \in C\setminus D,$ 
        there exists a neighborhood $\calU$ 
        of $x_0$ such that $f(x) \in T_C(x)$ 
        for all $x \in\calU \intersect C.$ 
        Therefore, (VC) holds everywhere in $C \setminus D$. 
    
        Additionally, by \cref{lem:hbc}, $\calH$ satisfies \cref{assump:hbc}, 
        so we can apply \cite[Proposition 2.34]{sanfelice_hybrid_2021} 
        to determine the characteristics of maximal solutions.
        By \cref{lem:UGS}, every solution is bounded, 
        ruling out \cref{item:finite time blowup}.
        Furthermore, $g(D) \subset C \union D = \calX,$ so, 
        per the note in \cite[Proposition 2.34]{sanfelice_hybrid_2021}, 
        \cref{item:leaves C union D} does not occur.
        By elimination, only \cref{item:complete} is possible, 
        therefore every maximal solution is complete.
    \end{proof}
}