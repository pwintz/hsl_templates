\usepackage{bbm}
\usepackage{xspace}

% =====================
% ======= TEXT ========
% =====================
%
\newcommand{\textfor}{\textup{for }}%
\newcommand{\textif}{\textup{if }}%
\newcommand{\textotherwise}{\textup{otherwise\xspace}}%
\newcommand{\otherwise}{\textotherwise}%
\newcommand{\midand}{\quad \textup{and}\quad}%
\newcommand{\midor}{\quad \textup{or}\quad}%
% \renewcommand {\if}{&\quad \text{if }}%
\newcommand{\st}{\ensuremath{^{st}}}%
\newcommand{\nd}{\ensuremath{^{nd}}}
\newcommand{\rd}{\ensuremath{^{rd}}}
\renewcommand{\th}{\ensuremath{^{th}}} % for ith, jth, etc.
\newcommand*{\circled}[1]{\raisebox{.5pt}{\textcircled{\raisebox{-.9pt} {#1}}}} % Creates a number enclosed in a circle
\newcommand{\lhs}{left-hand side }
\newcommand{\rhs}{right-hand side }
\newcommand{\wrt}{with respect to }
\newcommand{\Matlab}{\textsc{Matlab}\xspace}
%
% =============================
% ======= GENERAL MATH ======== 
% =============================
%
\newcommand*{\invs}{^{-1}}
\NewDocumentCommand{\hypt}{s m m}{
	% \hypt{a}{b} generates sqrt(a^2 + b^2). 
	% \hypt*{a}{b} puts parentheses around a and b.
    \IfBooleanTF #1%
        % Star=Parentheses.
        {\sqrt{\left(#2\right)^2 + \left(#3\right)^2}}%
        % No star=No Parentheses.
        {\sqrt{#2^2 + #3^2}}%
   }
\newcommand{\definedas}{\equiv}
\newcommand{\identifyas}{\equiv}
\newcommand{\iso}{\cong}
\newcommand{\evaluate}{\bigg\rvert}
\newcommand{\at}{\bigg\rvert}
\newcommand*{\evaluateat}[1]{\bigg\rvert_{#1}}
\newcommand*{\evaluatefromto}[2]{\bigg\rvert_{#1}^{#2}}
\newcommand{\dom}{\operatorname{dom}}
% 
% Create a system of equations.
% Usage:
% \system{x + y &= 2 \\ x - y &= 0}
\newcommand{\system}[1]{\left\{
    \begin{aligned}
        #1
    \end{aligned}\right.}
\newcommand{\neginfty}{{-\infty}}
%
% **** FUNCTIONS ****
\renewcommand{\arctan}{\tan\invs}
\newcommand{\atan}{\arctan}
\newcommand{\atantwo}{\mathtt{atan2}}
\renewcommand{\arcsin}{\sin\invs}
\newcommand{\asin}{\arcsin}
\renewcommand{\arccos}{\cos\invs}
\newcommand{\acos}{\arccos}
\newcommand{\abs}[1]{\left|#1\right|}
\newcommand{\absq}[1]{\abs{#1}^2}
%
% **** COMPLEX **** 
\newcommand{\real}{\,\mathfrak{Re}}
\newcommand{\imag}{\,\mathfrak{Im}}
\newcommand{\conj}[1]{\overline{#1}}
%
% ==========================
% ======= Fractions ======== 
% ==========================
%
\newcommand{\fracshort}[2]{\left.#1 \:\middle/\: #2\right.}
\newcommand{\oneover}[1]{\frac{1}{#1}}
\newcommand{\doneover}[1]{\dfrac{1}{#1}}
\newcommand{\oneovershort}[1]{\fracshort{1}{#1}}
\newcommand{\half}[1][1]{\frac{#1}{2}}
\newcommand{\halfshort}[1][1]{\fracshort{#1}{2}}
\newcommand{\third}[1][1]{\frac{#1}{3}}
\newcommand{\thirdshort}[1][1]{\fracshort{#1}{3}}
\newcommand{\quarter}[1][1]{\frac{#1}{4}}
\newcommand{\quartershort}[1][1]{\fracshort{#1}{4}}
\newcommand{\fifth}[1][1]{\frac{#1}{5}}
\newcommand{\fifthshort}[1][1]{\fracshort{#1}{5}}
\newcommand{\sixth}[1][1]{\frac{#1}{6}}
\newcommand{\sixthshort}[1][1]{\fracshort{#1}{6}}
\newcommand{\twelfth}[1][1]{\frac{#1}{12}}
\newcommand{\twelfthshort}[1][1]{\fracshort{#1}{12}}
%
% ===========================
% ======= SET THEORY ======== 
% ===========================
%
% Set name
\newcommand{\setname}{\mathcal} 
% Define a set in the form {<arg 1>|<arg 2>}
% \newcommand{\suchthat}{\mathrel{} \middle| \mathrel{} } % Alternative: \mid
\newcommand{\suchthat}{\mathrel{}\ifnum\currentgrouptype=16 \middle\fi|\mathrel{}}
\newcommand*{\setdef}[1]{\left\{#1 \right\}} 
\newcommand*{\closure}[1]{\overline{#1}}
\newcommand{\interior}{\operatorname{int}}
\newcommand{\union}{\cup}
\newcommand{\Union}{\bigcup}
\newcommand{\inter}{\cap}
\newcommand{\Inter}{\bigcap}
\newcommand{\intersect}{\inter}
\newcommand{\Intersect}{\Inter}
\newcommand{\boundary}{\partial}
\newcommand{\bnd}{\boundary}
%
% **** SETS **** 
\newcommand{\reals}{\mathbb{R}}
\newcommand{\realsn}{\reals^{n}}
\newcommand{\positivereals}{\reals_{>0}}
\newcommand{\preals}{\positivereals}
\newcommand{\nonnegativereals}{\reals_{\geq0}}
\newcommand{\nnreals}{\nonnegativereals}
\newcommand{\rationals}{\mathbb{Q}}
\newcommand{\integers}{\mathbb{Z}}
\newcommand{\naturals}{\mathbb{N}}
\newcommand{\complexes}{\mathbb{C}}
\newcommand*{\squarematrices}[1][n]{\reals^{#1 \times #1}}
\newcommand*{\symmetricmatrices}[1][n]{\mathbb{S}^{#1}}
\newcommand*{\pdmatrices}[1][n]{\symmetricmatrices[#1]_{++}}
\newcommand*{\psdmatrices}[1][n]{\symmetricmatrices[#1]_{+}}
\newcommand*{\unitball}[1][n]{\mathbb{B}^{#1}}
\newcommand*{\sphere}[1][n]{\mathbb{S}^{#1}}
\newcommand*{\torus}[1][n]{\mathbb{T}^{#1}}
\newcommand*{\realprojective}[1][n]{\reals \mathbb{P}^{#1}}
\newcommand*{\complexprojective}[1][n]{\complexes\mathbb{P}^{#1}}
%
% =============================
% ====== LINEAR ALGEBRA =======
% ============================= 
%
% **** VECTORS **** 
% If a value is not a vector don't use the "\"
\renewcommand*{\vec}[1]{\bm{\mathrm{#1}}} % Vector variable
\newcommand{\x}{{\vec{x}}}
\newcommand{\y}{{\vec{y}}} 
\newcommand{\z}{{\vec{z}}}
\renewcommand{\r}{{\vec{r}}}
\renewcommand{\u}{{\vec{u}}}
\renewcommand{\v}{{\vec{v}}}
\renewcommand{\a}{{\vec{a}}}
\renewcommand{\b}{{\vec{b}}}
\newcommand{\0}{\vec 0}
\newcommand{\onevec}{\mathbb{1}}
\newcommand{\RRF}{\operatorname{RRF}}
\newcommand{\RREF}{\operatorname{RREF}}
\newcommand*{\proj}[1]{\operatorname{proj}_{#1}}
\newcommand*{\comp}[1]{\operatorname{comp}_{#1}}
\newcommand*{\orth}[1]{\operatorname{orth}_{#1}}
%
% **** VECTOR AND MATRIX FUNCTIONS **** 
\newcommand*{\innerproduct}[2]{\left\langle #1, #2 \right\rangle}
\newcommand*{\ip}[2]{\innerproduct{#1}{#2}}
\newcommand*{\norm}[1]{\left\lvert#1\right\rvert} % double bars "\lVert ... \rVert" are used for signal norms so we use singal bars for euclidean vector norms.
\newcommand*{\normsq}[1]{\norm{#1}_{1}^2}
\newcommand*{\onenorm}[1]{\norm{#1}_{1}}
\newcommand*{\twonorm}[1]{\norm{#1}_{2}}
\newcommand*{\pnorm}[1]{\norm{#1}_{p}}
\newcommand*{\frobeniusnorm}[1]{\norm{#1}_{F}}
\newcommand*{\inftynorm}[1]{\norm{#1}_{\infty}}
\newcommand{\diag}{\operatorname{diag}}
\newcommand{\rank}{\operatorname{rank}}
\newcommand{\trace}{\operatorname{tr}}
%DON'T USE: \renewcommand {\null}{\text{null}}
\newcommand{\range}{\operatorname{range}}
\newcommand{\trans}[1][]{^{#1 \top}} % Matrix transpose
\newcommand{\transstar}{^{*\top}} % Vector or matrix transpose notation (\intercal is an alternative)
\newcommand{\herm}[1][]{^{#1 H}} % Hermitian 
\newcommand{\eig}{\operatorname{eig}}
%
% **** MATRICES **** 
% Create a large matrix. Usage \mat[<vertical spacing>]{a & b \\ c & d}
\newcommand{\mat}[2][1]{\begingroup
	\renewcommand*{\arraystretch}{#1}
	\begin{bmatrix}#2\end{bmatrix}
\endgroup}
% Create a small matrix suitable to use in the middle of text. 
% Usage \smallmat{a & b \\ c & d}
\newcommand*{\smallmat}[1]{\bigl[ \begin{smallmatrix} #1 \end{smallmatrix} \bigr]}
\newcommand{\diagmatrix}[2]{
\mat{
{#1} & & \\
& \ddots & \\
& & {#2}
}}
\newcommand{\tridiagmatrix}[3]{
\mat{
#2     & #3     &         \\
#1     & \ddots & \ddots  \\
       & \ddots & \ddots & #3 \\
       &        & #1     & #2
}}
\newcommand{\rotationmat}[1]{\mat{\cos{#1} & \sin{#1} \\ 
                                  -\sin{#1} & \cos{#1} }}
%
% =========================
% ======= CALCULUS ========
% =========================
%
\newcommand{\diff}{\mathop{}\!d}
\newcommand{\dt}{\diff t}
\newcommand{\dx}{\diff x}
\newcommand{\dy}{\diff y}
\newcommand{\dz}{\diff z}
\newcommand*{\derivative}[2][]{\frac{d#1}{d#2}}
\newcommand*{\derivativeshort}[2][]{\fracshort{d#1}{d#2}}
\newcommand*{\dd}[2][]{\derivative[#1]{#2}} % Shortcut for \derivative
\newcommand*{\ddshort}[2][]{\derivativeshort[#1]{#2}}
\newcommand*{\partialderivative}[2][]{\frac{\partial{#1}}{\partial{#2}}}
\newcommand*{\partialderivativeshort}[2][]{\fracshort{\partial{#1}}{\partial{#2}}}
\newcommand*{\pd}[2][]{\partialderivative[#1]{#2}}
\newcommand*{\pdshort}[2][]{\partialderivativeshort[#1]{#2}}
\newcommand*{\pdx}[1][]{\pd[#1]{x}}
\newcommand*{\pdy}[1][]{\pd[#1]{y}}
\newcommand*{\pdz}[1][]{\pd[#1]{z}}
\newcommand*{\ddx}[1][]{\derivative[#1]{x}}
\newcommand*{\ddy}[1][]{\derivative[#1]{y}}
\newcommand*{\ddz}[1][]{\derivative[#1]{z}}
\newcommand*{\ddt}[1][]{\derivative[#1]{t}}
\newcommand*{\dds}[1][]{\derivative[#1]{s}}
\newcommand{\dxdt}{\derivative[x]{t}}
\newcommand{\dydt}{\derivative[y]{t}}
\newcommand{\dfdx}{\derivative[f]{x}}
\newcommand{\dxdy}{\derivative[x]{y}}
\newcommand{\xdot}{\dot{x}}
\newcommand{\xddot}{\ddot{x}}
\newcommand{\ydot}{\dot{y}}
\newcommand{\yddot}{\ddot{y}}
\newcommand{\zdot}{\dot{z}}
\newcommand{\zddot}{\ddot{z}}
\newcommand{\rdot}{\dot{r}}
\newcommand{\rddot}{\ddot{r}}
\newcommand{\thetadot}{\dot{\theta}}
\newcommand{\thetaddot}{\ddot{\theta}}
\newcommand{\omegadot}{\dot{\omega}}
\newcommand{\omegaddot}{\ddot{\omega}}
\newcommand*{\secondderivative}[2][]{\frac{d^2{#1}}{d{#2}^2}}
\newcommand{\del}{\nabla}
\newcommand{\grad}{\nabla}
\newcommand{\laplace}{\del^2}
\newcommand{\hessian}{\del^2}
\newcommand{\Jacobian}[1]{\mat{
    \partialderivative[#1_1]{x_1} 
        & \cdots 
        & \partialderivative[#1_1]{x_n} \\
    \vdots & \ddots & \vdots \\
    \partialderivative[#1_n]{x_1} 
        & \cdots 
        & \partialderivative[#1_n]{x_n} }}
\newcommand{\Hessian}[1]{\mat{
    \partialderivative[^2#1]{x_1^2} 
        & \cdots 
        & \partialderivative[^2#1]{x_1\partial x_n} \\
    \vdots & \ddots & \vdots \\
    \partialderivative[^2#1]{x_n\partial x_1} 
        & \cdots 
        & \partialderivative[^2#1]{x_n^2} }}
% Exponent Taylor Series Definition
\newcommand*{\taylorexponent}[1]{\sum^\infty_{k=1} \frac{#1^k}{k!}}
%
% **** TRANSFORMS ****
\newcommand{\Lagrangian}{\mathcal{L}}
\newcommand*{\Fourier}[1][*]{ \mathcal{F}\left[\,#1\,\right] }
\newcommand*{\Fourierinvs}[1][*]{ \mathcal{F}\invs\left[\,#1\,\right] }
\newcommand*{\Laplace}[1][*]{ \mathcal{L}\left[\,#1\,\right] }
\newcommand*{\Laplaceinvs}[1][*]{ \mathcal{L}\invs\left[\,#1\,\right] }
%
% =============================
% ======= OPTIMIZATION ========
% =============================
%
% Example usage: \convexcombo[\alpha]{x}{y}
\newcommand*{\convexcombo}[3][\theta]{{#1}{#2} + \left(1 - {#1}\right){#3}}
\newcommand{\convex}{\operatorname{conv}}
\newcommand*{\minimize}[1]{\underset{#1}{\textup{minimize}}\quad} % Usage: \minimize{\x \in \reals}
\newcommand*{\maximize}[1]{\underset{#1}{\textup{maximize}}\quad} % Usage: \maximize{\x \in \reals}
\newcommand{\subjectto}{\textup{subject to}\quad}
\newcommand{\epi}{\operatorname{epi}}
\newcommand*{\argmax}[1]{\underset{#1}{\operatorname{argmax}}\:\:}
\newcommand*{\argmin}[1]{\underset{#1}{\operatorname{argmin}}\:\:}
%
% ===============================
% ======= Control Theory ========
% ===============================
%
\newcommand{\Kinfty}{\calK_{\infty}}
\newcommand{\KL}{\mathcal{KL}}
\newcommand{\KLL}{\mathcal{KLL}}
% 
%
% ===========================================
% ======= PROBABILITY AND STATISTICS ========
% ===========================================
%
\newcommand{\expectedvalue}{\mathbb{E}}
\newcommand{\E}{\expectedvalue}
\newcommand*{\Egiven}[2]{\E\left(#1 \suchthat #2\right)}
\renewcommand{\P}{\mathbb{P}}
\renewcommand*{\choose}[2]{_{#1}C_{#2}}
% Produces P(A|B)
\newcommand{\Pgiven}[2]{\P\left(#1 \suchthat #2\right)}
% Produces P(A intesect B) / P(B)
\newcommand{\Pgivenformula}[2]{\frac{\P(#1 \cap #2)}{\P(#2)}}
\newcommand{\Var}{\operatorname{Var}}
\newcommand{\Vargiven}[2]{\Var\left(#1 \suchthat #2\right)}
\newcommand{\Cov}{\operatorname{Cov}}
\newcommand{\Covgiven}[2]{\Cov\left(#1 \suchthat #2\right)}
\newcommand{\pdf}{p.d.f. }
\newcommand{\cdf}{c.d.f. }
%
% ===========================================
% ======= REAL ANALYSIS ========
% ===========================================
%
\newcommand{\limittoinfy}[1][n]{\lim_{#1 \to \infty}}
\newcommand{\sumtoinfty}[1][n]{\sum_{#1=1}^{\infty}}
%
% ===========================================
% ======= Manifolds ========
% ===========================================
%
\newcommand{\Lie}{\mathscr{L}}
%
% =============================
% ===== PROGRESS TRACKING =====
% =============================
%
\newcommand{\boxedeq}[1]{\begin{empheq}[box={\fboxsep=6pt\fbox}]{align}#1\end{empheq}}

% Misc
\newcommand{\xaxis}{$x$-axis }
\newcommand{\yaxis}{$y$-axis }
\newcommand{\zaxis}{$z$-axis }

\newcommand{\xvalue}{$x$-value }
\newcommand{\yvalue}{$y$-value }
\newcommand{\zvalue}{$z$-value }

\newcommand{\xcoord}{$x$-coordinate }
\newcommand{\ycoord}{$y$-coordinate }
\newcommand{\zcoord}{$z$-coordinate }

% \usepackage[dvipsnames]{xcolor}
\definecolor{darkgreen}{rgb}{0.0, 0.5, 0.0}
\definecolor{darkred}{rgb}{0.8, 0.0, 0.0}

\newcommand{\red}[1]{{\color{red}#1}}
\newcommand{\green}[1]{{\color{green}#1}}
\newcommand{\blue}[1]{{\color{blue}#1}}
\newcommand{\darkred}[1]{{\color{darkred}#1}}
\newcommand{\darkgreen}[1]{{\color{darkgreen}#1}}
\newcommand{\update}[1]{\blue{#1}}
\newcommand{\updatetwo}[1]{\green{#1}}
\newcommand{\delete}[1]{\darkred{#1}}
\newcommand{\macrocolor}[1]{{\ifdraft{\color{magenta}#1}{#1}}}
\newcommand{\correction}[1]{{\color{purple}#1}}

% Iterations
\newcommand{\I}{\textrm{I}}
\newcommand{\II}{\textrm{II}}
\newcommand{\III}{\textrm{III}}
\newcommand{\IV}{\textrm{IV}}
\newcommand{\V}{\textrm{V}}

% Specific to this paper
% Caligraphy letters
\newcommand{\calA}{{\mathcal{A}}}
\newcommand{\calB}{{\mathcal{B}}}
\newcommand{\calH}{{\mathcal{H}}}
\newcommand{\calK}{{\mathcal{K}}}
\newcommand{\calS}{{\mathcal{S}}}
\newcommand{\calT}{{\mathcal{T}}}
\newcommand{\calU}{{\mathcal{U}}}
\newcommand{\calX}{{\mathcal{X}}}
\newcommand{\calY}{{\mathcal{Y}}}
\newcommand{\calZ}{{\mathcal{Z}}}
\newcommand{\calAX}{\macrocolor{\widetilde{\calA}}}

% Zero-vector
\newcommand{\zerovec}[1]{0_{#1}}

\newcommand{\sigmaone}{\macrocolor{\sigma_1}}
\newcommand{\sigmazero}{\macrocolor{\sigma_0}}
\newcommand{\epsilonv}{\macrocolor{\varepsilon_v}}
\newcommand{\taudot}{\dot{\tau}}
\newcommand{\vdot}{\dot{v}}
\newcommand{\qdot}{\dot{q}}
\newcommand{\basin}{\macrocolor{\calB_{\calA}}}
\newcommand{\vdiffcoeff}{\macrocolor{\mu}}

%  From https://tex.stackexchange.com/questions/161050/is-there-a-painless-way-to-get-self-loop-decorations-on-mathematical-symbols/161060?noredirect=1#comment1489152_161060
\usepackage{tikz}
\usepackage{stackengine}
\newcommand{\LoopFunction}[1]{%
  \renewcommand\stacktype{L}%
  \renewcommand\useanchorwidth{T}%
  \renewcommand\stackalignment{l}%
  \setbox0\hbox{$#1$}%
  \stackon[0pt]{\usebox0}{\hspace{\the\wd0}\hspace{-1.6ex}%
    \tikz \node {} edge [in=80,out=30,loop] node {} ();}%
}

\newcommand{\plantstatespace}{\macrocolor{\reals^\nplant}}
\newcommand{\plantinputspace}{\macrocolor{\reals^\mplant}}
\newcommand{\calZswitchtoone}{\macrocolor{\calZ_{0\mapsto 1}}}
\newcommand{\calZholdzero}{\macrocolor{\calZ_{0}}}
\newcommand{\calZholdone}{\macrocolor{\calZ_{1}}}
\newcommand{\calZswitchtozero}{\macrocolor{\calZ_{1\mapsto 0}}}

% Distance function
\newcommand{\distA}[1]{\macrocolor{\abs{#1}_{\calA}}}
\newcommand{\distAX}[1]{\macrocolor{\abs{#1}_{\calAX}}}

% Dynamics
\newcommand{\fp}{\macrocolor{f_P}}
\newcommand{\fv}{\macrocolor{f_v}}
\newcommand{\ftau}{\macrocolor{f_\tau}}
\newcommand{\nplant}{\macrocolor{n}}
\newcommand{\mplant}{\macrocolor{m}}

% This can be set to either contain text "or" or the logic symbol \vee.
\newcommand{\OR}{\textup{ or }}
\newcommand{\AND}{,}

% Set-specific subscripts
\newcommand{\sszero}{^{\macrocolor{0}}}
\newcommand{\ssone}{^{\macrocolor{1}}}
\newcommand{\ssb}{^{\macrocolor{b}}}

% Generalized derivatives
\newcommand{\gdd}[1]{{#1}^\circ} % directional derivative
\newcommand{\ggrad}{\macrocolor{\partial^\circ}} % gradient

\newcommand{\tangentcone}[1]{\macrocolor{T}_{#1}}

\newcommand{\proofsketch}{\par\textit{Proof sketch.} }

\renewcommand{\sec}{\operatorname{s}}
\newcommand{\msec}{\operatorname{ms}}

\newcommand{\zzero}{z_0}
\newcommand{\zone}{z_1}

% Text
\newcommand{\UGAS}{GAS\xspace}

\newcommand{\VonedotDefinition}{
  \begin{center}
    \begin{tcolorbox}[colback=white,colframe=structure,box align=center, width=0.5\linewidth]
      Let $\dot V_{1}(z) := \ip{\del V(z)}{\fp(z, \qcolor{1}{\kappa_{1}}(z))}.$\qquad
    \end{tcolorbox}
  \end{center}
  \smallskip 
}
\newcommand{\calZholdonedef}{\{z \in \plantstatespace \mid \dot V_1(z) \leq -\sigmaone(\distA{z})\}}
\newcommand{\calZswitchtozerodef}{\{z \in \plantstatespace \mid \dot V_1(z) \geq -\sigmaone(\distA{z})\}}

\newcommand{\tV}{\macrocolor{\widetilde{V}}}
\newcommand{\tVc}{\macrocolor{\gdd{\widetilde{V}}}}
\newcommand{\eqss}{{eq}}
\newcommand{\neqss}{{neq}}
\newcommand{\xneq}{\macrocolor{x_\neqss}}
\newcommand{\zneq}{\macrocolor{z_\neqss}}
\newcommand{\vneq}{\macrocolor{v_\neqss}}
\newcommand{\qneq}{\macrocolor{q_\neqss}}
\newcommand{\xeq}{\macrocolor{x_\eqss}}
\newcommand{\zeq}{\macrocolor{z_\eqss}}
\newcommand{\veq}{\macrocolor{v_\eqss}}
\newcommand{\qeq}{\macrocolor{q_\eqss}}
\newcommand{\flowdir}{\macrocolor{w}}
% \renewcommand{\calAX}

