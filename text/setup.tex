% LTeX: enable=true
% Create \ifdraft{}{} conditional that switches based on whether "draft" is passed to document class.
\usepackage{ifdraft} 
\usepackage[utf8]{inputenc}
\usepackage{amsmath}
\usepackage{amssymb}
\usepackage{amsfonts}
\usepackage{mathrsfs} % Enables \mathscr{}

% ==============
% === Layout ===
% ==============
\usepackage{multicol} % Allows enumerations over two columns
\usepackage{float}
% \usepackage[bottom]{footmisc} % Forces footnotes to be at the very bottom of the page. https://tex.stackexchange.com/a/9427/153678

% ==================
% === Formatting ===
% ==================
\usepackage{tcolorbox}
\usepackage{xcolor}
\usepackage{color}
\usepackage{verbatim} 
\usepackage{markdown}
\usepackage{bm} % Allows bold Greek letters.
\usepackage{xspace} % To fix spacing after macros. see https://tex.stackexchange.com/a/17731/153678
% allows enumerate with letters, i.e. (a), (b), (c).
%\usepackage[shortlabels]{enumitem} 

% ===============
% === SPACING ===
% ===============

% Format paragraphs. 
\makeatletter%
\@ifclassloaded{article}{%
    \setlength{\parindent}{0em}
    \setlength{\parskip}{0.8em}
}

%%%% Fix Spacing Around "\left(\right)" %%%
\usepackage{mleftright} 
\mleftright % redefine \left as \mleft and \right as \mright.

% Set the line spacing to single-spaced in the "cases" environment. 
% Source: https://tex.stackexchange.com/a/555622/153678
\usepackage{etoolbox}
\AtBeginEnvironment{cases}{\renewcommand\baselinestretch{1}\selectfont}

% Command to adjust the horizontal spacing 
%   By Ryan Johnson. 
% 
% Example usage:
% \begin{equation}
% \mbox{$\squeezespaces{0.5}
% \begin{aligned}
%     a^2 + b^2 = c^2
% \end{aligned}
% $}
% \end{equation}
\newcommand*\squeezespaces[1]{% <- #1 is a number between 0 and 1
  \thickmuskip=\scalemuskip{\thickmuskip}{#1}%
  \medmuskip=\scalemuskip{\medmuskip}{#1}%
  \thinmuskip=\scalemuskip{\thinmuskip}{#1}%
  \nulldelimiterspace=#1\nulldelimiterspace
  \scriptspace=#1\scriptspace
}
\newcommand*\scalemuskip[2]{%
  \muexpr #1*\numexpr\dimexpr#2pt\relax\relax/65536\relax
}
% === END OF SPACING ===

% Support adding "todo" comments.
\usepackage[colorinlistoftodos,prependcaption,textsize=small,textcolor=white]{todonotes}
\setlength{\marginparwidth}{2cm} % This doesn't appear to do anything, except suppress a warning related to todonotes.

%% Improved LaTeX definitions %%
\usepackage{environ}
\usepackage{xparse} % Enhanced command definitions.

%% Misc/Uncategorized Packages %%
\usepackage{comment} % enables the use of multi-line comments (\ifx \fi) 
\usepackage{cancel} % For crossing out parts of equations.
\usepackage{empheq} % For putting boxes around equations
\usepackage{mathtools}
\usepackage{pdfpages} % Allows use of \includepdf[page={page number}]{filename}
% \usepackage{siunitx} % SI Units used as '\SI{60}{\mile\per\hour} https://tex.stackexchange.com/a/509965/153678

% =========================
% === CONDITIONAL LOGIC ===
% =========================
\usepackage{etoolbox} % Used for LaTeX conditionals. See: https://tex.stackexchange.com/a/5896/153678.
\newtoggle{solutions}

% Define 'switch' environment. 
% See https://tex.stackexchange.com/a/187710/153678
\usepackage{xifthen}
\newcommand{\ifequals}[3]{\ifthenelse{\equal{#1}{#2}}{#3}{}}
\newcommand{\case}[2]{#1 #2} % Dummy, so \renewcommand has something to overwrite...
\newenvironment{switch}[1]{\renewcommand{\case}{\ifequals{#1}}}{}

% ============================
% === Figures and Graphics ===
% ============================
\usepackage{pdfpages} % Allows use of \includepdf[page={page number}]{filename}
\usepackage{rotating}% For rotating figures, tables, etc. See https://tex.stackexchange.com/a/50071/153678
\usepackage{blkarray} % allows labeling rows/columns of block matrix. 
% Setup captions
% \usepackage{caption,setspace} % needed to set caption line spacing. See: https://tex.stackexchange.com/a/186327/153678
\usepackage{caption}
% Allows removal of "Figure 1:" in caption when "\caption*{}" is used. 
% \captionsetup{font={stretch=1.2}}
\captionsetup{belowskip=-4pt}

%%%% Support subfigures %%%%
% The package 'subfigure' is deprecated. Now, either 'subfig' or 'subcaption'
% is recommend.
\usepackage{graphicx} 
% \usepackage{caption}
% \usepackage{subfig}
%% Example of how to use 'subfig' package:
% \begin{figure}
% \centering
% \begin{subfigure}{.5\textwidth}
%   \centering
%   \includegraphics[width=.6\linewidth]{}
% \end{subfigure}%
% \begin{subfigure}{.5\textwidth}
%   \centering
%   \includegraphics[width=.6\linewidth]{}
% \end{subfigure}
% \end{figure}

\usepackage{subcaption}
%% Example of how to use 'subcaption' package:
% \begin{figure*} % Star makes multi-column figure. 
%     \centering
%     \subcaptionbox{Subfigure (a) caption.\label{subfig a label}}
%     {\includegraphics{image_1}}
%     \qquad
%     \subcaptionbox{Subfigure (b) caption.\label{subfig b label}}
%     {\includegraphics{image_2}}
%     \caption{Figure caption.}
%     \label{figure label}
% \end{figure*}

% ===================
% === CODE BLOCKS ===
% ===================

\usepackage[final]{listings}

% Allows comments sections
\usepackage{verbatim} 

% \lstset{frame=tb,
%   language=Python,
%   aboveskip=3mm,
%   belowskip=3mm,
%   showstringspaces=false,
%   columns=flexible,
%   basicstyle={\small\ttfamily},
%   numbers=none,
%   numberstyle=\tiny\color{gray},
%   keywordstyle=\color{blue},
%   commentstyle=\color{dkgreen},
%   stringstyle=\color{mauve},
%   breaklines=true,
%   breakatwhitespace=true,
%   tabsize=3
% }

% ==============================
% === REFERENCES AND LINKING ===
% ==============================
% \let\inserttitle\@title
% \let\inserttitle\@title

\makeatletter
% Enable hyperlinks
\usepackage{hyperref}[implicit=true]
\hypersetup{
    colorlinks=false,
    linkcolor=blue,
    filecolor=magenta,      
    urlcolor=blue,     
    anchorcolor = blue,
    citecolor = blue,    
    % Set the metadata for the produced PDF.
    pdftitle={UGAS while Exploiting Uncertified Controller},
    pdfsubject={},
    pdfauthor={Paul K. Wintz, Ricardo G. Sanfelice, João P. Hespanha},
    pdfkeywords={Hybrid Dynamical Systems, Control Systems, Switching, Lyapunov Functions, Model Predictive Control}
}   
\makeatother

% Display Theorem/Lemma/Corollary (1) in cross references.  
% IT'S IMPORTANT TO LOAD THIS LAST AFTER ALL THE OTHER \usepackages.
\usepackage[noabbrev, capitalise]{cleveref} 
% Make ranges of references use an dash between the equation numbers. 
% See https://tex.stackexchange.com/a/18988/153678
\newcommand{\crefrangeconjunction}{--} 

% Configure Clever Reference
% Make equations referenced as "(1)" instead of "Equation 1"
\crefname{equation}{}{}
\crefformat{equation}{(#2#1#3)}
% \crefrangeformat{equation}{(#3#1#4) to~(#5#2#6)}
\crefmultiformat{equation}{(#2#1#3)}%
{ and~(#2#1#3)}{, (#2#1#3)}{ and~(#2#1#3)}

\newcommand{\setupAssumption}[1][A]{
    % Setup enumerations with the assumption environment to use enumerations 
    % (A1), (A2), etc. If the optional argument is provided, 
    % then the letter is modified.
    \renewcommand*{\theenumi}{(#1\arabic{enumi})}% From https://tex.stackexchange.com/a/11901/153678
    \renewcommand*{\labelenumi}{(#1\arabic{enumi})}%
}


% =================
% === REFERENCES ===
% =================

% Setup references for presentations

\makeatletter%
\@ifclassloaded{beamer}%
{% if using beamer
    % Import the natbib package and sets a bibliography and citation styles
    \usepackage[url = false,doi = false,isbn = false,natbib]{biblatex}
    % \usepackage[sorting=none,url=false, doi=false, isbn=false]{biblatex}
    \addbibresource{biblio.bib} %Import the bibliography file
    % \setcitestyle{authoryear,open={((},close={))}}
}
\makeatother%

% \usepackage[style=authortitle]{biblatex}

% Import the natbib package and sets a bibliography and citation styles
% \usepackage[url = false,doi = false, isbn = false]{biblatex}
% \usepackage{natbib}
% \bibliographystyle{abbrvnat}
% \setcitestyle{authoryear,open={((},close={))}}

% \addbibresource{Uniting Feedback Controller.bib} %Imports bibliography file
% \renewbibmacro*{cite:author}{%
%     \printtext[bibhyperref]{%
%         \structure{hello}
%         % \printfield[citeauthor]{labeltitle}%
%         % \setunit{\space}%
%         % \printtext[parens]{\printdate}%
%     }%
% }
% \renewbibmacro*{cite:title}{%
%     \printtext[bibhyperref]{%
%         \printfield[citetitle]{labeltitle}%
%         \setunit{\space}%
%         \printtext[parens]{\printdate}%
%     }%
% }

% =========================
% === EDITORIAL MARKUPS ===
% =========================

\ifdraft{
    \newenvironment{workingnotes}{\sffamily \color{violet}}{}
}{ 
    \newenvironment{workingnotes}
    {\expandafter\comment}
    {\expandafter\endcomment}
}

\ifdraft{
    \newcommand{\workingnote}[1]{\begin{workingnotes} #1 \end{workingnotes}}
}{ 
    \newcommand{\workingnote}[1]{}
}

\ifdraft{
    \newcommand {\finalonly}[1]{{\color{red}$<$Hidden$>$}}
}{
    \newcommand{\finalonly}[1]{#1}
}

% Progress tracking
\ifdraft{
    \newcommand{\done}{\ensuremath{\checkmark}} % Variant: \surd
    \newcommand{\undone}{\ensuremath{\square}}
    \newcommand{\doneish}{\done-ish}
}{
    \newcommand{\done}{}
    \newcommand{\undone}{}
    \newcommand{\doneish}{}
}

% Support tracking changes and adding "todo" comments.
% \usepackage[todonotes={colorinlistoftodos,prependcaption,textsize=small,textcolor=white},commandnameprefix=ifneeded]{changes}
\usepackage[commandnameprefix=ifneeded,
            final,
            todonotes={colorinlistoftodos,prependcaption,
                       textsize=small,backgroundcolor=orange!10,
                       textcolor=black,linecolor=orange,bordercolor=orange}]{changes}

% ============================================
% === ENVIRONMENTS (Theorems, lemmas, etc) ===
% ====================================f========

\makeatletter%
\@ifclassloaded{beamer}{% 
    \theoremstyle{definition}%
    \ifdefined\definition\else
        \newtheorem{definition}{Definition}
    \fi
    \ifdefined\problem\else
        \newtheorem{problem}{Problem}
    \fi
    \newtheorem{assumption}{Assumption}% Use singular even when there are multiple assumptions in a particular block.
    \crefname{assumption}{Assumption}{Assumptions}
    \theoremstyle{remark}%
    \ifdefined\remark\else
        \newtheorem{remark}{Remark}
    \fi
    \ifdefined\example\else
        \newtheorem{example}{Example}
    \fi
    % \newtheorem{example}{Example}%
    \theoremstyle{plain}%
    \ifdefined\theorem\else
        \newtheorem{theorem}{Theorem}
    \fi
    \newtheorem{proposition}{Proposition}
    \ifdefined\lemma\else
        \newtheorem{lemma}{Lemma}
        \crefname{lemma}{Lemma}{Lemmas}%
    \fi
    \ifdefined\corollary\else
        \newtheorem{corollary}{Corollary}
        \crefname{corollary}{corollary}{corollaries}%
    \fi
}
\makeatother%

\makeatletter%
\@ifclassloaded{ieeeconf}{% if using ieeeconf
    \IEEEoverridecommandlockouts
    \overrideIEEEmargins

    % Load amsthm package. The following "\relax"
    % commands prevent an error that say
    % "Command \proof already defined." when we load amsthm.
    \let\proof\relax 
    \let\endproof\relax
    \usepackage{amsthm}
    
    \theoremstyle{definition}%
    \newtheorem{definition}{Definition}
    \newtheorem{assumption}{Assumption}
    \crefname{assumption}{Assumption}{Assumptions}
    \newtheorem{example}{Example}

    \theoremstyle{plain}%
    \newtheorem{theorem}{Theorem}
    \newtheorem{proposition}{Proposition}
    \newtheorem{lemma}{Lemma}

    \theoremstyle{remark}
    \newtheorem{remark}{Remark}

    % Set the bibliography style.
    \bibliographystyle{ieeetr}
}{% if not using ieeeconf
    \newcommand{\authorblockN}[1]{#1}
    % The amsthm package provides {theorem}, {lemma} and {proof} environments
    \usepackage{amsthm}
    \usepackage[toc,page]{appendix}
}
\makeatother%

\makeatletter%
\@ifclassloaded{article}{%
    \usepackage{amsthm}
    \theoremstyle{definition}
    \newtheorem{definition}{Definition}
    \crefname{definition}{Definition}{Definitions}
    \newtheorem{assumption}{Assumption}
    \crefname{assumption}{Assumption}{Assumptions}
    \theoremstyle{remark}
    \newtheorem{remark}{Remark}
    \newtheorem{example}{Example}
    \theoremstyle{plain}
    \newtheorem{theorem}{Theorem}
    \newtheorem{proposition}{Proposition}
    \newtheorem{lemma}{Lemma}
    \crefname{lemma}{Lemma}{Lemmas}
    \newtheorem{corollary}{Corollary}
    \crefname{corollary}{corollary}{corollaries}
    \newtheorem{conjecture}{Conjecture}
    \crefname{conjecture}{conjecture}{conjectures}
}
\makeatother%

% =================================
% === BEAMER PRESENTATION STYLE ===
% =================================
\makeatletter%
\@ifclassloaded{beamer}{% 
    % Hide Roman numerals from continued frames with breaks
    \setbeamertemplate{frametitle continuation}{}

    % Define Beamer colors.
    \definecolor{lightblue}{RGB}{60,60,200}
    \setbeamercolor{block title}{bg=lightblue!10!white}
    \setbeamercolor{block body}{bg=lightblue!5!white}
    \setbeamercolor{block title alerted}{bg=blue!10!white}
    \setbeamercolor{block body alerted}{bg=blue!3!white}
    \setbeamerfont{block title}{size={}}
    \setbeamertemplate{bibliography item}{\insertbiblabel}
    % Select font
    \usefonttheme{serif}
    \setbeamerfont{footnote}{size=\tiny}
    \setbeamercolor{page number in head/foot}{fg=lightblue}
    \setbeamercolor{author in head/foot}{fg=lightblue}
    \setbeamertemplate{navigation symbols}{}
    \setbeamertemplate{theorems}[numbered]
    \setbeamertemplate{frametitle}[default][center]
    % ======== Table of Contents ========
    % Set the TOC colors
    \setbeamercolor{section in toc}{fg=structure}
    \setbeamercolor{subsection in toc}{fg=structure}
    % Define Table of Contents slide at the beginning of each section.
    \AtBeginSection[]{%
    \begin{frame}<beamer>
        \frametitle{Outline}
        % Use 'show'/'hide' to show/hide subsections, and 'shaded' to gray them out. 
        \tableofcontents[currentsection,
                         currentsubsection,
                         subsectionstyle=show/show/shaded]
    \end{frame}
    }%
    % 
    % Set the size of page numbers.
    \setbeamerfont{footline}{size=\small}
    \setbeamertemplate{footline}{
        \hfill%
        \usebeamercolor[fg]{page number in head/foot}%
        \usebeamerfont{page number in head/foot}%
        \setbeamertemplate{page number in head/foot}[framenumber]%
        \usebeamertemplate*{page number in head/foot}\kern1em\vskip6pt%
    }
}
\makeatother%

% ======================================
% === ENVIRONMENTS FOR LECTURE NOTES ===
% ======================================
% The amsthm package provides {theorem}, {lemma} and {proof} environments
% \newenvironment{problem}[1][]
%     {\medskip\large \textbf{Problem \!(#1)}:\normalsize \itshape }
%     {\par}
\newenvironment{subproblem}[1][]
    {\normalsize \textbf{Part \!(#1)}:\normalsize \itshape }
    {\par}    
\newenvironment{hint}{\ttfamily\textup{[}}{\textup{]}}

% Don't print section numbers
% \setcounter{secnumdepth}{0}

% Spaced example
\newenvironment{spexample}[1][]{
\par
\vbox\bgroup
\begin{example}[#1]
}{
\end{example}\egroup
}

% Spaced exercise
\newenvironment{spexercise}[1][]{
\par
\vbox\bgroup\begin{exercise}[#1]
}{
\end{exercise}\egroup%
}

% Spaced definition
\newenvironment{spdefinition}[1][]{
\par
\vbox\bgroup
\begin{definition}[#1]
}{
\end{definition}\egroup
}

% Spaced theorem
\newenvironment{sptheorem}[1][]{
\par
\vbox\bgroup
\begin{theorem}[#1]
}{
\end{theorem}\egroup
}

% Spaced remark
\newenvironment{spremark}[1][]{
\par
\vbox\bgroup
\begin{remark}[#1]
}{
\end{remark}\egroup
}

% Spaced reminder
\newenvironment{spreminder}[1][]{
\par
\vbox\bgroup
\begin{reminder}[#1]
}{
\end{reminder}\egroup
}

% Define document colors
\definecolor{blue}{rgb}{0.169, 0.243, 0.714}
\definecolor{orange}{rgb}{1, 0.652, 0}
\definecolor{q0}{rgb}{0.169, 0.243, 0.714}
\definecolor{q1}{rgb}{1, 0.652, 0}
\newcommand{\qcolor}[2]{{\color{q#1}#2}}
